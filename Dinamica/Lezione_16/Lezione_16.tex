\documentclass[12pt, a4paper]{article}

\date{7 Novembre 2019}
\title{Lezione 16}
\author{Dinamica}

\begin{document}

\maketitle

\section{Fairbairn - Parte Seconda}

Il bambino rimuove sia l'aspetto eccitante che quello frustrante della madre, attreverso l'aggressione.

L'Io sviluppa pseudopodi per mantenere il legame con gli oggetti, e l\`i inizia la scissione dell'Io

\paragraph{Anche} le parte dell'Io legao agli oggetti rimossi viene rimossa.

\paragraph{Quindi} si formano una molteplicit\`a di Io, o meglio si creano relazioni oggettuali interne, non solo oggetti interni 

\subsection{Le esperienze che il bambino fa della madre}

\begin{itemize}
    \item La madre non gratificante viene scissa in allettante e deprivante
    \item Entrambi sono cattivi e vengono internalizzati per controllarli e \textbf{preservare la madre reale esterna}
    \item Facendo ciò una parte dell'Io viene scisso legandosi all'oggetto interno
    \item Affetti legati all'oggetto rifiutante: rabbia, tristezza, paura
    \item Affetti legati all'oggetto eccitante: amore non ricambiato, dipendenza dall'altro
    \item La parte dell'Io legata all'oggetto interno rifiutante \`e il sabotatore interno, che giustifica la realt\`a con la svalutazione del s\'e
    \item La parte dell'Io legata all'oggetto interno eccitante \`e la speranza (irrealistica) che tutto passi e diventi bello
\end{itemize}

\paragraph{L'Io centrale} rimuove (aggredisce in gergo Fairbairniano) le parti dell'Io interne, e l'Io anti-libidico aggredisce l'oggetto eccitante e l'Io libidico

\subsection{Le tre parti dell'Io}

\paragraph{L'Io centrale} \`e identificato con l'oggetto disponibile che viene idealizzato.

\paragraph{L'Io libidico} \`e identificato con l'oggetto eccitante, dipende da esso 

\paragraph{L'Io non libidico} \`e identificato con l'oggetto deprivante, svaluta il s\'e, \`e auto-punitivo, \`e il sabotatore interno

\subsection{Fairbairn vs Klein}

\begin{minipage}{.45\textwidth}
    \begin{flushleft}
        \begin{itemize}
            \item \textbf{Klein}
            \item Fantasie sono la base dell'attivit\`a psichica
            \item Oggetti interni anche come fonti del piacere
            \item Oggetti interni sono immagini universali modificate dalla realt\`a
            \item Patologia legata all'istinto di morte, aggressivit\`a eccessiva
        \end{itemize}
    \end{flushleft}
\end{minipage}
\begin{minipage}{.45\textwidth}
    \begin{flushright}
    \begin{itemize}
        \item \textbf{Fairbairn}
        \item Fantasie sono sostitutive, il bambino \`e orientato alla realt\`a
        \item Oggetti interni compensatori
        \item Oggetti interni reali
        \item Patologia derivante da deprivazione materna
    \end{itemize}
    \end{flushright}
\end{minipage}







































































\end{document}
