% Options for packages loaded elsewhere
\PassOptionsToPackage{unicode}{hyperref}
\PassOptionsToPackage{hyphens}{url}
%
\documentclass[
]{article}

\date{1 Ottobre 2019}
\title{Lezione 1}
\author{Dinamica}
\usepackage{lmodern}
\usepackage{amssymb,amsmath}
\usepackage{ifxetex,ifluatex}
\ifnum 0\ifxetex 1\fi\ifluatex 1\fi=0 % if pdftex
  \usepackage[T1]{fontenc}
  \usepackage[utf8]{inputenc}
  \usepackage{textcomp} % provide euro and other symbols
\else % if luatex or xetex
  \usepackage{unicode-math}
  \defaultfontfeatures{Scale=MatchLowercase}
  \defaultfontfeatures[\rmfamily]{Ligatures=TeX,Scale=1}
\fi
% Use upquote if available, for straight quotes in verbatim environments
\IfFileExists{upquote.sty}{\usepackage{upquote}}{}
\IfFileExists{microtype.sty}{% use microtype if available
  \usepackage[]{microtype}
  \UseMicrotypeSet[protrusion]{basicmath} % disable protrusion for tt fonts
}{}
\makeatletter
\@ifundefined{KOMAClassName}{% if non-KOMA class
  \IfFileExists{parskip.sty}{%
    \usepackage{parskip}
  }{% else
    \setlength{\parindent}{0pt}
    \setlength{\parskip}{6pt plus 2pt minus 1pt}}
}{% if KOMA class
  \KOMAoptions{parskip=half}}
\makeatother
\usepackage{xcolor}
\IfFileExists{xurl.sty}{\usepackage{xurl}}{} % add URL line breaks if available
\IfFileExists{bookmark.sty}{\usepackage{bookmark}}{\usepackage{hyperref}}
\hypersetup{
  hidelinks,
  pdfcreator={LaTeX via pandoc}}
\urlstyle{same} % disable monospaced font for URLs
\setlength{\emergencystretch}{3em} % prevent overfull lines
\providecommand{\tightlist}{%
  \setlength{\itemsep}{0pt}\setlength{\parskip}{0pt}}
\setcounter{secnumdepth}{-\maxdimen} % remove section numbering


\begin{document}
\maketitle

\section{Freud}
Manuale è sul sito, lezioni consigliate, si usa il manuale e le lezioni
di Freud ``introduzione alla psicanalisi'', (solo il primo ciclo) e una
serie di capitoli di libri scaricabili di vari autori.\\
Non usare wikipedia.

Nel manuale si fanno tutti i capitoli tranne Jung \textbf{per l'esame
totale}\\
Jung è stato \emph{intellettualizzato}, ovvero non c'è stata
elaborazione successiva Jung è stato scelto come successore di Freud
anche per evitare che la psicanalisi venisse semitizzata, ma Jung ha
scritto una brutta lettera alla Svizzera dicendo che Freud da ebreo non
avrebbe mai capito l'animo ariano.

Studieremo Freud perché costituisce un polo dialettico per tutti i
pensatori successivi. (Tipo teoria dell'attaccamento) La psicanalisi è

\begin{itemize}
\tightlist
\item
  Un metodo d'indagine dell'inconscio
\item
  Un metodo terapeutico
\item
  Un insieme di teorie sul funzionamento della mente
\end{itemize}

Fatti biografici su Freud:

\begin{itemize}
\tightlist
\item
  Nasce a Freiberg nel 1856
\item
  Trasferimento a Vienna nel 1860 per motivazioni culturali e
  soprattutto di tolleranza verso gli ebrei
\item
  Si iscrive a Medicina, studiando sotto Brentano
\end{itemize}

Brentano era un filosofo di cui Freud si appassionò

Brentano non credeva nell'inconscio ma dell'intenzionalità e della
contrapposizione di oggetto e soggetto

Si appassiona al lavoro di un filosofo\\
Inizia a sezionare anguille\\
Essendo ebreo Freud venne fermato nella sua carriera di neurofisiologo
accademico (Goethe), quindi diventa neuropsichiatra, dovendo guadagnare
(vuole sposare la fidanzata)\\
In quel tempo andava di moda l'isteria e va da Charcot a Parigi, un
importante neurologo, e il suo primo importante problema fu `come
distinguere tra i pazienti che avevano attacchi epilettici e gli
isterici'\\
Charcot fu anche il primo a fare notare che l'isteria potesse colpire
anche gli uomini\\
L'isteria si manifestava come disturbi pseudo-neurologici, come cecità,
paralisi, spasmi, che mimavano problemi neurologici ma non erano
supportati da nessun tipo di lesione, inoltre presentavano decorsi
contrastanti con l'anatomia dei nervi.\\
Nell'antichità veniva vista come una migrazione dell'utero nelle donne,
nel medioevo era segno di eresia e stregoneria.

Freud si fa notare da Charcot promettendo di tradurre la sua opera in
tedesco\\
Viene assegnato ad un reparto di neurologia e neuropsichiatria
infantile, dove inizia a fare le sue osservazioni\\
Successivamente apre il suo studio a Vienna, dove visita principalmente
signorine isteriche dell'altra categorie.\\
Tené lezioni a Vienna senza però essere pagato.\\
Freud diceva di avere più paura dei cattolici che dei nazisti,
sottovalutando la minaccia.\\
Si spaventò molto quando la Gestapo arrestò Anna, sua figlia.\\
Decise quindi di emigrare a Londra, fuggendo grazie all'aiuto di una sua
paziente, la principessa Bonapart.\\
Porta con sé tutta la sua famiglia, il cane e la tata, tranne 4 sorelle
che si rifiutano e muoiono nei campi.\\
Prima di andarsene la Gestapo gli fa visita e c'è l'aneddoto della
lettera di raccomandazione.

Si apre a Londra una disputa tra sua figlia e qualcun altro, lui ha il
cuore spezzato per il trasloco e muore l'anno successivo.

Aveva un amico dipendente della morfina, e lo aiutò facendogli assumere
cocaina, uccidendolo con la dipendenza.

I contributi di Freud neurologo:

\begin{itemize}
\tightlist
\item
  Notò la somiglianza tra le cellule nervose tra vertebrati e
  invertebrati, trovando differenze solo nel numero di cellule e
  connessioni
\item
  Scopre il neurone indipendentemente da Cajal
\item
  Scrive un trattato sulle agnosie e le nomina come tali
\item
  Suggerisce l'uso della cocaina come anestetico
\item
  Scopre un metodo di colorazione con l'argento dei neuroni
\end{itemize}

Nel 1895 scrive Progetto per una psicologia scientifica, basato in parte
sui suoi lavori di neurologia, sostiene che ci siano correlazioni tra
strutture cerebrali e funzioni psichiche, partorendo il concetto di
Energia Psichica, sostenendo che arrivi dall'esterno e dall'interno,
regolata attraverso i principi dell'inerzia e di costanza,
sostanzialmente omeostasi.

Lo scopo di questo progetto era di trovare correlazione tra processi
psichici e la circolazione di energia all'interno di strutture che noi
oggi chiamiamo neuroni.\\
Postula l'esistenza di 3 tipi di neuroni:

\begin{itemize}
\tightlist
\item
  Fi, ricevono energia dal mondo esterno e la scaricano immediatamente
  (percezione)
\item
  Omega, ricevono energia dai Fi o dal corpo e trasformano la quantità
  di energia in un tipo di qualità (Memoria)
\item
  Dell'Io, postula l'esistenza di un Io che serve a inibire
  l'eccitazione, ci permette di distinguere il ricordo dalla realtà
  (principio di realtà)
\end{itemize}

Tandell o Caddel ha difeso fortemente Freud e la psicanalisi, sostenendo
che fosse in grado di cambiare la struttura neurale, e che la memoria
implicasse un incremento di permeabilità nelle sinapsi, e che
l'apprendimento ne fosse responsabile. Distingue in circuiti modulatori
(dopaminergici, colinergici, serotoninergici) e quelli
percettivi/motori.\\
La sua idea è ripresa dai neuropsicanalisti (Damasio).

La sua prima pubblicazione è un saggio scritto assieme al mentore Breuer
sull'isteria.\\
Breuer e Goethe incoraggiarono Freud ad andare a Parigi.

Immagine di Charcot in aula (riflesso di Babinski, contrazione alluce)
che sopprime (o induce) i sintomi attraverso l'ipnosi.\\
Charcot ipotizzò che la suscettibilità ad essere ipnotizzati fosse un
sintomo dell'isteria, causato dallo shock nervoso che provoca quindi
autoipnosi)\\
È stato il primo a dichiarare che l'isteria fosse solo causata dalla
psiche, senza eziologia neuroanatomica.\\
Charcot fece fare fotografie alle pazienti che mostravano sintomi
epilettici, sessualizzanti, deliranti.\\
Prima dell'ipnosi i sintomi isterici venivano trattati con
l'elettrostimolazione del clitoride, o con massaggi e poi vibratori
(inducendo orgasmi).

Freud negli `Studi sull'isteria' presenta una serie di casi, tra i quali
il più eclatante è quello di Anna O., che era una paziente di Breuer.\\
Era una signorina perbene con una fervente immaginazione con una vita
molto monotona che si produceva dei sogni ad occhi aperti che chiamava
il suo \emph{teatro privato}, nel 1980 si ammala il padre, persona al
quale era molto legata, e lei passò sera dopo sera nella sua stanza a
vegliare su di lui, cominciando pian piano ad avvertire una sorta di
debolezza fisica (periodo dell'incubazione).\\
Un anno dopo cominciò a manifestare una psicosi manifesta (paralisi,
contrazioni, agnosia, alessie), si crea una personalità triste e una
agitata che allucina serpenti neri.\\
Breuer va a casa sua e la ipnotizza, facendosi raccontare i sogni ad
occhi aperti. Ad un certo punto lei smette di parlare tedesco e parla
solo inglese, e iniza a chiamare le sedute `la cura della parola'.\\
Quando muore il padre, Anna riconosce solo Breuer. Dall'81 all'82
compaiono le due personalità, una giornaliera e una serale, e ad una
certa inizia a scrivere sul suo diario eventi accaduti esattamente un
anno prima.\\
Parlando del primo sintomo che aveva avuto, il sintomo scomparve (sempre
sotto ipnosi), lo `raccontava via'.\\
Breuer ad una certa, quando Anna O. sta meglio, viene richiamato da lei
mentre stava avendo un parto isterico, dicendo che il figlio era di
Breuer, allora lui si spaventò e andò in seconda luna di miele a
Venezia.\\
Lei finì in una clinica e dopo qualche anno diventò una bravissima
assistente sociale.

Coniò le parole

\begin{itemize}
\tightlist
\item
  Chimney sweeping
\item
  Cura della parola
\end{itemize}

Perché i sintomi sparivano parlandone? Il trauma creerebbe una scarica
di energia che nello stato psichico non si riusciva a gestire, e
parlarne crea un `ponte'.\\
Anna O. non riusciva per esempio più a deglutire, che derivava dalla
visione del cagnolino della dama di compagnia (che lei odiava) che
beveva nella stanza di lei.

Trauma -\textgreater{} energia -\textgreater{} rimozione dalla coscienza
del contenuto ideativo, ma non dell'energia affettiva, che viene
convertita in -\textgreater{} sintomo fisico simbolico

L'attenzione va quindi posta sia sull'evento che sulla sensibilità della
persona, che si trova in uno stato autoipnotico quindi incapace di
fuggire.\\
La cura consiste nella scarica catartica nel momento in cui rivivo il
trauma, e attraverso la presa di coscienza dell'inconscio.

L'ipnosi è quindi l'ignoranza di qualcosa che abbiamo rimosso, ponendolo
nell'inconscio. Il sintomo si esprime in maniera simbolica, come
compromesso tra un desiderio e una difesa.

\end{document}
