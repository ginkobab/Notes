\documentclass[12pt, a4paper]{article}

\date{12 Novembre 2019}
\title{Lezione 18}
\author{Dinamica}

\begin{document}

\maketitle

\section{Matte-Blanco - Parte Seconda}

\subsection{L'infinito}

\begin{quote}
    \emph{Ogni qual volta una parte equivale al tutto: ci troviamo di fronte ad un insieme infinito per definizione...}
    \begin{flushright}
        Canter-Dedekind
    \end{flushright}
\end{quote}

Quando si applica la simmetria scompaioni:

\begin{itemize}
    \item Spazio-tempo
    \item Distinzioni tra parte e tutto
    \item Le distinzioni tra individui, cose singole
    \item Principio di non-contraddizione
\end{itemize}
Quindi al pensiero differenziato o diviso si sostituisce una tendenza al \textbf{omo\-ge\-neità}

Ma questa si appoggia/parassita sulla logica aristotelica, altrimenti sa\-reb\-be il caos;

\paragraph{Intreccio tra le due logiche}  
\begin{quote}
    \emph{Nel mezzo della struttura della logica bivalente o aristotelica, il principio di simmetria fa la sua apparizione in certi punti, e, come un potente acido, dissolve ogni logica a portata di mano\ldots}
    \begin{flushright}
        Matte-Blanco
    \end{flushright}
\end{quote}

\begin{quote}
    \emph{Pensare (cioè eterogenizzare) l'omo\-ge\-nei\-tà significa accostarsi in termini di divisione \ldots a qualcosa che in sé, cioè nella sua essenza o natura, è impensabile o indivisible: una forma di traduzione \ldots o di pensicchiare. Se ci si potesse porre ``nel mezzo'' dell'omo\-ge\-nei\-tà ci renderemmo conto che essa non è semplicemente individsa, e non solo indivisibile da fatto ma anche \textbf{essenzialmente invidivisibile}.} 

\emph{Il guaio è, tuttavia, che
noi siamo anche eterogenei e il massimo che possiamo fare è riuscire a imprigionare l'omogeneità nel mezzo di una struttura bi-logica\ldots!}
    \begin{flushright}
        Matte-Blanco
    \end{flushright}
\end{quote} 
Se ci troviamo sempre di fronte a intrecci bi-logici tra simmetria e asimmetria, e se questi caratterizzano anche la nostra essenza come esseri umani, allora dobbiamo pensare una definizione diversa per essere umano.

\paragraph{La mente}  è paragonata a tante borse di simmetria rivestite da strati più o meno spesso di asimmetria.

La proporzione tra simmetria e asimmetria può variare indefinitamente da individuo a individuo, e generalmente la cultura influenza la parte asimmetrica della mente.

\paragraph{La logica simmetrica}  si usa anche nella vita quotidiana: equivalenze, identità e similitudin  sono relazioni simmetriche $\rightarrow$ Il principio di simmetria opera nel pensiero logico.

\paragraph{Se la simmetrizzazione sconfina }  emerge l'inconscio. Se mancano i contenitori delle sacche tutta la coscienza diventa simmetrica, quindi caotica.\\
È una rivisitazione del concetto di realtà oggettiva e soggettiva.

\paragraph{Esempi:}  

\begin{itemize}
    \item Non ricordo se l'hai detto tu o io. \emph{Simmetrizzazione del sé}
    \item Un paziente schizofrenico, vede una porta che si sta per aprire, si spaventa e urla: ``Gli animali mi mangiano!''. \emph{Simmetrizzazione di porte e bocche}
    \item Se il capo è percepito come il padre pericoloso: allora capo e padre appartengono alla stessa classe, di cui la funzione proposizionale è la pericolosità.
\end{itemize}

\paragraph{Esiste un continuum} tra livelli di simmetria, e solo nel più profondo non c'è nulla di cosciente.

\subsection{Le emozioni}

Ogni pensiero veicola un'emozione, pensieri aemotivi contengono difese o censura.

\paragraph{Distingue tra:}  
\begin{itemize}
    \item Componente sensazione-sentimento, collegate al corpo, per essere colta dalla coscienza deve essere rivestita da spazio e tempo
    \item Aspetto di pensiero: legata all'oggetto e all'attività cognitiva con cui si relaziona l'oggetto
\end{itemize}

\paragraph{Le emozioni primitive implicano:}  
\begin{itemize}
    \item Generalizzazione delle caaratteristiche attribuite all'oggetto
    \item Massimizzazione della grandezza di queste caratteristiche 
    \item Di conseguenza irradiazione dall'oggetto concreto a tutti gli altri
    \item Quando vediamo le cose in modo emozionale, identifichiamo l'individuo con la classe a cui appartiene e gli attribuiamo tutte le potenzialità comprese nella funzione proposizionale o enunciato aperto che definisce la classe.
    \item L'emozione non conosce individui ma solo classi e funzioni proposizionali
\end{itemize}
\paragraph{Il terapeuta aiuta} il paziente a contenere l'emozione introducendo il pensiero.

\subsection{Forme di conoscenza}

\begin{itemize}
    \item Studio asimmetrico di un fenomeno: luce della comprensione che introduce differenziazioni
    \item La simmetria implica la fusione
    \item La coscienza non può contenere l'infinito     
    \item Chi sperimenta un emozione: ha una conoscenza completa:
\end{itemize}

\begin{quote}
    \emph{Perché conooscenza ed essere sono la stessa cosa. Non è la co\-no\-scenza di uno spettatore ma la conoscenza inerente all'essere. Per l'essere simmetrico -non è buio- ma la totalità della luce. Al contrario della conoscenza asimmetrica, è conoscenza senza parti}
    \begin{flushright}
        Matte-Blanco
    \end{flushright}
\end{quote}


\subsection{Caso di MB}

\begin{quote}
    \emph{Per impegni presi nella consegna di un lavoro, dovetti interrom\-pere la mia attività clinica per dieci giorni. Quando uno dei miei analizzandi venne alla prima seduta dopo il periodi di interruzione, cominciò a parlare del papa. Lo criticò per una recente lettera in cui si era riferito ai comunisti, condannandoli, e lo aveva chiamato un delinquente. Si era reso conto inoltre che le seudte analifiche non servivano in senso profondo ma gli
    permettevano solo di esprimere i suoi sentimenti e problemi personali. Disse quindi che sarebbe dovuto andare ad un seminario e sentiva che sarebbe stato molto noioso, decise di andare quindi ad uno strip-tease, ma quando si rese conto dell'oscenità se ne andò. Disse anche che in questo periodo si era lasciato viziare da sua moglie, che era stata molto dolce, la percepisce come madre.}
    \emph{A questo punto diedi un interpretazione: io ero il papa irresponsabile e delinquente. Andare allo strip-tease stava rifiutando me come professore e cercava di conoscermi come madre nelll'intimità dello spogliarsi.}
    \emph{Il paziente si ricorda con vergogna di aver spiato la madre spogliarsi da piccolo}
\end{quote}

\paragraph{Traduzione/Dispiegamento}  
\begin{itemize}
    \item Interrozione: evoca un emozione ed espressione del pensiero emozionale
    \item Interruzione delle sedute = rifiuto
    \item Papa=analista
    \item 2 donno=madre che rifiuta per dare attenzioni alla sorella
\end{itemize}

\paragraph{La simmetria}  tratta emozionalmente tutte le sottoclassi come identiche alla classe più ampia.

Tratta gli indiviui contenuti omogeneamente.

\paragraph{Al contempo }  ci sono asimmetrie:
\begin{itemize}
    \item Rabbia con papa
    \item Indifferenza verso analista
    \item Disapprovazione delle sopogliarelliste
\end{itemize}

\paragraph{A livello profondo}  sono tutti uguali, la classe primitiva è la madre che rifiuta di nutrire, importante per l'esperienza del paziente.

\subsection{Implicazioni terapeutiche}

Differenza tra materiale inconscio rimosso e materiale coscienze nella \textbf{struttura}












\end{document}
