% Options for packages loaded elsewhere
\PassOptionsToPackage{unicode}{hyperref}
\PassOptionsToPackage{hyphens}{url}
%
\documentclass[
]{article}
\date{8 Ottobre 2019}
\title{Lezione 4}
\author{Dinamica}
\usepackage{lmodern}
\usepackage{amssymb,amsmath}
\usepackage{ifxetex,ifluatex}
\ifnum 0\ifxetex 1\fi\ifluatex 1\fi=0 % if pdftex
  \usepackage[T1]{fontenc}
  \usepackage[utf8]{inputenc}
  \usepackage{textcomp} % provide euro and other symbols
\else % if luatex or xetex
  \usepackage{unicode-math}
  \defaultfontfeatures{Scale=MatchLowercase}
  \defaultfontfeatures[\rmfamily]{Ligatures=TeX,Scale=1}
\fi
% Use upquote if available, for straight quotes in verbatim environments
\IfFileExists{upquote.sty}{\usepackage{upquote}}{}
\IfFileExists{microtype.sty}{% use microtype if available
  \usepackage[]{microtype}
  \UseMicrotypeSet[protrusion]{basicmath} % disable protrusion for tt fonts
}{}
\makeatletter
\@ifundefined{KOMAClassName}{% if non-KOMA class
  \IfFileExists{parskip.sty}{%
    \usepackage{parskip}
  }{% else
    \setlength{\parindent}{0pt}
    \setlength{\parskip}{6pt plus 2pt minus 1pt}}
}{% if KOMA class
  \KOMAoptions{parskip=half}}
\makeatother
\usepackage{xcolor}
\IfFileExists{xurl.sty}{\usepackage{xurl}}{} % add URL line breaks if available
\IfFileExists{bookmark.sty}{\usepackage{bookmark}}{\usepackage{hyperref}}
\hypersetup{
  hidelinks,
  pdfcreator={LaTeX via pandoc}}
\urlstyle{same} % disable monospaced font for URLs
\setlength{\emergencystretch}{3em} % prevent overfull lines
\providecommand{\tightlist}{%
  \setlength{\itemsep}{0pt}\setlength{\parskip}{0pt}}
\setcounter{secnumdepth}{-\maxdimen} % remove section numbering


\begin{document}

\maketitle
\section{Freud - Parte Quarta}
Freud descrive alcune caratteristiche di funzionamento dei
\textbf{processi primari}:

\begin{itemize}
\tightlist
\item
  Spostamento
\item
  Condensazione
\item
  Assenza di mutua contraddizione
\item
  Assenza di successione temporale, \emph{l'inconscio è immortale,
  nessun contenuto decade}
\item
  La realtà esterna può essere sostituita dalla realtà interna
\end{itemize}

Se siamo molto sotto stress o drogati possiamo regredire a principio del
piacere.\\
Il principio di realtà si sviluppo grazie al principio dell'io.

\textbf{La seconda topica} - \textbf{Io, Es, Super-io}

\textbf{Es} è completamente inconscio, significa in tedesco ``esso'',
non è riconosciuto come proprio (non è parte dell'Io)\\
è dove vengono relegate le pulsioni, avvicinabile solo attraverso
analogie, si occupa del soddisfacimento dei piaceri, e possiamo
immagginarlo destrutturato e inferirlo attraverso gli effetti sull'Io

\textbf{Io} si sviluppa separandosi dall'Es per effetto del principio di
realtà. Lo strato più esterno dell'Io è a contatto con il mondo esterno.
\textbf{Anche l'Io ha una parte inconscia} (infatti i meccanismi di
difesa come rimozione e sublimazione sono inconsci). Per Freud infatti
la maggior parte della psiche è inconscia.\\
L'io dovrà poi mediare tra Es e Super-Io.\\
Nella parte cosciente dell'Io opera il principio di realtà.\\
Una delle funzioni dell'Io è la registrazione dell'\emph{angoscia
segnale}, ovvero un segno di invasione della coscienza da parte dei
contenuti inconsci, e fare sì che si attivino i meccanismi di difesa.\\
L'Io è concepito coordinato nella parte cosciente, mentre nella parte
incosciente troviamo la censura onirica e i processi difensivi. È
descritto come quella parte dell'Es modificata dal mondo esterno, il
paladino della Ragione.\\
L'Io media tra Es, Super-io e esigenze della realtà, si dice che abbia
infatti 3 padroni. È il polo difensivo della mente.

\textbf{Super-Io} si forma nella risoluzione del conflitto Edipico,
grazie all'identificazione con le caratteristiche parentali e si
manifesta con vergogna, senso di colpa e senso di inferiorità.\\
Un Super-io arcaico è patologico, mentre uno psicopatico ha un Super-io
sottosviluppato.\\
È un'istanza vigilante che punisce l"io con vergogna, colpa e senso di
inferiorità. Nasce con dei meccanismi di difesa, l'identificazione e
l'introiezione \textbf{del Super-Io} del genitore.

Excursus a caso: La proiezione è un meccanismo di difesa che fa sì che
tutto ciò di inaccettabile in me lo proietto su un oggetto esterno

L'analogia Freudiana è con cavallo e cavaliere (?) Es è il cavallo, Io
il cavaliere. Normalmente il cavaliere guida e il cavallo porta, ma un
cavallo troppo forte potrebbe decidere da sé la direzione.

Freud introduce l'idea della molteplicità della nostra mente, composta
di diverse antropomorfizzazioni di nostre parti distinte.

Progressione evolutiva delle relazioni con l'oggetto:

\begin{enumerate}
\def\labelenumi{\arabic{enumi}.}
\tightlist
\item
  Stadio autoerotico
\item
  Narcisismo primario (libido investe l'io, onnipotenza e perfezione
  illusoria)
\item
  Amore oggettuale (prima madre come oggetto anaclitico)
\item
  Oggetti omosessuali (rispecchiano il sé)
\item
  Oggetto eterosessuale
\end{enumerate}

Narcisismo primario: Non esistono ancora oggetti (teorizzato per via
della concezione ormai superata che il bambino abbia una percezione del
mondo esterno estremamente confusionaria)

Freud deduce l'esistenza di questo stadio grazie allo studio dei
pazienti psicotici, che mostrano questo tipo di struttura mentale, che
secondo Freud è dovuta a regressione, ma se stanno regredendo al passato
significa che nel passato ciò è già accaduto.

Psicosi significa perdere contatto con la realtà, è divisa in:

\begin{itemize}
\tightlist
\item
  Schizofrenia paranoica, paziente non allucina ma ha deliri. Non hanno
  coscienza di essere malati.
\item
  Maniaci depressivi (bipolari), fasi di depressione estremamente grave
  alternati a stati maniacali, parlano e fanno tutto velocissimi, hanno
  deliri di onnipotenza.
\end{itemize}

Per Freud le psicosi sono un \emph{ritiro della libido dalla realtà},
regredendo al narcisismo detto secondario in questo caso.\\
Più sono gravi, più sono regrediti.

Passaggio dal narcisismo primario agli stati successivi: metafora
dell'ameba che emette pseudopodi, che possono essere ritirati e
reinglobati.

Per Freud esistono per ogni processo varianti sane e patologiche, ad
esempio il lutto è la variante sana del ritiro dalla libido, mentre la
psicosi quella patologica.

\textbf{Teoria della tecnica}

Dire qualcosa a qualcuno non cambia il subconscio, quindi per
modificarlo bisogna:

\begin{itemize}
\tightlist
\item
  Trovare rimozioni
\item
  Formare un'alleanza con l'io per eliminare le resistenze
\item
  Effettuare un transfert per passare i metodi positivi di coping con le
  pulsioni e l'inconscio del medico, e perché ciò succeda il medico va
  investito di libido.\\
  Freud pensava che non potendo investire il medico gli psicotici non
  potessero andare in analisi, concezione riveduta e superata
  successivamente.
\item
  Dimostrare attraverso il transfert che i sentimenti del paziente non
  sono provocati dalla sua situazione presente, ma sono una coazione a
  ripetere, e ciò può essere possibile solo trasformando la coazione in
  ricordo e lavorandoci così sopra (paziente M?)
\end{itemize}

\end{document}
