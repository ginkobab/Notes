\documentclass[12pt, a4paper]{article}

\renewcommand{\labelitemii}{$\star$}
\date{22 Ottobre 2019}
\title{Lezione 8}
\author{Dinamica}

\begin{document}
\maketitle

\section{Bion}

\emph{Riflessioni sul significato di ``pensare''}

\paragraph{Biografia} Inglese nato in India, a 8 anni viene mandato in collegio. Studia storia e poi medicina. Prima fronte prima guerra mondiale a 17 anni, poi psichiatra nella seconda, gli viene assegnato il compito di trovare terapie di gruppo per i soldati che non volevano combattere, affetti da ptsd. Si avvicina alla psicanalisi attraverso Klein. Studia i gruppi, gli schizofrenici, si interessa all'eipstemologia, ovvero ``cosa significa pensare
all'interno della psicanalisi''. Passa 2 anni in California, ha un forte legame con Roma per la seconda moglie.\\
\smallskip
\paragraph{Concetti base}
\begin{itemize}
    \item Corpo fonte primaria di sensazione che si trasforma in emozioni, che vanno \emph{organizzate} attraverso il pensiero.
    \item Bion si focalizza sulle \emph{funzioni}.
    \item Apprendere dall'esperienza implica \emph{tollerare} l'angoscia e la frustrazione derivate dall'incertezza. L'alternativa \`e saturare la non conoscenza con conoscenza fittizia.
    \item Bion sostiene che esistano pensieri indipendentemente da i pensatori (i pensieri sono sempre intesi come affettivi).
    \item Nell'accogliere  pensieri dobbiamo essere disposti a modificarci con essi.
    \item Pensare \`e un metodo per affrontare i pensieri evitandoli o trasformandoli.
\end{itemize}

\emph{Come imparano il bambino e l'analizzando a riconoscere cosa \`e verit\`a e cosa invece conoscenza fittizia?}

\paragraph{I Modelli:} Bion astrae dalla clinica, e crea modelli sulla tecnica e lo sviluppo.

\paragraph{I gruppi} \`E fondamentalmente gestaltista. Sostiene che ci possano essere parti psicotiche nelle dinamiche di gruppo.
Definisce il gruppo come uno \textbf{stato mentale}

\subparagraph{I nuclei psicotici} nei ``sani'' sono inaccessibili.
Pensa un sistema proto-mentale, una matrice indifferenziata transindividuale, dalla quale emergono emozioni discrete

\paragraph{Distingue} 2 tipi di stati mentali:

\subparagraph{W} Di lavoro specializzata, che utilizzano il processo secondario. Sono come l'Io, razionali, ostacolati da AdB ma non sempre (azienda madre ad es.)

\subparagraph{AdB} Fondati suglli assunti di base, che utilizzano il processo primario. Gli AdB sono di tre tipi, e quando uno \`e attivo gli altri sono latenti. Gli AdB possono essere paragonati alle resistenze, perch\'e ostacolano il processo normale di pensiero. \\ Gli AdB individuati da Bion sono:
\begin{itemize}
    \item Lotta/Fuga
        \begin{itemize}
            \item Gruppi: militari o minacciati.
            \item Emozioni: rabbia e odio
            \item Motivazione: Ricerca del colpevole
        \end{itemize}
    \item Dipendenza
        \begin{itemize}
            \item Gruppi: Chiesa, gruppi con leader forti
            \item Emozioni: colpa, depressione
        \end{itemize}
    \item Accoppiamento
        \begin{itemize}
            \item Gruppi: Nobili, VIP
            \item Motivazione: Ricerca del messia
        \end{itemize}
\end{itemize}


Tendiamo a cercare persone con AdB simili. \\ Gli AdB sembrano onnipotenti e magici, e ci permettono di evitare la frustrazione. Questo concetto \`e ripreso da Hume, secondo il quale la ragione era al servizio dell'emozione. 

Il pensiero \`e intriso di affetti, quindi l'affetto \`e forma di conoscenza. I pensieri per Bion sono prevalentemente inconsci.

\subsection{I Modelli di funzionamento della mente $\alpha$ e $\beta$}

\paragraph{Elementi $\beta$} come emozioni grezze, insconsce, vanno trasformate per essere utilizzate come elementi $\alpha$. Ciò può essere fatto solo se ho la funzione $\alpha$, se non la ho vengono espulsi con IP, come avviene nei bambini e negli schizofrenici.

\paragraph{Fuzione $\alpha$} \`E il risultato della rappresentazione delle percezioni sensoriali. Gli elementi $\alpha$ non sono parte della realt\`a, come i colori. La funzione $\alpha$ crea un ordine nella realt\`a. 

\paragraph{Passaggio} $\beta \rightarrow \alpha$: \\ Per Freud il processo era: 
\begin{enumerate}
    \item Assenza dell'oggetto
    \item Allucinazione
    \item Frustrazione
    \item Processi secondari
\end{enumerate}

Anche per Bion alla nascita si hanno solo elementi $\beta$, e il processo dopo la percezione della \textbf{fame} \`e:
\begin{enumerate}
    \item Se funzione $\alpha$ \textbf{presente}:
    \item Sopportazione dello stimolo
    \item Trasformazione fame in immagine (del seno)
    \item Nel momento in cui appare l'immagine il bambino riconosce che la realt\`a differisce da essa
    \item Il bambino \textbf{pensa} al seno reale
\end{enumerate}
\bigskip
\begin{enumerate}
    \item Se funzione $\alpha$ \textbf{assente}:
    \item Allucinazione
    \item Paura + Fame
    \item IP, produzione di oggetti bizzarri
\end{enumerate}



























\end{document}
