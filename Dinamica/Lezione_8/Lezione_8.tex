\documentclass[12pt, a4paper]{article}

\date{22 Ottobre 2019}
\title{Lezione 8}
\author{Dinamica}

\begin{document}
\thispagestyle{empty}
\maketitle

\section{Bion}

\emph{Riflessioni sul significato di ``pensare''}

\paragraph{Biografia} Inglese nato in India, a 8 anni viene mandato in collegio. Studia storia e poi medicina. Prima fronte prima guerra mondiale a 17 anni, poi psichiatra nella seconda, gli viene assegnato il compito di trovare terapie di gruppo per i soldati che non volevano combattere, affetti da ptsd. Si avvicina alla psicanalisi attraverso Klein. Studia i gruppi, gli schizofrenici, si interessa all'eipstemologia, ovvero ``cosa significa pensare
all'interno della psicanalisi''. Passa 2 anni in California, ha un forte legame con Roma per la seconda moglie.\\
\smallskip
\paragraph{Concetti base}
\begin{itemize}
    \item Corpo fonte primaria di sensazione che si trasforma in emozioni, che vanno \emph{organizzate} attraverso il pensiero.
    \item Bion si focalizza sulle \emph{funzioni}.
    \item Apprendere dall'esperienza implica \emph{tollerare} l'angoscia e la frustrazione derivate dall'incertezza. L'alternativa \`e saturare la non conoscenza con conoscenza fittizia.
    \item Bion sostiene che esistano pensieri indipendentemente da i pensatori (i pensieri sono sempre intesi come affettivi).
    \item Nell'accogliere  pensieri dobbiamo essere disposti a modificarci con essi.
    \item Pensare \`e un metodo per affrontare i pensieri evitandoli o trasformandoli.
\end{itemize}

\emph{Come imparano il bambino e l'analizzando a riconoscere cosa \`e verit\`a e cosa invece conoscenza fittizia?}




































\end{document}
