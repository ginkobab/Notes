% Options for packages loaded elsewhere
\PassOptionsToPackage{unicode}{hyperref}
\PassOptionsToPackage{hyphens}{url}
%
\documentclass[
]{article}
\date{7 Ottobre 2019}
\title{Lezione 3}
\author{Dinamica}
\usepackage{lmodern}
\usepackage{amssymb,amsmath}
\usepackage{ifxetex,ifluatex}
\ifnum 0\ifxetex 1\fi\ifluatex 1\fi=0 % if pdftex
  \usepackage[T1]{fontenc}
  \usepackage[utf8]{inputenc}
  \usepackage{textcomp} % provide euro and other symbols
\else % if luatex or xetex
  \usepackage{unicode-math}
  \defaultfontfeatures{Scale=MatchLowercase}
  \defaultfontfeatures[\rmfamily]{Ligatures=TeX,Scale=1}
\fi
% Use upquote if available, for straight quotes in verbatim environments
\IfFileExists{upquote.sty}{\usepackage{upquote}}{}
\IfFileExists{microtype.sty}{% use microtype if available
  \usepackage[]{microtype}
  \UseMicrotypeSet[protrusion]{basicmath} % disable protrusion for tt fonts
}{}
\makeatletter
\@ifundefined{KOMAClassName}{% if non-KOMA class
  \IfFileExists{parskip.sty}{%
    \usepackage{parskip}
  }{% else
    \setlength{\parindent}{0pt}
    \setlength{\parskip}{6pt plus 2pt minus 1pt}}
}{% if KOMA class
  \KOMAoptions{parskip=half}}
\makeatother
\usepackage{xcolor}
\IfFileExists{xurl.sty}{\usepackage{xurl}}{} % add URL line breaks if available
\IfFileExists{bookmark.sty}{\usepackage{bookmark}}{\usepackage{hyperref}}
\hypersetup{
  hidelinks,
  pdfcreator={LaTeX via pandoc}}
\urlstyle{same} % disable monospaced font for URLs
\usepackage[normalem]{ulem}
% Avoid problems with \sout in headers with hyperref
\pdfstringdefDisableCommands{\renewcommand{\sout}{}}
\setlength{\emergencystretch}{3em} % prevent overfull lines
\providecommand{\tightlist}{%
  \setlength{\itemsep}{0pt}\setlength{\parskip}{0pt}}
\setcounter{secnumdepth}{-\maxdimen} % remove section numbering


\begin{document}
\maketitle
\section{Freud - Parte Terza}

Nella teoria Freudiana le pulsioni sono viste come motivatore principale
dell'azione, e possono essere:

\begin{itemize}
\tightlist
\item
  Sessuali
\item
  Dell'Io
\end{itemize}

Sono viste in contrapposizione, anche se può sembrare strano.
\emph{Forse sono in contrasto per i rischi legati al parto, sopratutto
ai tempi}

Per pulsione intendiamo la rappresentanza psichica di uno stimolo
endosomatico o esosomatico.\\
Le pulsioni sono al limite tra mente e corpo, e permettono alla psiche
di funzionare.

Differiscono per luoghi di generazione e mete.

La riduzione della pulsione è legata a piacere.

Gli stimoli possono provenire dall'esterno o dall'interno, ma da quelli
interni non si può fuggire fisicamente,\\
la mente si struttura quindi per gestire le pulsioni

\textbf{Situazione Prototipo}\\
\textgreater{} Bambino in preda alla fame strilla\\
La madre lo nutre

Secondo Freud il bambino si ricorderà dell'evento, e la volta successiva
che avrà fame il bambino allucinerà il seno (\emph{allucinazione
gratificatoria}), di fatto \emph{confondendo il mondo esterno con quello
interno}.\\
Ma l'allucinazione non riduce la fame, \sout{quindi} il bambino prova
frustrazione, e da questa viene spinto verso la realtà esterna.

Se fosse sufficiente l'allucinazione gratificatoria vivremmo in un mondo
allucinatorio.

\textbf{La fonte pulsionale}:\\
Le diverse fonti delle pulsioni sessuali sono definite come zone
erogene, e sono connesse con funzioni biologiche importanti nel momento
dello sviluppo.

\textbf{L'oggetto della pulsione}:\\
L'oggetto per Freud è la parte più variabile, definita solo dalla
capacità di gratificare.\\
L'assenza dell'oggetto costituisce la condizione iniziale dello
sviluppo.\\
Prima il suo ambiente è privo di \emph{oggetti}, ed è la fase del
\sout{narcisismo primario}

Il bambino ha pulsioni solo parziali? (quando, cosa vuol dire?)

La libido è la fonte principale dell'energia psichica, e nel corso dello
sviluppo si modificano la sua fonte e i suoi oggetti.

Le pulsioni possono essere:

\begin{itemize}
\tightlist
\item
  Soddisfatte
\item
  Represse
\item
  Trasformate
\item
  Sublimate, ovvero trasformate in un atto socialmente desiderabile
\end{itemize}

Per esempio poco tempo fa il fumare era sublimazione della fissazione
orale

La nevrosi è definita come una fissazione o regressione ad uno stato
precedente

La sequenza di sviluppo della libido è determinata filogeneticamente ed
è collettiva.

\begin{itemize}
\item
  Fase orale: Lo sfamarsi fornisce i significati elettivi attorno ai
  quali organizzare il mondo esterno (ovvero gli oggetti). Le funzioni
  connesse sono legate al mangiare, all'espellere (soffiare) e
  all'essere mangiati.
\item
  Fase anale: Ci si arriva tra i 2 e 4 anni, è legata alla funzione
  della defecazione. Tutto è espresso in funzione di \emph{espellere} e
  \emph{trattenere}. Le feci sono viste come preziose (bambini
  arrabbiati non fanno la cacca). In questa fase per esempio avviene la
  `battaglia del controllo sfinterico'. L'analità è legata con la
  tirchiezza, anche analogamente al disturbo ossessivo compulsivo.
\end{itemize}

Nessuno parla più di pulsioni sessuali o di luoghi di origine,
nonostante ciò le osservazioni di Freud non sono affatto\\
I significati della fase fallica sono costruiti attorno al fallo, anche
per le bambine che lo costruiscono i significati in base alla mancanza
del fallo\\
Oggi nessuno pensa che queste fasi siano universali.\\
Dalla fase fallica parte il processo edipico: Il primo oggetto dei
bambini è la mamma. Succede che il bambino comincia ad avere fantasie
sul voler possedere la madre anche sessualmente (ma in senso infantile,
non di sessualità adulta), quindi il bambino si vaneggia dell'attenzione
della madre con il papa, fino a che si sottomette alla superiorità
paterna e teme la castrazione, rinuncia alla madre e si identifica nel
padre.\\
La bambina nella teoria di Freud invece (teoria criticata) si volge
verso il padre, con la convinzione di essere già stata castrata, con la
speranza di potergli dare bambini, e successivamente deve passare da una
sessualità fallica (clitoridea) a una vaginale. La fase si conclude
quando hanno dei rapporti genitali completi?\\
Il tipo di fantasia che il bambino ha durante la fase edipica dipende
anche da come è riuscito a superare le fasi pregenitali, per esempio se
è stato fissato sull'oralità avrà fantasie basate sulla dipendenza.
Domanda: perché si supera una fase? Avviene naturalmente, la domanda è
perché si rimane fissati? Perché trasporto qualcosa, come un eccesso di
frustrazione o di gratificazione.\\
Freud aveva una teoria basata su determinismi complessi, ovvero
un'interazione tra fattori innati e ambientali.\\
Ritornando alle fantasie edipiche, se il bambino è rimasto fissato alla
fase anale (trattengo/espello), si avranno fantasie relative.\\
Edipo si risolve grazie all'angoscia da castrazione, e l'erede
dell'Edipo è il super io.\\
Il bambino si identifica con le proibizioni, la morale di cui il padre è
portatore.\\
Nessuno oggi sostiene l'universalità dell'Edipo, esistono tuttavia
evidenze empiriche che mostrano come i bambini tendono ad avere
comportamenti aggressivi verso genitori dello stesso sesso e affettuosi
verso genitori del sesso opposto.\\
Inoltre all'interno delle teorie psicoanalitiche ci sono state diverse
interpretazioni della teoria Edipica, per esempio che i bambini cadano
nel disturbo solo in caso ci siano stati atti problematici
(comportamenti seduttivi).\\
Gli autori Kleiniani hanno rifolmulato l'edipo in termini di rivalità,
dicendo che il legame tra madre e bambino è duale prima del padre, e con
l'arrivo di questo il bambino deve far fronte per la prima volta ad una
situazione triangolare e si trova a volte escluso.\\
I bambini piccoli egocentrici/narcisisti vorrebbero attenzione costante,
e le attenzioni dedicate al padre (non sessuali) generano rivalità.

Tra 5/6 anni c'è un periodo di latenza, infatti il bambino va a scuola e
compaiono i primi ostacoli alla pubertà, come la vergogna e il pudore.\\
In questa fase la sublimazione è estremamente importante, un esempio
sono gli sport di gruppo maschile (abbracci e sculacciate per
esultazioni come sublimazione di impulsi sessuali e aggressivi).

La formazione reattiva invece è una formazione psicologica in senso
opposto a quello del desiderio represso. Aiuta a controllare impulsi
aggressivi e sessuali.

Il punto di vista dinamico prevede l'esistenza di polarità conflittuali.
Inizialmente abbiamo visto:

\begin{itemize}
\tightlist
\item
  desideri inconsci in conflitto con massa di idee dominanti
\item
  io vs sessuali
\item
  libido vs istinto di morte
\item
  istinto di vita vs istinto di morte
\item
  io vs super io
\end{itemize}

Le persone sono naturalmente bisessuali, e i bambini sono perversi
polimorfi, secondo Freud, ma secondo lui la maturità consiste nell'avere
rapporti genitali che includono per le donne il passaggio dalla maturità
clitoridea a quella vaginale, quindi secondo lui solo l'eterosessualità
è l'unica forma matura di sessualità, e secondo lui l'omosessualità era
legata al narcisismo.\\
Questo è dovuto al contesto culturale, che prima era un reato e poi una
considerato patologico.

Ora nessuno è più d'accordo, esistono condizioni narcisistiche, sia
etero che omosessuali, e una persona in analisi ci va per dei conflitti
interni ma non per passare all'eterosessualità (terapie di conversione
americane con tecniche comportamentiste di aversione lol)

Esiste ancora l'isteria? Si, raramente, si chiama disturbo di
conversione (dell'affetto)

\textbf{Le contrapposizioni delle pulsioni}\\
Le pulsioni dell'io (sete, fame) necessitano di un oggetto, e secondo
Freud sono educabili dalla realtà proprio per questo, quindi qui il
principio del piacere viene sostituito dal principio di realtà.\\
Se non avviene la soddisfazione, resta la pulsione.

Le pulsioni sessuali invece, secondo Freud, inizialmente non hanno
bisogno dell'oggetto ma parassitano su altre funzioni fisiologiche come
la fame, per essere gratificata.\\
Queste pulsioni possono essere gratificate anche dall'autoerotismo
(sempre relative alla fase sessuale, per esempio nell'orale `ciucciare
il dito')

Queste pulsioni libidiche se non vengono soddisfatte inducono angoscia,
la stessa angoscia che può causare nevrosi.\\
Freud indicava come nevrosi attuali delle nevrosi causate
dall'insoddisfazione sessuale, curata con rapporti sessuali.

Nel 1920 Freud passa a proporre la contrapposizione Libido vs
Aggressività, e nel 23 diventano istinto di vita e di morte.\\
Perché introduce la pulsione di morte?

Il primo problema è dove collegare l'aggressività, negli anni '20 aveva
appena sperimentato l'aggressività attraverso la prima guerra
mondiale.\\
Quindi c'erano fattori personali, e poi la necessità di spiegare
compulsioni a ripetere, sadismo, masochismo, ma anche come spiegare il
fatto che la libido a volte rimanga fissata a una meta oppure torna
indietro regredendo.

La prima risposta era che la libido fosse `viscosa', che tendesse a
rimanere appiccicata, ma poi postula che esista una pulsione
contrapposta all'eros.\\
La meta dell'eros è unificare e mantenere.\\
La meta dell'istinto distruttivo è di distruggere e dividere.

È stato molto molto criticata la teorizzazione dell'istinto di morte,
considerata al più un contributo dell'autobiografia Freudiana.

Il punto di vista topico è la strutturazione dell'apparato psichico per
Freud.\\
Inconscio/preconscio/coscienza\\
Es/io/Superio

Parla del luogo della psiche come della rappresentazione delle funzioni.

\textbf{Inconscio} paragonato a grande anticamera dove impulsi
(pulsioni) si muovono come entità, poi c'è un guardiano, che esamina e
censura i singoli impulsi psichici e non gli ammette nel salotto
(coscienza) se non gli vanno a genio.

Gli impulsi nell'anticamera dell'inconscio non sono visibili, ma se si
spingono fino alla soglia e sono stati rimandati indietro vuol dire che
sono stati rimossi, ma anche gli impulsi passati oltre la soglia sono
coscienti solo se possono passare alla coscienza solo se riescono ad
attrarre l'attenzione della coscienza, altrimenti rimangono preconsci.

Il principio di piacere vige a livello di inconscio.\\
Il principio di realtà vige a livello di conscio e preconscio.

Processi primario e secondario. Primario inconscio, secondario conscio e
preconscio.

Il principio primario e del piacere sono inconsci e sono i primi a
regolare la vita psichica.\\
I principi secondario e della realtà si formano più tardi ed agiscono a
livello conscio e preconscio.

I principi del piacere e primario non sono la stessa cosa, sono due modi
di funzionare dell'inconscio.

\textbf{Il processo primario} dal punto di vista topico è situato
nell'inconscio, dal punto di vista energetico è in una situazione in cui
l'energia è libera. Dato che l'energia è libera di spostarsi il processo
primario è descritto anche attraverso i processi di condensazione e
condensamento.\\
Posso infatti saltare di contenuto in contenuto dato che l'energia è
libera, non legata agli oggetti. Inoltre essendo libera, l'energia può
investire il ricordo e le immagini mnestiche per avere le gratificazione
allucinatoria che necessita.

Se io sono in preda del principio del piacere cerco soddisfazione
immediata, quindi utilizzo il processo primario per allucinare e
scaricare.\\
Ed ecco perché io posso soddisfare i miei desideri inconsci nel sogno,
perché lì vige il processo primario. Per certi versi esso è una
gratificazione allucinatoria.

Nel processo secondario l'energia è legata, quindi non posso gratificare
immediatamente le mie pulsioni, devo considerare la realtà, perché gli
oggetti investiti non sono più sostituibili da allucinazioni.\\
Il processo secondario permette il principio di realtà, è prerogativa di
esso.\\
Perché si instaura il processo secondario? Per via della frustrazione.

\end{document}
