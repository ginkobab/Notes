\documentclass[12pt, a4paper]{article}

\date{11 Novembre 2019}
\title{Lezione 17}
\author{Dinamica}

\begin{document}

\maketitle

\section{Fairbarb - Parte Terza}

\subsection{Psicopatologia}

Dobbiamo pensare alla mente di un individuo come ad una molteplicità di Io. La psicopatologia è data dall'internalizzazione degli oggetti cattivi, effettuata per avere l'illusione di controllo e per idealizzare il genitore esterno.
Il bambino acquisisce cos\`i una sicurezza illusoria.

Se un bambino diventerà malato dipende da:

\begin{itemize}
    \item Grado in cui gli oggetti icattivi si sono installati nell'inconscio
    \item Il grado di cattiveria che li caratterizza
    \item Il grado in cui l'io si identifica con gli oggetti internalizzati
    \item La natura e la forza delle difese che proteggono l'Io da questi oggetti
\end{itemize}

Quanto il bambino sia ancora in grado di relazionarsi con gli oggetti dipende da quanto Io centrale è ancora disponibile.

\paragraph{L'utilità clinica}  Ci permette di capire la scelta di oggetti sadici, frustranti ma anche per le dipendenze dagli altri

\paragraph{Il sabotatore interno}  aggredisce sia l'Io libidico che l'oggetto eccitante

\paragraph{La cura}  Non è l'interpretazione ma la \textbf{relazione} tra paziente e analista.
Se la psicopatologia riflette gli effetti di relazioni oggettuali insoddisfacenti sperimentate nei primi anni e perpetuati in forma esagerata nella realtà interna.

L'analista diventa una figura genitoriale, perché il paziente possa riprendere il suo sviluppo emotivo.

\paragraph{La meta}  è di ridurre le scissioni dell'Io (l'analisi è anche sintesi), creare delle brecce nel sistema chiuso del mondo interno, rendendo il mondo accessibile all'influenza del mondo esterno, riducendo anche la persistenza della dipendenza infantile.

\begin{quote}
    \emph{C. è un uomo di mezza età che entrò in analisi per episodi di depressione e ritiro. Il padre era stato attento ma duro, distante e estremamente esigente. La madre era una casalinga molto competente e disponibile, con la quale però non riusciva a entrare in contatto emotivo. Intorno alla madre avvertiva tristezza, la sentiva piangere ma quando usciva dalla stanza sorrideva. Ricordò anche sistuazioni in cui si svegliava di notte perché il padre suonava ballate malinconiche
    al buio del salotto. C. scivolava al piano di sotto e, nascondendosi, ascoltava in silenzio, condividendo in segreto questi rari momenti, molto emotivi, con il padre.}
    
    \emph{La personalità di C. si era formata lungo linee simili a quelle dei suoi genitori: era molto attivo, responsabile e ottimista. Pian piano iniziò a vedere le sue depressioni episodiche, periodi atipici di totale inutilità e disperazione, come legami preziosi con il nucleo emotivo della vita dei suoi genitori, al quale non aveva accesso attraverso le interazioni reali con loro. Sorprendentemente, si sentiva più vicino a loro, più unito alla sua famiglia quando era depresso.}

    \emph{Un immagine onirica ricorrente era un uomo medusa, distrutto, triste, impotente e senza spina dorsale. Questa immagine sembrava riassumere il legame depressivo con i suoi genitori, una tristezza priva di ossatura, senza struttura, perché i legami luttuosi con la loro emotività erano stati scissi e incapsulati, senza parlarne.}
\end{quote}

\section{Ignacio Matte-Blanco}

\paragraph{Nasce} in una famiglia molto rinomata, bisnonno preside\-nte cileno, studia medicina e diventa psichiatra e cos\`i ha il primo incontro con il pensiero di Freud. 

Diventa professore associato di fisiologia a 25 anni. Va a Londra a studiare psicanalisi, diventa collega della Klein, vicino anche alla Hein, dopodiché torna a Santiago, lavora con pazienti schizofrenici, comincia a sviluppare il suo pensiero, ma capisce di aver bisogno di una formazione mate\-matica più estesa.

Studia matematica, molla la moglie, sposa una studentessa, fa 6 figli e si trasferiscono a Roma.

\paragraph{I contemporanei}  pensavano che non avesse senso ascoltare i discorsi degli schizofrenici.

Matte-Blanco parte dalle definizioni dell'inconscio di Freud, ovvero:
\begin{itemize}
    \item Contenuti
    \item Derivati pulsionali
    \item Soggetti alla rimozione
    \item Contenuti sono isomorfi ai contenuti della coscienza
    \item Non ha accesso alla coscienza
\end{itemize}

\begin{quote}
    \emph{Le imperanti regole della logica non hanno alcun peso nell'incon\-scio; esso potrebbe essere chiamato il regno dell'illogico}
    \begin{flushright}
        Freud, 1938
    \end{flushright}
\end{quote}

\subsection{Caratteristiche dell'Inconscio secondo Freud}

\begin{itemize}
    \item Assenza di mutua contraddizione
    \item Spostamento
    \item Condensazione
    \item Assenza di tempo
    \item Sostituzione della realtà esterna con quella interna
\end{itemize}

\paragraph{Se la logica}  è completamente assente, allora tutto può essere uguale a tutto. Esistono quindi principi logici anche se non aristotelici?

\paragraph{Matt-Blanco}  cerca i principi logici che agiscono nei processi di pensiero dei pazienti schizofrenici

\begin{itemize}
    \item L'inconscio è accessibile solo se assoggettato alle leggi della coscienza, ovvero se avviene una traduzione della struttura del materiale inconscio nella struttura del materiale conscio, altrimenti non è contenibile nella coscienza
\end{itemize}

\begin{quote}
    \emph{L'inconscio si rivela attraverso lacune o strappi nel nostro abituale tessuto logico ordinario, ci introduce in un mondo nuovo, a-spaziale ed a-temporale che si trova continuamente dispiegato nelle maglie del nostro pensiero eterogenico e dividente}
    \begin{flushright}
        Bria, 1981
    \end{flushright}
\end{quote}

\paragraph{Due principi}  sono individuati da Matt-Blanco

\begin{enumerate}
    \item \textbf{Il principio di generalizzazione}: rappresenta la logic aclassica formando classi di equivalenza sempre più inclusive. Permette di conoscere oggetti del mondo. 

        Teoricamente esiste una classe che include ogni cosa. Tra tutte le generalizzazioni possibili l'inconscio ne sceglie alcune, saltando da un elemento particolare a classi sempre più ampie, e l'elemento singolo è sempre tratttato come appartenente a classi più ampie, pur conservando tracce dell'elemento da cui è partito, abbiamo a che fare con le \textbf{funzioni proposizionali}: silvia indica se stessa ma necessariamente anche altro.
    \item \textbf{Il principio di simmetria}: nega la logica formale, permette di scambiare elementi della stessa classe. L'inconscio tratta le relazioni asimmetriche come se fossero simmetriche.
\end{enumerate}

\paragraph{Le funzioni proposizionali}  sono le caratteristtiche che definiscono le diverse classi. Sono determinate a volte culturalmente (sessualità) ma la maggior parte sono personali.

\paragraph{Il principio di simmetria implica:}  

\begin{itemize}
    \item Assenza di tempo
    \item Il tutto è incluso in ogni parte
    \item Basta condividere una caratteristica per essere identici
\end{itemize}

\paragraph{Se il principio di generalizzazione}  opera in ambito discreto, ovvero sulla differenza tra membri di una classe, esprime la logica classica bivalente;

Ma se formiamo classi per mezzo del principio di simmetria, ogni elemento della classe si identifica completamente con ogni altro elemento, e lcon la classe stessa: si viene a creare omogeneità all'interno della classe.

\subsection{Riassumendo}

Vedo pazienti che esprimono l'inconscio, intravedo significati e voglio vedere se è possibile formalizzare le caratteristiche del tipo di logica che l'inconscio usa come deviazioni dalla logica aristotelica.

Allora cerco dei principi che possano spiegare tutte le manifestazioni descritte da Freud e al contempo descritte dai pazienti, e \textbf{scopre} attraverso il principio di generalizzazione, e appoggiandosi alla teoria degli insiemi, il principio di simmetria. Questi due principi assieme sembrano riuscire a spiegare entrambi i casi.

Questo significa che grazie a questi due principi è possibile delineare il funzionamento dell'inconscio in maniera più chiara, proponendo una visione innovativa dell'uomo: non è solo il funzionamento conscio e inconscio ad essere caratterizzato dall'eterogeneità indotta dalla logica ambivalente, ma è il nostro modo di essere: possiamo agire in proporzioni diverse \textbf{sentendoci uniti o separati da tutto}, e nella nostra natura viviamo entrambi allo stesso tempo.

\end{document}
