\documentclass[12pt, a4paper]{article}

\date{21 Ottobre 2019}
\title{Lezione 7}
\author{Dinamica}

\begin{document}

\maketitle

\section{Klein - Seconda Parte}

\paragraph{L'identificazione proiettiva} e la proiezione come funzionano?

L'identificazione proiettiva ha effetti sul ricevente, che, secondo i post-kleinani percepisce ciò che viene proeittato. 
Secondo la Klein nell'IP il bambino proietta \emph{una parte di s\'e} sull'altro.

La proiezione invece avviene solo nella mente del bambino/paziente  

\paragraph{Destini} Il bambino può proiettare oggetti interni buoni (escrementi come regalo) perch\'e, se il bambino introieta oggetti cattivi, proietta fuori quelli buoni per ``salvarli''.

A volte introietto oggetti cattivi per controllarli.
\emph{Bisogna equilibrare proiezioni e introiezioni}, altrimenti l'Io si indebolisce, e con esso il contatto con la realt\`a

\paragraph{La madre} e il terapeuta possono modificare le risposte del bambino, \emph{bonificando} l'aggressivit\`a innata. \\
In questo modo rafforza l'Io introiettando l'oggetto positivo.
Se questo percorso riesce per la Klein la vita mentale sar\`a stabile, e non ci saranno psicosi

\paragraph{Caso Clinico} Rachel sognava fiori e persone di cacca, separate (scissione), ma ne voleva l'integrazione, solo che temeva la distruzione dei fiori (oggetto buono).
La sua caregiver era schizofrenica. \\
\emph{L'oggetto cattivo interno \`e percepito come parte di s\'e}

\paragraph{La Posizione Depressiva} Ci si arriva se non c'\`e troppa aggressivit\`a e permette di introiettare l'oggetto buono. \\
Il bambino si rende conto dell'unitariet\`a della figura materna, ovvero oggetto buono e cattivo sono unitari. \\
\underline{Il rischio} e la paura fondamentale \`e di aver distrutto l'oggetto buono aggredendo quello cattivo. \\
Lo stesso vale per gli oggetti interni. \\
Si possono esmprimere queste paure con terrore di avvelenamento e senso di colpa. \\
Dopo il senso di colpa parte il tentativo di riparazione:\\ Angeli nel cielo, negazione della morte, anche attraverso \emph{fantasie di onnipotenza}.\\
Lo sforzo riparatorio cerca di far prevalere l'amore nella relazione ambivalente con l'oggetto (madre).

\textbf{Per superare} la posizione depressiva devo aver fiducia nella riparazione.

\paragraph{La differenza} tra posizione e stato sta nella possibilit\`a della posizione di riattivarsi nel corso della vita.

\paragraph{Se non tollero} l'angoscia depressiva insorge la \emph{difesa maniacale}, ovvero si nega di aver perduto e danneggiato l'oggetto d'amore e quindi si nega la dipendenza dall'altro, attraverso i meccanismi di \textbf{svalutazione}, che \`e l'opposto dell'idealizzazione, e \textbf{controllo onnipotente}.

\paragraph{Si supera} la posizione depressiva introiettando l'oggetto buono, e si supera cos\`i il rischio psicotico.
\\
\bigskip
\\
\fbox{\begin{minipage}{17em}
\paragraph{Gli schizoidi}non hanno capacit\`a di relazione, ma un mondo interno ricco, spesso scienziati teorici, es. Einstein.
\end{minipage}}
\bigskip

\paragraph{Un tipico sogno} della posizione depressiva: la moglie \`e assolutamente buona, i colleghi assolutamente cattivi, il paziente sogna di nutrire i pesci con il sale e riuscire a salvarne solo alcuni.
\textbf{Ricostruisce cos\`i l'ambivalenza interna all'oggetto.}

\paragraph{L'invidia} \`e aggressivit\`a diretta verso gli oggetti buoni, per avere ciò che hanno dentro di loro. Il problema \`e che l'oggetto non \`e pi\`u interiorizzabile una volta aggredito e distrutto.

\paragraph{La gratitudine} \`e antitetica dell'invidia. Ogni forma di deprivazione produce invidia. L'invidia ha lo \textbf{scopo} di negare l'indipendenza.

\paragraph{L'avidit\`a} \`e invece il desiderio di possedere il contenuto di un oggetto senza dare importanza al contenitore.

\paragraph{In una persona sana} l'invidia e l'odio sono transitori. La gratitudine permette generosit\`a.

\subsection{L'edipo per Klein}

Il primo oggetto \`e la madre, il padre \`e posseduto da lei. L'edipo ha a che vedere con il \emph{tollerare la relazione triadica}. Genitori come amanti odiati e amati. La madre \`e ciò che possiede ogni oggetto buono, quindi entrambi i sessi desiderano la madre. La bambina percepisce che sar\`a come lei, in grado di produrre oggetti buoni, e anche il bambino maschio vuole essere come lei (identificazione e invidia)

I post-Kleiniani pensano la crescita sana come capacit\`a di accettare le differenze e i rapporti triadici.

\paragraph{La psicopatologia} Aggressione, invidia e angoscia eccessive creano fissazioni, che ora sono la patologia stessa, non la causa come in Freud.

\subsection{Conclusioni}
Klein come psicologa dell'Es e delle relazioni oggettuali, non pi\`u variabili ma fisse.
\paragraph{Le critiche} 
\begin{itemize}
    \item Teorica del peccato originale (innatismo)
    \item Linguaggio antropomorfizzato
    \item Modalit\`a patomorfe e adultomorfe nella genesi teorica
    \item Contraddizione tra complessit\`a fantasie interne e percezione esterna
\end{itemize}











\end{document}
