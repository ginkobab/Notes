\documentclass[12pt, a4paper]{article}

\date{4 Novembre 2019}
\title{Lezione 13}
\author{Dinamica}

\begin{document}

\maketitle


\section{Winnie - Parte Seconda}

\subsection{La disintegrazione}

La non-integrazione permette di sviluppare il senso di s\'e. La controparte negativa, che si incontra nella patologia, \`e la \textbf{disintegrazione}.

\`E una difesa che scatta quando l'holding materno non \`e adeguato.

\begin{quote}
    \emph{``La disintegrazione può essere grave quanto la non-attendibilit\`a dell'ambiente, ma ha il vantaggio di essere prodotta dal bambino stesso''}
    \begin{flushright}
        Winnie, 1962
    \end{flushright}
\end{quote}

\paragraph{La sensazione} dell'esistenza della madre dura x minuti. Se la madre resta via pi\`u di x minuti, allora la rappresentazione svanisce, e il bambino diventa turbato. Il turbamento viene risanato in x + y minuti, quando arriva la madre, ma se avesse atteso x + y + z minuti il bambino sarebbe traumatizzato. Dopo quel tempo il ritorno della madre non ripara lo stato del bambino, che ha vissuto una \emph{frattura nella continuit\`a di vita}, che ha provocato
un'impensabile angoscia, contro la quale si organizzano ora le difese.

\paragraph{L'impensabilit\`a} del trauma \`e ciò che lo rende inaccessibile alla coscienza, e ne rende impossibile il ricordo.

\subsection{La personalizzazione}

Non possiamo andare per scontato che \emph{psich\'e} e \emph{soma} siano connessi e che la mente abiti nel corpo, dato che ci sono pazienti che si percepiscono all'esterno.

Perch\'e si connettano la madre deve esercitare la funzione di \textbf{handling} del corpo del bambino, che signifca \textbf{non solo toccarlo} e mangeggiarlo, \textbf{ma anche pensarlo come mente in un corpo}, e grazie a ciò il bambino avvia il processo di personalizzazione.

La madre echeggia i bisogni fisici del bambino che quindi entra in contatto con il proprio corpo.

Si creano al contempo un nesso fra psiche e soma e al contempo dei confini.

Essere personalizzati significa anche di poter \emph{vivere i propri istinti}, come la sessualit\`a, senza esserne sopraffatti, ed anche l'aggressivit\`a

\begin{quote}
    \textbf{N.B.} Winnie distingue, a differenza della Klein, aggressivit\`a e distruttivit\`a.
\end{quote}

\subsection{La Realizzazione}

A 6 mesi il bambino comincia ad acquisire il senso di realt\`a, grazie alla madre che gli fornisce una forma semplificata del mondo (vedi Bion e Foregny)

\paragraph{Il processo:} Il bambino \`e eccitato $\rightarrow$ evoca l'oggetto del bisogno $\rightarrow$ la madre gli porge l'oggetto proprio in quel momento

\paragraph{In questo modo} il bambino ha l'illusione di aver creato l'oggetto, e questo inizia ad avvicinare il bambino alla realt\`a, dato che essa \`e sovrapposta alla fantasia.


\begin{quote}
    \textbf{N.B.} L'illusione \`e la forma benigna dell'allucinazione.
\end{quote}
Il bambino che crede di aver creato l'oggetto si sente \emph{onnipotente}, ma ciò \`e necessario in quel momento dello sviluppo, ed \`e paragonata alla creativit\`a.

\paragraph{\`E importantissimo} che la realt\`a sia prevedibile, altrimenti non si crea il senso di realt\`a, che \`e visto come un arricchimento, a differenza di Freud.

\begin{quote}
\emph{``La soggettivit\`a ha infinito valore, ma \`e tanto allarmante e magica che non la si può gustare se non come parallela all'oggettivo''}
    \begin{flushright}
        Winnie
    \end{flushright}
\end{quote}

\subsection{L'ordine cronologico}

Il passaggio dal principio di piacere a quello di realt\`a, legato in Freud alla frustrazione e all'accettazione malvolentieri della realt\`a, per Winnie diventa il passaggio:

\begin{quote}
    \begin{center}
    Dall'oggetto soggettivo  $\rightarrow$ All'oggetto oggettivo \\
    Appercezione $\rightarrow$ Percezione
    \end{center}
\end{quote}
\bigskip
\begin{minipage}{.45\textwidth}
    \begin{flushleft}
        \begin{center}
Bambino\\
$\Downarrow$\\
Onnipotenza allucinatoria\\
$\Downarrow$\\
Funzioni accresciute\\
$\Downarrow$\\
Sperimentazione realt\`a
        \end{center}
    \end{flushleft}
\end{minipage}
\begin{minipage}{.45\textwidth}
    \begin{flushright}
        \begin{center}
Madre\\
$\Downarrow$\\
Preoccupazione materna\\
$\Downarrow$\\
Diminuisce nel tempo\\
$\Downarrow$\\
Graduale fallimento 
        \end{center}
    \end{flushright}
\end{minipage}


\subsection{L'area transazionale}

\`E anche detto spazio potenziale, \`e lo spazio tra la realt\`a interna e quella esterna.

\paragraph{Il primo a discutere} la funzione della copertina di Linus \`e stato Winnie.

La copertina di Linus \`e essenziale nel momento in cui il bambino va a dormire, nel momento in cui la madre non c'\`e, e l'oggetto la rappresenta soggettivamente, e \textbf{allo stesso tempo} \`e oggettivamente esistente, e d\`a al bambino anche l'impressione di controllo, anche se non quanto l'oggetto interno.

\paragraph{L'area transazionale} \`e parte dello sforzo di tutta la vita di mantenere mondo interno ed esterno separati e al contempo interrelati.

\paragraph{L'arte, la religione, il gioco} e ogni esperienza di introspezione che parte da oggetti reali avviene nell'area transazionale.

\paragraph{La creativit\`a} \`e data, secondo Winnie, dalla capacit\`a di connettere mondo interno ed esterno, senza perdere il senso di realt\`a, o facendolo temporaneamente.

\paragraph{L'analisi} avviene nell'area transazionale

\paragraph{Si può usare} l'area transazionale autonomamente solo se l'oggetto interno \`e vivo e reale e sufficientemente buono.

\subsection{Verso l'indipendenza}

Winnie \`e stato uno dei primi autori a sottolineare l'importanza degli stati in cui il bambino non \`e in preda ai bisogni.

\paragraph{Il bambino deve} riuscire a connettere due aspetti dell'interazione con la madre: \textbf{la madre oggetto} e \textbf{la madre ambiente}. 

\subparagraph{La madre oggetto} \`e l'esperienza della madre che il bambino fa' quando \`e in preda ai bisogni (la madre Kleinana).  

    \subparagraph{La madre ambiente} \`e la madre sullo sfondo che \`e in contatto con il bambino in maniera non intrusiva e che ha il compito di proteggerlo dagli urti del mondo esterno (la madre della Psicologia dell'Io)

\end{document}
