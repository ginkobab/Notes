\documentclass[12pt, a4paper]{article}

\date{11 Novembre 2019}
\title{Lezione 17}
\author{Psicobiologia}

\begin{document}

\maketitle

\section{Stimolazione Magnetica Transcranica}

È una tecnica di stimolazione, non di visualizzazione. 

\paragraph{In termini logici}  le tecniche di visualizzazione non permettono di effettuare inferenze causa-effetto

\paragraph{La TMS}  permette di effettuare dei ``danni temporanei'', attraverso la produzione di un campo magnetico generato da una corrente che passa nella bobina. Questo interferisce con i circuiti neurali traducendosi in corrente che interferisce con l'attività sottostante.

Può stimolare solo le regioni corticali e superficiali, con una risoluzione di circa 1 cm sulla corteccia.

\paragraph{La TES}  permette di \emph{modulare} i neuroni, portandoli più vicini alla soglia di scarica attraverso una corrente diretta, tipicamente si usa la tDCS (transcranic Direct Current Stimolation), ma recentemente si sono sviluppate le tACS (transcranic Alternate Current Stimulation) e la tRNS (transcranic Random-Noise Simulation)

\paragraph{I sintomi indotti da TMS:}
\begin{itemize}
    \item Potenziali motori evocati (MEP)
    \item Fosfeni (allucinazione visive per stimolazione V1)
    \item Fosfeni in movimento (stimolazione V5)
    \item Prestazione comportamentale e tempi di reazione (TR)
\end{itemize}

\paragraph{Le coordinate}  sono basate su due sistemi diversi: All.\ e MNI

\paragraph{Intensità}  
\begin{itemize}
    \item Soglie individuali
    \item Intensità fissa
    \item Modelli del campo elettrico indotto
\end{itemize}

\paragraph{Frequenza}  
\begin{itemize}
    \item Inibitoria: 1 Hz o meno
    \item Eccitatoria: 5--20 Hz
\end{itemize}

\paragraph{Protocollo}  
\begin{itemize}
    \item Online: La TMS viene effettuata insieme al compito
    \item Offline: La TMS viene data prima, usata in clinica
\end{itemize}

\paragraph{Effetti clinici}  
\begin{itemize}
    \item Locali: attivazione o inibizione della corteccia stimolata
    \item Effetti sul network
    \item Effetti chimici
\end{itemize}

\paragraph{Sicurezza e Etica}  Vengono pubblicate delel linee guida ogni tot.\ anni, che limitano l'utilizzo a determinati casi e frequenze.

Ci sono possibili complicazioni etiche nell'utilizzo al di fuori dalla clinica, per esempio se voglio utilizzare la tDCS per aumentare le capacità cognitive di un cervello sano.

\section{Cambiamenti cognitivi nell'invecchiamento}

\begin{itemize}
    \item I primi studi iniziano negli anni '40
    \item Sono connessi al QI
    \item Metodi trasversali
    \item Inizialmente concezione di invecchiamento lineare
    \item Metafora della collina
\end{itemize}

\paragraph{Ipotesi del massimo adolescenziale}  Lo sviluppo cognitivo si arresta a fine adolescenza.

\paragraph{LSDP}  (Life Span Development Theory Psychology) introduce il metodo longitudinale, considerando diversi aspetti (e.g.\ emozionali).

Sostiene che lo sviluppo duri tutta la vita, alcune capacità si perdono, altre migliorano, altre rimangono costanti. È quindi un processo multidimensionale.\\
Considera aspetti:
\begin{itemize}
    \item Biologici, legati all'età
    \item Ambientali, come la scolarità
    \item Interazione tra loro
\end{itemize}

\paragraph{Cosa succede nell'invecchiamento:}
\begin{itemize}
    \item L'intelligenza fluida (capacità operazionali) decadono
    \item Alcune capacità di riserva possono svilupparsi per compensare
    \item La bilancia tra guadagni e perdite è sempre meno positivo
\end{itemize}

\subsection{Modello bifattoriale dell'intelligenza di Cattell} 


\begin{minipage}[t]{.45\textwidth}
\textbf{Intelligenza fluida}\\
Permette di adattarsi a situazioni nuove, a nuovi problemi. Valutata con prove che si basano sul ragionamento e sulla scoperta di leggi e comprensione di relazioni. Declina prima.
\end{minipage} 
\begin{minipage}[t]{.47\textwidth} 
\textbf{Intelligenza cristallizzata}\\
È la conoscenza nozionistica, si basa sulla cultura e sulle esperienze personali. Declina più tardi.
\end{minipage}


\paragraph{Approccio analitico/locale} Assume che la prestazione cognitiva si basi su processi concettualmente distinti

\paragraph{Approccio globale/macro}  Sostiene che l'invecchiamento sia il risultato di una modificazione nelle risorse mentali a disposizione, individuando un numero limitato di meccanismi (memoria di lavoro, inibizione cognitiva, velocità di elaborazione).

\paragraph{Processi automatici e controllati}  quelli controllati decadono maggiormente

\paragraph{Decadono anche i processi percettivi}  
\end{document}
