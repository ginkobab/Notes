\lecture{10}{Tue 24 Mar 2020 10:06}{La Sindrome Maniacale}

\section{La Sindrome Maniacale}
E' una condizione di completo benessere, abnorme stima di sé, superficialità nella valutazione dei rischi, visione eccessivamente ottimistica.
\paragraph{Alterazione dell'Umore}  
\begin{itemize}
	\item Umore euforico persistentemente e immotivatamente elevato e/o irritabile
	\item Sensazioni di pienezza, di potenza, di gioia immotivata
	\item Senso di perfetta e magica sintonia con il mondo circostante
	\item Se contrastato, perde facilmente la pazienza, diventa ostile e irascibile
	\item Repentini cambiamenti di umore
\end{itemize}
\paragraph{Alterazione dell'Ideazione}  
\begin{itemize}
	\item Stima ipertrofica di sé e delle proprie capacità
	\item Il flusso del pensiero è accelerato e può giungere fino alla fuga delle idee
	\item Facilità allo scherzo, al motteggio, al riso
	\item Contatto interpersonale facile e immediato, poi intrusivo
	\item Attento ai particolari e manchevolezze nelle persone che lo circondano, che sottolinea con ironia o insistenza
\end{itemize}
\paragraph{Alterazione del Comportamento}  
\begin{itemize}
	\item Spinta ad agire difficilmente controllabile dall'esterno
	\item Manca la valutazione delle conseguenze delle proprie azzioni
	\item Esibizionismo e disinibizione psicomotoria
\end{itemize}

\section{I disturbi dell'umore}
\paragraph{I disturbi unipolari}  si caratterizzano per \emph{l'assenza di episodi Maniacali o Ipomaniacali in anamnesi} 
\paragraph{I disturbi bipolari}  implicano la presenza di Episodi Maniacali o Ipomaniacali, solitamente accompagnati dalla presenza di Episodi Depressivi Maggiori
\medskip\\
\includegraphics[width=\linewidth]{./images/image19}
\medskip\\
\section{Episodi di alterazione dell'umore}
\paragraph{Episodio depressivo maggiore}  
\begin{itemize}
	\item \textbf{A.} 5 o più dei seguenti sintomi presenti per almeno 2 settimane e rappresentano un cambiamento rispetto al funzionamento precedente. Deve essere presente almeno il sintomo 1 o 2
		\begin{enumerate}
			\item Umore depresso per la maggior parte del giorno, quasi tutti i giorni
			\item Marcata diminuzione di interesse o piacere per tutte, o quasi tutte le attività per la maggior parte del giorno, quasi tutti i giorni
			\item Perdita o aumento di peso
			\item Insonnia o ipersonnia
			\item Agitazione o rallentamento psicomotori quasi tutti i giorni (deve essere osservabile da altri)
			\item Affaticabilità o mancanza di energia quasi tutti i giorni
			\item Sentimenti di autosvalutazione o di colpa eccessivi o inappropriati
			\item Ridotta capacità di pensare o concentrarsi, o indecisione
			\item Pensieri ricorrenti di morte, ricorrente ideazione o tentativi di suicidi
		\end{enumerate}
	\item \textbf{B.}  I sintomi causano disagio clinicamente significativo o compromissione del funzionamento in ambito sociale, lavorativo o in altre aree importanti
	\item \textbf{C.} L'episodio non è attribuibile agli effetti fisiologici di una sostanza o a un'altra condizione medica generale
\end{itemize}
\paragraph{Episodio depressivo maniacale}  
\begin{itemize}
	\item \textbf{A.} Un periodo definito di umore anormalmente e persistentemente elevato, espanso o irritabile e di aumento anomalo dell'attività finalizzata o dell'energia, della durata di almeno una settimana e presente per la maggior parte del giorno quasi tutti i giorni
	\item \textbf{B.} Durante il periodo di alterazione dell'umore e di aumento di energia o attività, 3 o più dei seguenti sintomi (quattro se l'umore è solo irritabile) sono presenti a un livello significativo e rappresentano un cambiamento evidente rispetto al livello abituale:
	\begin{enumerate}
		\item Autostima ipertrofica o grandiosità
		\item Diminuito bisogno di sonno (non è insonnia)
		\item Maggiore loquacità del solito o spinta continua a parlare
		\item Fuga delle idee o esperienza soggettiva che i pensieri si succedano rapidamente
		\item Distraibilità, riferita o osservata
		\item Aumento dell'attività finalizzata (sociale, lavorativa, scolastica o sessuale) o agitazione psicomotoria.
		\item Eccessivo coinvolgimento in attività che hanno un alto potenziale di conseguenze dannose.
	\end{enumerate}

	\item \textbf{C.} L'alterazione dell'umore è sufficientemente grave da causare una marcata compromissione del funzionamento sociale o lavorativo o da richiedere l'ospedalizzazione per prevenire danni a sé o agli altri, oppure sono presenti manifestazioni psicotiche
	\item \textbf{D.} L'episodio non è attribuibile agli effetti fisiologici di una sostanza o di un'altra condizione medica
\end{itemize}
\paragraph{Episodio depressivo ipomaniacale}  
\begin{itemize}
	\item \textbf{A.} Un periodo definito di umore anormalmente e persistentemente elevato, espanso o irritabile e di aumento anomalo e persistente dell'attività finalizzata o dell'energia, della durata di almeno 4 giorni consecutivi, e presente per la maggior parte del giorno quasi tutti i giorni
	\item \textbf{B.} Durante il periodo di alterazione dell'umore e di aumento di energia o attività, 3 o più dei seguenti sintomi (quattro se l'umore è solo irritabile) sono presenti a un livello significativo e rappresentano un cambiamento evidente rispetto al livello abituale:
	\begin{enumerate}
		\item Autostima ipertrofica o grandiosità
		\item Diminuito bisogno di sonno (non è insonnia)
		\item Maggiore loquacità del solito o spinta continua a parlare
		\item Fuga delle idee o esperienza soggettiva che i pensieri si succedano rapidamente
		\item Distraibilità, riferita o osservata
		\item Aumento dell'attività finalizzata (sociale, lavorativa, scolastica o sessuale) o agitazione psicomotoria.
		\item Eccessivo coinvolgimento in attività che hanno un alto potenziale di conseguenze dannose.
	\end{enumerate}

	\item \textbf{C.} L'episodio è associato ad un evidente cambiamento nel funzionamento, che non è caratteristico dell'individuo quando è asintomatico
	\item \textbf{D.} L'alterazione dell'umore e il cambiamento nel funzionamento sono osservabili dagli altri.
	\item \textbf{E.} L'episodio non è sufficientemente grave da causare una marcata compromissione del funzionamento sociale o lavorativo, o da richiedere l'ospedalizzazione. Se sono presenti manifestazioni da richiede l'ospedalizzazione. Se sono presenti  manifestazioni psicotiche, l'episodio è, per definizione, maniacale.
	\item \textbf{F.}  L’episodio non è attribuibile agli effetti fisiologici di una sostanza
(per es., una sostanza di abuso, un farmaco o altro trattamento) o
di un’altra condizione medica.
\end{itemize}
\subsection{Caratteristiche dei disturbi dell'umore}
\paragraph{Polarità}  
\paragraph{Ciclicità} il decorso naturale del disturbo è a cicli che si auto limitano
\medskip\\
\includegraphics[width=\linewidth]{./images/image20}
\medskip\\
\section{I disturbi depressivi}
\paragraph{Il disturbo depressivo maggiore}  
\begin{itemize}
	\item Episodio depressivo maggiore senza alcun episodio maniacale o ipomaniacale
	\item Ricaduta e ricorrenza
	\item Esordio in qualsiasi momento della vita, l'incidenza aumenta in adolescenza
	\item Sintomi addizionali
\end{itemize}
E' caratterizzato da:
\begin{itemize}
	\item Uno o più episodi depressivi maggiori
	\item Esclusione di disturbo schizoaffettivo\ldots ecc.
	\item Esclusione di episodio maniacale o ipomaniacale
	\item Esclusione di reazione da lutto
\end{itemize}
\paragraph{Differenza tra ricaduta e ricorrenza}  
\begin{itemize}
	\item Ricaduta: ricomparsa dei sintomi quando l'episodio non si è ancora completamente risolto
	\item Ricorrenza: comparsa di un nuovo episodio
\end{itemize}
