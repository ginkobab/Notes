\lecture{14}{Thu 02 Apr 2020 10:33}{Disturbi Psicotici}{

Non esiste una definizione univoca del termine psicotico, ma l'elemento comune è il \textbf{rapporto distorto con la realtà} dei pazienti.
Uno psicotico potrebbe essere definito come \emph{colui il quale non sa distinguere ciò che è reale da ciò che non lo è}.
Non bastano i comportamenti bizzarri o incomprensibili per diagnosticare uno stato psicotico, non sono né necessari né sufficienti.
I disturbi psicotici possono essere presenti anche nei disturbi dell'umore.
\paragraph{Classificare i disturbi psicotici} 
\begin{itemize}
	\item Schizofrenia e altri disturbi psicotici, sono disturbi che includono sintomi psicotici come elemento primario e preminente del quadro clinico
	\item Disturbi dell'umore
\end{itemize}

\section{La schizofrenia}  
Disturbo mentale caratterizzato da sintomi psicotici multipli, con un decorso per lo più cronico, e un deterioramento molto grave del funzionamento globale.
}
Nel 1896 Kraepelin la definisce dementia praecox, caratterizzandola come irreversibile, ma successivamente venne riportata ad un disturbo mentale nel 1911.
\paragraph{DSM 5}  
\begin{itemize}
	\item La presenza di due o più di questi segni e sintomi per una porzione di tempo durante un mese
	\begin{itemize}
		\item Deliri
		\item Allucinazioni
		\item Eloquio disorganizzato
		\item Comportamento grossolanamente disorganizzato e catatonico
		\item Sintomi negativi, cioè diminuzione dell'espressione delle emozioni o abulia
	\end{itemize}
	\item La persistenza di alcuni segni per almeno 6 mesi.
	\item Disfunzione sociale/lavorativa 
	\item Compromissione del funzionamento
	\begin{itemize}
		\item Il funzionamento è a livello inferiore a quello precedente ai sintomi
		\item Se il disturbo comincia nell'infanzia o nell'adolescenza, incapacità a raggiunger il funzionamento atteso, più che un deterioramento 
		\item Confrontare il soggetto con i fratelli non affetti può essere utile per questa valutazione
		\item Il ciclo educativo è frequentemente interrotto e il soggetto può essere incapace di terminare la scuola
		\item Molti soggetti sono incapaci di conservare un lavoro per periodi di tempo prolungati, e sono impiegati a un livello inferiore dei loro genitori 
		\item La maggioranza dei pazienti non si sposano, e i più hanno contatti sociali limitati
	\end{itemize}

\end{itemize}
\paragraph{Decorso della schizofrenia}  
\begin{itemize}
	\item Fase prodromica
	\item Fase attiva
	\item Fase residua
\end{itemize}
\paragraph{I sintomi prodromici}  L'esordio è preceduto da un periodo di cambiamento subdolo rispetta al precedente funzionamento del paziente. Questa fase precede l'insorgenza dei sintomi, e ha una durata variabile
\paragraph{Fase residua}  Sul piano sintomatologico ricorda la fase prodromica, rimane una marcata compromissione della funzionalità socio lavorativa, sintomi negativa, ritiro sociale, scarsa attenzione all'igiene e all'aspetto esteriore 
\medskip\\
\includegraphics[width=\linewidth]{./images/image18}
\medskip\\
\paragraph{Il disturbo schizotipico di personalità} Si ritrova nei soggetti che presentano sintomi prodromici della schizofrenia ma in un quadro stabile, che non evolve.
\paragraph{Il disturbo schizoide di personalità}   
