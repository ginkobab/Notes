\lecture{17}{Tue 16 Apr 2020 10:49}{Ansia e Disturbi d'Ansia}

\section{Disturbi d'ansia}

\paragraph{L'ansia} è definibile come risposta normale e innata alla minaccia o all'assenza di persone o oggetti che assicurano e significano sicurezza.
\begin{itemize}
	\item Si accompagna all'attivazione di risposte che coinvolgono la psiche e il soma.
	\item Diffusa e orientata al futuro
	\item Sensazione generica di apprensione rispetto a situazioni nuove
\end{itemize}

\paragraph{La paura} è un'emozione preminente nei disturbi d'ansia. E' accompagnata da evitamento, come le fobie.
Nella paura si identifica un oggetto come stimolo chiaramente definito, cosa non sempre vera nell'ansia. L'altra differenza è la proiezione nel futuro.
La paura e l'ansia condividono però componenti comportamentali, fisiologiche e cognitive.


\paragraph{Ansia normale e patologica}  L'ansia è funzionale in molti casi, aiutando a preparare al futuro. 

La distinzione è tra ansia di stato e di tratto.

\paragraph{DSM} 
\begin{itemize}
	\item Fattori causali biologici $\rightarrow$ Fattori genetici
	\item Fattori causali psicologici $\rightarrow$ Nevroticismo, condizionamento
	\item Terapie $\rightarrow$ Esposizione, farmaci ansiolitici e antidepressivi


\end{itemize}
