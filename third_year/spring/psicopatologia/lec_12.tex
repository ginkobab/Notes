\lecture{12}{Mon 30 Mar 2020 10:37}{Prevalenza dei disturbi dell'umore}

\section{Prevalenza dei disturbi bipolari}

Prevalenza lifetime del DDM è circa 17\%, prevalenza a 12 mesi 7\%, due volte più frequente nelle donne. La prevalenza lifetime del disturbo bipolare è circa intorno all'1\%.

I disturbi bipolari hanno frequenza pari in uomini e donne, l'esordio è di solito in adolescenza o prima età adulta. Età media di esordio 18-22 anni. I giorni di umore depresso sono circa tre volte più numerosi di quelli con umore maniacale/ipomaniacale.

\paragraph{Specificatori dei disturbi dell'umore}
\begin{itemize}
	\item Con ansia, sensazione agitato, irrequieto, difficoltà di concentrazione, paura che possa accadere qualcosa di terribile
	\item Con caratteristiche miste
	\item Con caratteristiche melanconiche
	\item Con caratteristiche atipiche
	\item Con caratteristiche psicotiche 
	\item Con catatonia
	\item Con cicli rapidi
	\item Con esordio nel peripartum
	\item Con andamento stagionale
\end{itemize}

