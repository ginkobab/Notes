\lecture{9}{Mon 23 Mar 2020 10:05}{Disturbi dell'umore}{

Dobbiamo distinguere tra:
\begin{itemize}
	\item Umore: stato affettivo di base di un individuo, prevalente e prolungato o disposizionale.
	\item Emozione: esperienza spontanea e transitoria con specifici correlati somatici. Ha insorgenza acuta, provoca dei cambiamenti nell'esperienza soggettiva, nel comportamento e nella fisiologia.
	\item Sentimenti: concetto vago, simile a quello di emozione ma senza correlati somatici.
	\item Affetti: termine estensivo che comprende umore, sentimenti, atteggiamenti, preferenze.
\end{itemize}

\paragraph{Le emozioni} 
\medskip
\includegraphics[width=\linewidth]{./images/image15}
\medskip\\
Possono essere patologiche quando alcuni dei parametri sopra sono distorti.
\paragraph{Disturbi dell'intensità}
\begin{itemize}
	\item Appiattimento emotivo, ridotta espressione delle emozioni (Apatia, bulimia)
	\item Accentuata espressione delle emozioni (Mania, depressione, panico, fobie, delirium e disturbi di personalità)
\end{itemize}
\paragraph{Disturbi della variabilità} detta coartazione emotiva quando c'è perdita di variabilità, e labilità nel caso opposto.

\paragraph{Disturbi sul riconoscimento delle emozioni}  Se manca riconoscimento verso gli altri, ovvero manca empatia, si trova in narcisismo, antisocialità, psicopatia. Se manca consapevolezza rispetto alle proprie emozioni si parla di \emph{alessitimia}. 

\paragraph{L'umore}  Ci si accorge di essere davanti ad una patologia dell'umore nel caso in cui questo sia inadeguato o provochi sofferenza. Si manifesta nella depressione o nella mania.
\paragraph{Depressione}  il termine depressione viene usato per riferirsi a tre costrutti diversi: il sintomo, la sindrome e il disturbo depressivo dell'umore. Parliamo di depressione diagnosticata solo nell'ultimo caso. 
\paragraph{Depressione come sintomo}  intendiamo umore negativo, la persona riporta di essere triste, demotivato, giù di morale, oppure potrebbe essere evidenziato da chi gli sta attorno. Ha anche dei correlati somatici, può essere chiaramente localizzabile nel corpo. Molto spesso negli anziani, descrizione "come se". L'esperienza emotiva nella depressione è evidenziata dall'intensità, durata e immodificabilità. Questo la differenzia dalla tristezza, che è causata da un evento.
\paragraph{Depressione come sindrome} \medskip
\includegraphics[width=\linewidth]{./images/image16}
\medskip\\
\begin{itemize}
	\item Sintomi della sfera emotivo-affettiva:
	\begin{itemize}
		\item Deflessione timica: non influenzabile da interventi esterni
		\item Anedonia
		\item Indifferenza e inadeguatezza 
		\item Sentimenti di perdita e mancanza di sentimenti (depersonalizzazione) 
	\end{itemize}
	\item Sintomi cognitivi e percettivi
	\begin{itemize}
		\item Diminuita capacità di concentrarsi, di seguire un discorso articolato, di prendere decisioni e di memorizzare
		\item Rallentamento del flusso di pensiero 
		\item Impoverimento dei contenuti mentali che ripropongono gli stessi temi dolorosi
		\item Sensazione di perdita della dimensione futura.
	\end{itemize}
	\item Contenuti di pensiero
	\begin{itemize}
		\item tristezza 
		\item incapacità 
		\item preoccupazioni
		\item colp\medskip\\
			\includegraphics[width=\linewidth]{./images/image17}
			\medskip\\a
		\item autoaccusa
		\item morte
		\item inguaribilità 
		\end{itemize}
	\item Sintomi neurovegetativi
	\begin{itemize}
		\item Perdita di appetito, riduzione di peso
		\item Disturbi del sonno
		\item Perdita del piacere sessuale
		\item Affaticabilità, senso di peso psicofisico e torpore
	\end{itemize}
	\item Sintomi psicomotori
	\begin{itemize}
		\item Marcato rallentamento motorio
		\item Andatura lenta
		\item Difficoltà nei movimenti che vengono effettuati con sforzo evidente
		\item Resta a lungo seduto immobile o a letto tutto il giorno
		\item Trascura l'alimentazione, l'abbigliamento e l'igene
		\item Ridotta mimica e gestualità
	\end{itemize}
\end{itemize}
\paragraph{La comparsa:} in 3 fasi: esordio, periodo di stato e sintomi residui, per poi ritornare all'umore normale.
\paragraph{La depressione come disturbo dell'umore}  Viene diagnosticata nel seguente modo:

