\lecture{1}{Sun 08 Mar 2020 19:15}{Introduzione}{
Quali sono gli elementi che distinguono nell'\textbf{immaginario comune} la normalità?

\begin{enumerate}
	\item Compromissione di competenze
	\item Ricerca di aiuto
	\item Sofferenza
	\item Irrazionalità ?
	\item Trasgressione ?
	\item Devianza statistica
\end{enumerate}

In realtà pochi di questi criteri funziona, come il \textbf{dolore}, ma non sempre: a volte i pazienti si sentono estremamente bene (mania, ipomania, psicopatia), e a volte persone non malate soffrono.

Anche la compromissione o disabilità funziona come principio diagnostico, ed è il più utilizzato nei manuali diagnostici, ma ha dei limiti: condizioni rare possono essere vantaggiose, alcune condizioni non sono misurabili e alcune patologie possono essere normali nella società contemporanea.

\paragraph{Esempio Clinico}  

Cinzia: ragazza normale di 21 anni, ad una certa smette di uscire e di andare bene a scuola, e sta chiusa in casa tutto il giorno.

Il processo diagnostico consiste nella verifica di \textbf{ipotesi}, fermarsi troppo presto nel processo è molto rischioso
\medskip\\
\includegraphics[width=\linewidth]{./images/image1}
\medskip\\

I criteri: la presenza di più fattori è necessaria, assieme al loro covariare nel tempo.
Anche un solo episodio depressivo è sufficiente per dire che una persone ha un disturbo mentale.
Gli stessi comportamenti hanno diversi significati a diverse età.
Il contesto a volte spiega il comportamento meglio di una malattia.
Anche i fattori culturali devono essere considerati, ogni cultura ha modalità di reazione al lutto considerate normali.


