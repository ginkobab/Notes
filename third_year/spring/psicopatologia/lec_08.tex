\lecture{8}{Thu 19 Mar 2020 09:08}{Deliri}{

\paragraph{Il criterio dell'impossibilità} nel delirio non viene citata nel DSM V.
Il problema non sta nella plausibilità dell'idea, ma del modo in cui ci arriva, nelle prove che porta e nei comportamenti che produce.

\paragraph{La struttura auto-centrica}  è quasi sempre presente, tutto ha significato per il paziente. Per esempio il paziente:
\begin{itemize}
	\item è oggetto di persecuzioni
	\item è destinatario di messaggi allusivi
	\item è colui a cui le voci si rivolgono
	\item è responsabile di tutti i mali
	\item è colui a cui viene sottratto il pensiero 
	\item impersonifica la divinità 
	\item acquisisce capacità o poteri
\end{itemize}
\paragraph{Forma e contenuto}  del delirio: La forma ci permette di eliminare alcune possibilità di diagnosi e di evidenziarne altre, ma anche il contenuto può informare sul tipo di patologia. Ad esempio se i temi di colpa, rovina e morti sono presenti nel delirio, andremo a cercare un disturbo depressivo, mentre in caso di un quadro di eccitamento maniacale ci aspetteremmo deliri con contenuti di grandiosità.

Questo ci è utile per trovare congruenze o incongruenze tra l'umore dominante e i contenuti deliranti.

\paragraph{Idee simil-deliranti}
Presenti in disturbi dell'umore
\medskip\\
\includegraphics[width=\linewidth]{./images/image9}
\medskip\\
\paragraph{Deliri veri o primari} presenti in patologie 
\medskip\\
\includegraphics[width=\linewidth]{./images/image10}
\medskip\\
\section{Classificazione dei deliri per contenuto}
\begin{itemize}
	\item Persecuzione
	\item Riferimento
	\item Colpa
	\item Negazione
	\item Grandezza
	\item Erotico
	\item Ipocondriaco
	\item Mistico
\end{itemize}

\paragraph{Il delirio di persecuzione}  è estremamente frequente, si trova in schizofrenia, nella mania in disturbi deliranti. Consiste nel fatto che il paziente:
\begin{itemize}
	\item Si sente oggetto di attenzione \emph{ostile} da parte di altre persone
	\item Si sente osservato, spiato, seguito o controllato
	\item Teme di essere drogato o avvelenato (delirio di veneficio)
	\item È convinto di complotti ai suoi danni
	\item Vede fatti, gesti o oggetti come dotati di significato ostile
\end{itemize}
\paragraph{Il delirio di riferimento}  è la convinzione che situazioni, oggetti, persone, fatti assumano un particolare significato allusivo riferito alla propria persona. Quasi sempre i contenuti sono a connotazione ostile, ma \emph{manca un contenuto persecutorio chiaramente espresso}. È frequente nella schizofrenia.

\paragraph{Deliri di colpa}  o di indegnità o rovina. Il paziente:
\begin{itemize}
	\item Si sente responsabile di danni e sciagure
	\item Si sente rovinato, distrutto indegno, di essere umano
	\item Pensa di aver condotto se stesso e le persone vicine a lui alla rovina economica e sociale
	\item È convinto che continuare a vivere significhi perpetuare in eterno questa condizione
\end{itemize}
C'è rischio di suicidio o di omicidio-suicidio.

\paragraph{Delirio di negazione}  o nichilistico. Piuttosto raro, si può trovare in disturbi depressivi o nella schizofrenia. Consiste nella convinzione della non esistenza di sé, o di una parte di sé.

\paragraph{Delirio di grandezza} contenuto speculare a quello di colpa, indegnità o rovina, accompagna spesso la mania. Il paziente si può sentire ricco potente, eccetera.

\paragraph{Delirio erotico}  piuttosto raro, maggiore prevalenza nel sesso femminile, la paziente può essere convinta di possedere un'attrattività erotica fuori dal comune, di essere oggetto di corteggiamenti e di essere in grado di effettuare ogni tipo di conquista sessuale.

\paragraph{Delirio di gelosia}  Prevalente nel sesso maschile dopo i 40 anni, spesso associato all'alcolismo. Convinzione dell'infedeltà sopratutto a livello sessuale del partner. Non centrato su un'unica persona, ma su molteplici relazioni. Può essere difficile classificarlo come delirio.

\paragraph{Delirio mistico}  Il punto critico è legato alla caratteristica auto-centrica. Il paziente può riferire di comunicare direttamente con Dio, esserne messaggero ecc.

\paragraph{Delirio ipocondriaco} Convinzione di soffrire di una malattia fisica in assenza di qualsiasi rilievo obiettivo. È piuttosto comune, può fare comparsa anche nei disturbi dell'umore depressivi, negli anziani, mentre nei giovani è spesso in rapporto a fattori organici come traumi cranici o abuso di sostanze.

\paragraph{Deliri bizzarri}  sono chiaramente non plausibili e non comprensibili.

\paragraph{Deliri di influenzamento}  tema del controllo, il paziente si sente trasformata in un automa in balia di forze esterne e minacciose, abbinate a disturbi della forma.
\begin{itemize}
	\item Inserimento del pensiero, il paziente sa che alcuni contenuti della sua coscienza non gli appartengono
\item Furto del pensiero, il paziente percepisce il suo pensiero come asportato o rubato, associato a \emph{blocco del pensiero}
	\item Trasmissione del pensiero, il paziente sente che la sua attività di pensiero, le sue emozioni ricordi e desideri non sono più privati ma diffusi e conosciuti pubblicamente.
\end{itemize}

\paragraph{Sintomi di primo rango} Indicano la presenza di schizofrenia. 
\begin{itemize}
	\item Si devono presentare frequentemente nella schizofrenia
	\item Non devono verificarsi in condizioni diverse dalla schizofrenia
	\item Non dev'essere ambiguo identificare il sintomo
\end{itemize}
E sono:
\begin{itemize}
	\item Esperienze dispercettive, voci dialoganti, eco del pensiero.
	\item Passività del pensiero, inserzione, sottrazione e diffusione del pensiero.
	\item Percezioni deliranti, percezione normale interpretata dal paziente in modo delirante e considerata molto significativa.
	\item Esperienze di passività più generali, nell'ambito dell'affettività, del corpo, della volontà.
\end{itemize}

\paragraph{Disturbi del pensiero non deliranti}  
\begin{itemize}
	\item Le ossessioni: non è un'idea (o impulso o immagine) strana o assurda, ma è patologica per quanto la persona la considera rilevante, per la sua persistenza e ricorrenza. Implicano ansia e senso di colpa nel soggetto. I pensieri sono particolarmente ripugnanti per l'individuo, per esempio persone religiose possono essere tormentate da pensieri blasfemi. Il paziente appaiono contro la volontà del paziente, che attua una resistenza. Per questo le allucinazioni non possono essere ossessivi
	\item Le idee prevalenti: non sono false e incorreggibili, non sono percepite prive di senso, quindi non c'è resistenza dall'individuo, al contrario. Sono simili a convinzioni politiche, etiche o religiose radicali. Possono essere associate a personalità abnormi, e ad un'intensa componente affettiva. Contenuti frequenti:
		\begin{itemize}
			\item Idee persecutorie (tipo querulomane o religioso)
			\item Gelosia
			\item Ipocondria
			\item Dismorfofobia
			\item Parassitofobia
			\item Anoressia
			\item Innamoramento
		\end{itemize}
\end{itemize}
\includegraphics[width=\linewidth]{./images/image14}
