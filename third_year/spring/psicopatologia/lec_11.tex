\lecture{11}{Thu 26 Mar 2020 10:06}{Disturbi depressivi}{


\paragraph{Depressione maggiore:} Indagine clinica. \\
\begin{itemize}
	\item Fare domande aperte sull'umore attuale;
	\item Cercare informazioni su episodi pregressi;
	\item Indagare comorbodità;
\end{itemize}

\paragraph{Disturbo depressivo persistente:}  può essere accoppiato con la distimia, umore depresso persistente che dura almeno per due anni
\medskip\\
\includegraphics[width=\linewidth]{./images/image11}
\medskip\\
\medskip\\
\includegraphics[width=\linewidth]{./images/image12}
\medskip\\
Può rimanere stabile anche tutta la vita, sfociando in un disturbo di personalità.
Persone con esordio precoce hanno maggiore probabilità di avere comorbodità.

\paragraph{Caso clinico: "Perdita di interesse nella vita"}  
\begin{itemize}
	\item Primo episodio in persona con funzionamento generale buono
	\item Sintomi compaiono gradualmente
	\item Attualmente grave compromissione del funzionamento
	\item Anamnesi negativa per problemi di salute mentale
	\item Familiarità positiva per disturbi depressivi
	\item Il quadro clinico sembrerebbe essere caratterizzato primariamente da un problema di umore
\end{itemize}
\paragraph{Disturbo da disregolazione dell'umore dirompente}
\begin{itemize}
	\item Umore persistentemente irritabile o arrabbiata
	\item Gravi e ricorrenti scoppi di collera
	\item Incoerenza con lo stadio di sviluppo degli episodi
	\item Scoppi 3 o più volte a settimana
	\item Bisogna indagare gli effetti di contesto, devono essere presenti in almeno due (casa, scuola)
\end{itemize}
\paragraph{Disturbo disforico premestruale}  
\begin{itemize}
	\item Umore depresso
	\item Labilità affettiva
	\item Interferenza con normale funzionamento
\end{itemize}

\paragraph{I Disturbi Bipolari}  
\begin{itemize}
	\item Disturbo bipolare I: Uno o più episodi maniacali, solitamente accompagnati da episodi depressivi maggiori o ipomaniacali. Esclusione di schizofrenia.
	\item Disturbo bipolare II: Versione meno grave dell'I
	\item Disturbo ciclotimico: Analogo alla distimia, dura due anni, non si riesce mai a identificare nessuno dei due disturbi
\end{itemize}
\medskip
\includegraphics[width=\linewidth]{./images/image13}
