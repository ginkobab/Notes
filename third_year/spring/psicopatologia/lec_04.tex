\lecture{4}{Tue 10 Mar 2020 14:32}{La Psicopatologia Generale}{

Nel corso del tempo sono più che triplicate le diverse categorizzazioni di disturbi mentali nei DSM.

L'introduzione del DSM ha contribuito a migliorare l'\textbf{attendibilità} della diagnosi (due psichiatri diagnosticano la stessa malattia) ma la validità non è migliorata (es. Rosenhan)

Per \textbf{psicopatologia} si può intendere la psicopatologia descrittiva o interpretativa.

\paragraph{ La Psicopatologia descrittiva} si occupa dello studio e della classificazione degli elementi invarianti dei disturbi mentali, procedendo per le grandi aree dello sviluppo psicologico (percettivo, dello sviluppo, etc.), inglobando la psicopatologia \emph{oggettiva e soggettiva} 

\paragraph{La Psicopatologia oggettiva} ha come ambito d'indagine la sola realtà esterna, ed è eredità di Kraeplin, e nasce come contrapposizione all'approccio soggettivo.

\paragraph{La Psicopatologia soggettiva} ha come obiettivo la valorizzazione delle esperienze soggettive, rifiuta i metodi delle scienze naturali, fondato da Jasper. Per spiegare gli eventi interni bisogna entrare nella prospettiva empatica.

\paragraph{Il limite della comprensibilità} è raggiunto quando non sono più possibili altro che spiegazioni causali, bisogna quindi spostarsi dalla comprensione alla spiegazione.  

\paragraph{Primario/Secondario} Nel dominio della spiegazione: primario si riferisce alla causa immediata, mentre secondario si intende l'effetto. Nel dominio della comprensione primario significa inderivabile, mentre secondario ciò che emerge dal primario.

\paragraph{Variabili importanti} 
\begin{itemize}
	\item Obiettivi dell'intervento: possono essere specifici o generali
	\item Il setting 
	\item Il contratto terapeutico.
	\item Valutazione clinica
	\item Importanza attribuita alla relazione
	\item Tecniche e procedure
\end{itemize}


