\lecture{7}{Tue 17 Mar 2020 14:24}{Disturbi del pensiero}{

Possono essere di \textbf{forma} o \textbf{contenuto}  
\begin{itemize}
	\item Nei disturbi di forma ci sono problemi di ragionamento e a volte anche nel contenuto
	\item Nei disturbi di contenuto il problema sta solo nell'oggetto del pensiero, non nella modalità.
\end{itemize}
\paragraph{La diagnosi} avviene grazie alla comunicazione.  

\section{Disturbi della forma del pensiero} 
\paragraph{Caratteristiche}  
\begin{itemize}
	\item Alterazione della velocità
	\item Disorganizzazione
	\item Assenza di un'idea centrale
	\item Nessi causali distorti
	\item Incapacità di comprendere significati
\end{itemize}
\paragraph{I disturbi:}  
\begin{enumerate}
	\item Accelerazione: vengono prodotte troppe idee, viene compromessa l'efficacia della comunicazione ma i nessi causali sono conservati. Elevata distraibilità, pressione interna a parlare, la sovrapproduzione di idee viene percepita internamente come alta prestazione
	\item Fuga delle idee: forma estrema di accelerazione, la persona si distrae facilmente guidata da criteri come assonanza, somiglianza, rima, fattori casuali, \emph{tipica della mania} 
	\item Rallentamento: lentezza, bassa produttività, ridotta efficacia comunicativa. \emph{Tipica della depressione}, sensazione di fatica nel passare da un concetto all'altro, difficoltà nel prendere decisioni, mancanza di concentrazione, memoria e chiarezza. Il paziente può sviluppare un'\emph{idea sovrastimata} o \emph{delirante} che i pensieri gli sfuggano dalla mente.
	\item Ridondanza procedurale: abbondare di concetti, premesse infinite
	\item Perseverazione: non riesce a modificare il comportamento in base al feedback, pessimo punteggio nel Winsconsin Card Sorting Test.
	\item Tangenzialità: non si riesce ad andare nel cuore della comunicazione
	\item Illogicità: gravità estrema, non si arriva alle conclusioni logiche, i nessi sono insensati.
	\item Distraibilità
	\item Associazioni per assonanza
	\item Neologismo
\end{enumerate}
\paragraph{I disturbi estremi}  
\begin{itemize}
	\item Blocco del pensiero: arresto non intenzionale dell'eloquio e presumibilmente del pensiero, percezione soggettiva che il pensiero si blocchi, spiegazione del paziente come \emph{sottrazione del pensiero}. Associato alla \emph{schizofrenia}. 
	\item Concretismo: ridotta o assente capacità di operare astrazioni, può apparire in \emph{deficit intellettivi} o \emph{demenza} 
\end{itemize}
\section{Disturbi del contenuto del pensiero}
\paragraph{I disturbi}  
\begin{itemize}
	\item Delirio: sempre un fenomeno patologico, indice di follia, ci sono 3 criteri definitori:
		\begin{enumerate}
			\item Impossibilità del contenuto, criterio confondente, non solo i deliri bizzarri sono deliranti.
			\item Certezza impareggiabile della veridicità dell'idea, come per dogmi religiosi, filosofici o scientifici.
			\item Incorreggibilità, immodificabilità, non è necessariamente patologica, ma c'è differenza tra certezza patologica e non.
		\end{enumerate}

	\item Idea prevalente: non necessariamente patologica
	\item (Ossessione)
\end{itemize}

