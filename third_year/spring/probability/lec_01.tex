\lecture{1}{Sun 26 Apr 2020 09:38}{Concezioni della probabilità}

\section{Concezioni della probabilità}

\paragraph{Concezione classica}  rapporto tra il numero dei casi favorevoli e il numero dei casi possibili, a condizione che siano \textbf{equiprobabili}, il che limita le possibilità applicative.

\paragraph{Concezione frequentista} si basa sulla legge dei grandi numeri, rapporto tra numero di successi e numero di prove. All'aumentare del numero di prove il rapporto si stabilizza sulla probabilità reale. Il limite è che bisogna avere la possibilità di effettuare un numero elevatissimo di prove.

\paragraph{Teoria assiomatica della probabilità}  di Kolmogorov





