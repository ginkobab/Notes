\documentclass[12pt, a4paper]{article}
\usepackage{amsmath}

\date{15 Ottobre 2019}
\title{Lezione 5}
\author{Analisi I}

\begin{document}

\maketitle

\section{Insiemi Numerici}

$\mathbf{N} = \{1, 2, 3, \ldots \}$ Interi \\
$\mathbf{Z} = \{\ldots, -1, 0, 1, \ldots \}$ Interi negativi \\
$\mathbf{Q} = \{\ldots, -\frac{1}{2}, \ldots, -\frac{1}{3}, \ldots 0, \ldots, \frac{1}{3}, \ldots \frac{1}{2}, \ldots\}$ Razionali  

\paragraph{Osservazione} intuitivamente $\frac{1}{n} \to 0$ andando verso $\infty$

\section{Successioni}

\`E una funzione definita in $\mathbf{N}$. Essendo una funzione si potrebbe usare $f(x)$, ma $f$ di solito si usa in $\mathbf{R}$. \\ Usiamo quindi $a(1), a(2)$ che viene scritto secondo la notazione come $a_1, a_2$.

Dobbiamo però ricordarci che $a_1$ \`e una funzione e non confonderla con la sua immagine, e dobbiamo ricordare che \`e  ordinata (soprattutto per successioni infinte)

\paragraph{Esempi} 
\begin{flushleft}
$a_n = \frac{n}{n+1}$ a valori di $\mathbf{Q}$ \\
$b_n = -1^n$ a valori di $\mathbf{Z}$ o in $\{-1, 1\}$ \\
$b_1 = -1, b_2 = 1, b_3 = -1 \ldots$ 
\end{flushleft}

\section{Limiti}

Li introduziamo non per calcolare cose gi\`a visibili, o ovvie, ma per dimostrare nuovi concetti  

\paragraph{Definizione di Limite di una successione} 
\begin{quote}
Il limite di una successione $a_n$, scritto anche ${\{a_n\}}^\infty_0$ \`e una successione a valori in $\mathbf{R}$. \\ 
Diciamo che $a_n$ converge ad $A \in \mathbf{Q}$, e scriviamo $a_n \to A$ oppure $\lim_{n \to \infty} a_n = A$ \\
Se per ogni approssimazione $\varepsilon > 0$ esiste un indice $\mu$ (dipendente da $\varepsilon$) tale che se $n \geq n_0 \Rightarrow A - \varepsilon \leq a_n \leq A + \varepsilon$ ovvero $ | a_n - A | \leq \varepsilon$
\end{quote}

\paragraph{Esercizi} Dimostriamo che 

\begin{equation*}
    \frac{n+1}{n-1} \to 1
\end{equation*}

\paragraph{Soluzione} Fissato $\varepsilon > 0$ dobbiamo trovare $n_0$ tale che $n \geq n_0 \Rightarrow |1 - \frac{n+1}{n-1}| \leq \varepsilon$  

\subparagraph{Osservazione} Dobbiamo ragionare `a rovescio', per arrivare a $ n \geq n_0$ (con $n_0$ esplicito) \\ Dobbiamo cio\`e riempire gli spazi in questa catena
\begin{equation*}
    | 1 - \frac{n+1}{n-1} \leq \varepsilon \Leftarrow \cdots \Leftarrow n \geq?
\end{equation*}

\begin{eqnarray*}
    |1- \frac{n+1}{n-1}| \leq \varepsilon \Leftrightarrow \endline
    |\frac{n-1-(n+1)}{n-1}| \leq \varepsilon \Leftrightarrow 
    |\frac{-2}{n-1}| \leq \varepsilon \Leftrightarrow \\
    \frac{2}{n-1} \leq \varepsilon \Leftrightarrow 
    n - 1 \geq \frac{2}{\varepsilon} \Leftrightarrow 
    n \geq 1 + \frac{2}{\varepsilon}
\end{eqnarray*}

Nel conto precedente abbiamo risolto una disequazione nella variabile $n_0$, in particolare il miglior $n_0$.  

Nella pratica non \`e sempre agevole o conveniente trovare il miglior $n_0$, n\'e la definizione lo richiede.

\paragraph{Esempio} Verifica che 

\begin{eqnarray*}
|\frac{3n}{2n+\sqrt{n} + 7}| \leq \varepsilon \\
 | \frac{6n + 3\sqrt{n} + 21 - 6n}{4n + 2\sqrt{n} +14} \leq \varepsilon
\end{eqnarray*}






















\end{document}
