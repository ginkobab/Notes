% Options for packages loaded elsewhere
\PassOptionsToPackage{unicode}{hyperref}
\PassOptionsToPackage{hyphens}{url}
%
\documentclass[
]{article}
\date{3 Ottobre 2019}
\title{Lezione 2}
\author{Dinamica}
\usepackage{lmodern}
\usepackage{amssymb,amsmath}
\usepackage{ifxetex,ifluatex}
\ifnum 0\ifxetex 1\fi\ifluatex 1\fi=0 % if pdftex
  \usepackage[T1]{fontenc}
  \usepackage[utf8]{inputenc}
  \usepackage{textcomp} % provide euro and other symbols
\else % if luatex or xetex
  \usepackage{unicode-math}
  \defaultfontfeatures{Scale=MatchLowercase}
  \defaultfontfeatures[\rmfamily]{Ligatures=TeX,Scale=1}
\fi
% Use upquote if available, for straight quotes in verbatim environments
\IfFileExists{upquote.sty}{\usepackage{upquote}}{}
\IfFileExists{microtype.sty}{% use microtype if available
  \usepackage[]{microtype}
  \UseMicrotypeSet[protrusion]{basicmath} % disable protrusion for tt fonts
}{}
\makeatletter
\@ifundefined{KOMAClassName}{% if non-KOMA class
  \IfFileExists{parskip.sty}{%
    \usepackage{parskip}
  }{% else
    \setlength{\parindent}{0pt}
    \setlength{\parskip}{6pt plus 2pt minus 1pt}}
}{% if KOMA class
  \KOMAoptions{parskip=half}}
\makeatother
\usepackage{xcolor}
\IfFileExists{xurl.sty}{\usepackage{xurl}}{} % add URL line breaks if available
\IfFileExists{bookmark.sty}{\usepackage{bookmark}}{\usepackage{hyperref}}
\hypersetup{
  hidelinks,
  pdfcreator={LaTeX via pandoc}}
\urlstyle{same} % disable monospaced font for URLs
\setlength{\emergencystretch}{3em} % prevent overfull lines
\providecommand{\tightlist}{%
  \setlength{\itemsep}{0pt}\setlength{\parskip}{0pt}}
\setcounter{secnumdepth}{-\maxdimen} % remove section numbering


\begin{document}

\maketitle

\section{Freud - Parte Seconda}

Ricapitolando

Trauma

Rimozione contenuto: difesa

Conversione affetto

Perché non ricordano?

Breuer: non ricordano perché erano in uno stato alterato, la cura
consiste nel provocare di nuovo lo stato alterato (ipnosi), per
rimuovere la dissociazione tra evento e affetto, scaricando l'energia
attraverso un'up reazione

Freud: Dato che non è vero che l'ipnosi non è un fatto prettamente
patologico, come diceva \ldots\ldots. Quindi dice che ciò che rende
rimosso il contenuto non è lo stato ipnotico, ma l'incompatibilità con
la massa delle idee dominanti, quindi la cura non è nella suggestione di
per sé, ma nel rendere cosciente ciò che è rimosso (riorganizzando la
struttura delle idee dominanti?).

Il trauma ricorrente per Freud sono le `seduzioni infantili', che vanno
da abusi a abbracci (non erano nella cultura).

Le innovazioni di Freud:

\begin{itemize}
\tightlist
\item
  Ricostruire assieme al paziente la sua storia
\item
  Individua come eziologia della malattia un evento
\item
  Individua delle difese, che fanno parte del funzionamento psichico
\item
  Quindi non c'è differenza qualitativa tra i normali e i patologici
\end{itemize}

Freud scrive che molti dei traumi sessuali denunciati dalle sue pazienti
non si erano mai verificati, e che si trattava invece di desideri\\
Lo capisce per la diffusione del trauma, e per un suo fratello
isterico.\\
Secondo Freud inoltre l'inconscio non sa distinguere tra realtà e
fantasia.

La teoria passa quindi a essere basata sui desideri inconsci.

Alla fine degli anni `90 dell' `900 in America avvenne una specie di
isteria di massa, dove una marea di donne in terapia, probabilmente a
causa di regressioni all'infanzia provocate dal terapista, denunciò
abusi, addirittura cambiando leggi per non fare cadere in prescrizione i
reati.\\
I racconti arrivavano a parlare di rapimenti e sacrifici a sette
sataniche e data l'assenza di prove, venne chiesto ad un'associazione di
'esperti di memoria' se fosse possibile o fosse un'isteria di massa,
concludendo che la maggior parte dei casi fosse isterico.

Lo scopo dell'analisi diventa così non capire cosa sia accaduto oppure
no, ma capire la realtà psichica del paziente, esplorando le fantasie, e
interpretando i ricordi alla stregua dei sogni.

Freud rinuncia così all'ipnosi, definendola capricciosa e instabile, e
produce anche il rischio di influenzare troppo il paziente.\\
Anche se è impossibile essere neutri nell'osservazione, togliere
l'ipnosi è un passo verso ciò, evitando di contaminare i dati, infatti
se i risultati terapeutici dipendono dai metodi del medico, sono
chiaramente biasati.

Freud introduce così la libera associazione. Freud pensava che tutti
fossero in grado, anche se in realtà alcuni devono imparare a farlo.\\
Anche il terapeuta entra in uno stato particolare, l'attenzione libera
fluttuante, una sorta di allerta passiva.\\
Analizza le libere associazioni, ma anche le resistenze, ovvero tutte le
modalità che si oppongono a lasciar esprimere l'inconscio, come non
voler andare sul lettino (che incoraggia la regressione, rilassa
permettendo una sorta di deprivazione sensoriale, e permette a Freud di
non influenzare il paziente attraverso il suo volto, perché il terapista
deve essere neutro, oggettivo, non fare trapelare nulla.)

Freud si dedica ora all'autoanalisi (interpretazione dei sogni).\\
Il sogno non è altro che una forma particolare del nostro pensare
inconscio, secondo Freud, ma anche oggi vengono ritenuti un accesso al
nostro inconscio.\\
Il sogno è come una vita parallela, e durante l'analisi aumenta molto
l'attività onirica.

Il sogno è la via regia per arrivare all'inconscio, ovvero per
soddisfare i desideri in maniera camuffata.\\
I sogni infatti sono eliminazioni allucinatorie o simulate di stimoli
che possono disturbare il sogno (resistenza alla sveglia).\\
Nel sogno si demoliscono le forze dinamiche che impediscono il sorgere
di desideri inconsci, se il desiderio però venisse rappresentato
direttamente il sogno verrebbe interrotto.

Freud distingue quindi tra pensieri onirici latenti, sottoposti a
trasformazioni producendo quelli manifesti.\\
Attraverso queste trasformazioni i contenuti latenti diventano
irriconoscibili.\\
Il sogno è composto da

\begin{itemize}
\tightlist
\item
  Stimoli somatici
\item
  Residui diurni
\item
  Pensieri del sogno
\item
  Ricordi
\end{itemize}

Meccanismi:

\begin{itemize}
\tightlist
\item
  Condensazione: un'unica immagine può rappresentare diverse idee
  associative
\item
  Spostamento: traslo l'immagine da stimolo rilevante a irrilevante (non
  crede però nei simboli universali Junghiani, i simboli sono solo
  personali)
\item
  Raffigurabilità plastica: il sogno è fatto di immagini, è come passare
  dall'astratto al concreto, e dato che non è possibile rappresentare
  negazioni o relazioni, un elemento può significare il contrario di sé,
  sopratutto se è manifesto
\item
  Elaborazione secondaria: avviene appena prima di svegliarsi, permette
  di creare una storia, delle connessioni tra le immagini
\end{itemize}

C'è un'analogia nel rapporto tra contenuto latente e manifesto, desideri
inconsci e sintomi, entrambe sono difese che mantengono i contenuti
nell'inconscio, censurandolo.

Il sogno è un esempio di come avviene un conflitto tra forze
contrastanti, il conflitto è alla base del funzionamento del sintomo.\\
Il fatto che il sogno si esprima attraverso immagini visive, unito al
particolare stato del sognatore, fa sì che il soggetto si ritiri quasi
completamente dal mondo circostante, il che porta a una triplice
regressione:

\begin{itemize}
\tightlist
\item
  Dal conscio all'inconscio
\item
  Dal presente al passato
\item
  Da una forma più evoluta del pensiero a una meno evoluta delle
  immagini visive
\end{itemize}

Excurcus su cosa viene definito ancora valido:

\begin{itemize}
\tightlist
\item
  La libera associatività (spostamento)
\item
  Ipermnesia
\item
  Filosofi della scienza attuali propongono idee simili

  \begin{itemize}
  \tightlist
  \item
    Tunnel dell'io, sogno con funzione auto-organizzante
  \item
    Being No One, libro figo sulle autorappresentazioni
  \item
    Funzione adattiva, gli antichi sognavano minacce così si preparavano
    alle minacce reali, per noi sono stimolazioni sociali che è la
    minaccia ora
  \end{itemize}
\end{itemize}

Freud propone la metapsicologia, sostenendo che ogni manifestazione
psichica può essere descritta da 3 diversi punti di vista:

\begin{itemize}
\tightlist
\item
  In termini Economici: Energia psichica (parte più arcaica)
\item
  In termini dinamici (mente come conflitto tra due poli)
\item
  Prospettiva topica (dove avvengono i conflitti)
\end{itemize}

Tutto questo lo sviluppa in un'opera dove si trova positivismo, idee
newtoniane, ossessione per la sessualità e morte

\textbf{Il punto di vista Economico}

Il nostro sistema psichico ha un'energia intrinseca che ha una meta,
idea basata su principi Fechneriani, trattenere energia causa dolore e
rilasciare piacere, quest'energia ha un oggetto attraverso il quale può
essere scaricata, e secondo Freud questo è variabile.\\
Per esempio il feticista usa una scarpa, i Don Giovanni le vagine.
Quest'energia può investire e disinvestire gli oggetti.\\
Investendo un oggetto gli do un significato, investendo libido. Ritirare
l'investimento equivale ritirare libido, come nel caso del lutto, che
disinveste l'oggetto esterno per investire la rappresentazione interna o
oggetto internalizzato.\\
L'energia ha diverse caratteristiche:

\begin{itemize}
\tightlist
\item
  meta
\item
  grandezza
\item
  oggetto
\end{itemize}

L'energia assume forme diverse:

\begin{itemize}
\tightlist
\item
  Libido e Pulsione di autoconversazione
\item
  Libido e Aggressività
\item
  Istinto di vita e Istinto di morte
\end{itemize}

Il principio di piacere contrapposto al principio di realtà

\begin{itemize}
\tightlist
\item
  Il primo vuole soddisfare prima possibile i desideri, i bambini hanno
  solo questo (anche gli adolescenti)
\item
  Il secondo ci permette di gratificare gli impulsi in maniera
  socialmente accettabile
\end{itemize}

Ma se la mente funziona secondo il principio di piacere, perché le
persone ricercano situazioni dolorose? Come le donne che si avvicinano a
uomini pericolose dopo aver avuto un'infanzia con un padre violento.\\
Una risposta è: ciò che è piacevole per l'Es non è sempre piacevole per
l'Io\\
Un'altra è: coazione a ripetere, che può essere utilizzata dal terapista
attraverso il transfert, che ci permette di trasferire le relazioni
primarie sul terapista. Oggi diremmo che la ripetiamo anche per
padroneggiare l'evento

Un esempio di transfert è il parto isterico di Anna O. e secondo Freud
può essere utilizzato per curare il paziente.\\
Per alcuni stili terapeutici odierni il lavoro principale è quello del
transfert

Per la psicanalisi cognizione e emozione sono uniti, non si può avere
uno senza l'altro a meno che non stia agendo un meccanismo di difesa.

\textbf{Il punto di vista Dinamico}

\end{document}
