% Options for packages loaded elsewhere
\PassOptionsToPackage{unicode}{hyperref}
\PassOptionsToPackage{hyphens}{url}
%
\documentclass[
]{article}
\date{15 Ottobre 2019}
\title{Lezione 6}
\author{Dinamica}
\usepackage{lmodern}
\usepackage{amssymb,amsmath}
\usepackage{ifxetex,ifluatex}
\ifnum 0\ifxetex 1\fi\ifluatex 1\fi=0 % if pdftex
  \usepackage[T1]{fontenc}
  \usepackage[utf8]{inputenc}
  \usepackage{textcomp} % provide euro and other symbols
\else % if luatex or xetex
  \usepackage{unicode-math}
  \defaultfontfeatures{Scale=MatchLowercase}
  \defaultfontfeatures[\rmfamily]{Ligatures=TeX,Scale=1}
\fi
% Use upquote if available, for straight quotes in verbatim environments
\IfFileExists{upquote.sty}{\usepackage{upquote}}{}
\IfFileExists{microtype.sty}{% use microtype if available
  \usepackage[]{microtype}
  \UseMicrotypeSet[protrusion]{basicmath} % disable protrusion for tt fonts
}{}
\makeatletter
\@ifundefined{KOMAClassName}{% if non-KOMA class
  \IfFileExists{parskip.sty}{%
    \usepackage{parskip}
  }{% else
    \setlength{\parindent}{0pt}
    \setlength{\parskip}{6pt plus 2pt minus 1pt}}
}{% if KOMA class
  \KOMAoptions{parskip=half}}
\makeatother
\usepackage{xcolor}
\IfFileExists{xurl.sty}{\usepackage{xurl}}{} % add URL line breaks if available
\IfFileExists{bookmark.sty}{\usepackage{bookmark}}{\usepackage{hyperref}}
\hypersetup{
  hidelinks,
  pdfcreator={LaTeX via pandoc}}
\urlstyle{same} % disable monospaced font for URLs
\setlength{\emergencystretch}{3em} % prevent overfull lines
\providecommand{\tightlist}{%
  \setlength{\itemsep}{0pt}\setlength{\parskip}{0pt}}
\setcounter{secnumdepth}{-\maxdimen} % remove section numbering


\begin{document}

\maketitle

\textbf{Fantasia Inconscia - Klein} Per Freud la fantasia era
l'allucinare. Klein sviluppa, dicendo che si nasce con una riserva di
fantasie innate che ci accompagnano. Per esempio, qualsiasi tendenza che
il bambino ha, implica che ci sia un oggetto (come secondo la teoria
dell'intenzionalità di Brentano), quindi il bambino nasce non solo con
la tendenza ad alimentarsi, ma anche con quella di avere un oggetto
\textbf{fisso} (a differenza di Freud secondo il quale l'oggetto varia),
ovvero il \emph{seno}.

La fantasia diventerà l'espressione degli istinti per Klein. Si crea
quindi una relazione che sussiste dall'inizio tra oggetti e fantasie
(non c'è quindi la fase del narcisismo)\\
Le fantasie cercano oggetti esterni e poi li ``colorano'', li
trasformano.\\
Nel contempo la realtà trasforma la fantasia, anche se è sempre più
importante la fantasia secondo Klein.\\
Non c'è bisogno di realtà negativa per avere una fantasia negativa.

In Freud è la tensione biologica ad attivare la psiche.\\
In Klein la pulsione è già psicologica.

Il primo oggetto è quindi il seno materno, scisso prestissimo in buono e
cattivo: ciò genera l'ambivalenza odio e amore.\\
Il rapporto col seno ne implica l'introiezione e la proiezione.\\
Questi processi collaborano nella formazione dell'Io e del Super Io.

La mente per Klein è uno spazio che contiene oggetti che consentono di
relazionarsi con il mondo esterno. Sono entità concrete, che dipendono
dall'oggetto esterno (dal modo in quale è stato introiettato e
sperimentato), e lo condiionano a loro volta attraverso la proiezione.

Sensazione -\textgreater{} Esperienza mentale -\textgreater{}
Interpretata come relazione con l'oggetto\\
Allattamento -\textgreater{} soddisfazione -\textgreater{} oggetto
benevolo nello stomaco (con volontà propria e intenzionalità)

Perché il seno come primo oggetto? È il prototipo della fantasia di un
\textbf{oggetto parziale} inoltre i bambini percepiscono poco, non
l'intero, e sono dominati dal principio del piacere.\\
È una modalità prima di simbolizzare (es violincellista
suonare=masturbarsi)

La differenza tra mamma e papà è il seno e il pene.\\
L'atto primario è Nascosto, quindi interessante, si rappresentano quindi
il padre con l'oggetto interessante ovvero il pene.\\
Gli oggetti parziali sono deformazioni prodotte dalle fantasie, che
crescendo si adattano alla realtà, fino ad arrivare alla
rappresentazione completa, passando però per la \textbf{rappresentazione
parentale combinata} ovvero:

Il bambino è curioso sul contenuto del corpo materno, che potrebbe
essere anche il pene paterno (atto primario), o tutto il papà. Il padre
potrebbe contenere il seno, insomma c'è confusione e il passaggio
intermedio è costituito da unione tra i genitori \textbf{dal quale il
figlio è escluso}.

Esistono diverse \textbf{posizioni} secondo la Klein, a differenza degli
stadi proposti da Freud.\\
Le posizioni descrivono configurazioni tra tipo di rapporto oggeettuale,
l'angoscia avvertita e le difese attivate da essa.\\
Sono di fatto modi di organizzare esperienze.

Primi 6 mesi posizione schizoparanoide (schizo = scisso)\\
Fino all'anno posizione depressiva

In un lutto grave si riattiva la posizione depressiva, ma se viene preso
male può attivarsi anche la paranoide.

\textbf{La posizione schizo-paranoide}

L'io primitivo proietta nell'oggetto la pulsione di morte (come
Freud?)\\
La pulsione di vita è proiettata pure verso l'esterno.\\
Queste proiezioni sono accompagnate da scissioni dell'Io, che le crea
per proteggersi dall'ambivalenza interna di amore e odio.

Le difese primitive:

\begin{itemize}
\tightlist
\item
  Introiezione
\item
  Scissione
\item
  Proiezione
\item
  Identificazione proiettiva: vengono proiettati oltre agli impulsi
  parti di sé e ciò fa sentire all'altro ciò che proietta, per
  controllarlo. Ciò provoca il cosiddetto contro-transfert, ovvero ciò
  che prova l'analista in presenza del paziente, o ciò che il paziente
  genera negli altri. Si trova spesso in psicotici (non è teoria
  Kleiniana, è dei successori)\\
\item
  Idealizzazione
\item
  Dinego psicotico: scotomizzo una parte della realtà
\end{itemize}

\end{document}
