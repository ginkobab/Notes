% Options for packages loaded elsewhere
\PassOptionsToPackage{unicode}{hyperref}
\PassOptionsToPackage{hyphens}{url}
%
\documentclass[
]{article}
\date{14 Ottobre 2019}
\title{Lezione 5}
\author{Dinamica}
\usepackage{lmodern}
\usepackage{amssymb,amsmath}
\usepackage{ifxetex,ifluatex}
\ifnum 0\ifxetex 1\fi\ifluatex 1\fi=0 % if pdftex
  \usepackage[T1]{fontenc}
  \usepackage[utf8]{inputenc}
  \usepackage{textcomp} % provide euro and other symbols
\else % if luatex or xetex
  \usepackage{unicode-math}
  \defaultfontfeatures{Scale=MatchLowercase}
  \defaultfontfeatures[\rmfamily]{Ligatures=TeX,Scale=1}
\fi
% Use upquote if available, for straight quotes in verbatim environments
\IfFileExists{upquote.sty}{\usepackage{upquote}}{}
\IfFileExists{microtype.sty}{% use microtype if available
  \usepackage[]{microtype}
  \UseMicrotypeSet[protrusion]{basicmath} % disable protrusion for tt fonts
}{}
\makeatletter
\@ifundefined{KOMAClassName}{% if non-KOMA class
  \IfFileExists{parskip.sty}{%
    \usepackage{parskip}
  }{% else
    \setlength{\parindent}{0pt}
    \setlength{\parskip}{6pt plus 2pt minus 1pt}}
}{% if KOMA class
  \KOMAoptions{parskip=half}}
\makeatother
\usepackage{xcolor}
\IfFileExists{xurl.sty}{\usepackage{xurl}}{} % add URL line breaks if available
\IfFileExists{bookmark.sty}{\usepackage{bookmark}}{\usepackage{hyperref}}
\hypersetup{
  hidelinks,
  pdfcreator={LaTeX via pandoc}}
\urlstyle{same} % disable monospaced font for URLs
\usepackage[normalem]{ulem}
% Avoid problems with \sout in headers with hyperref
\pdfstringdefDisableCommands{\renewcommand{\sout}{}}
\setlength{\emergencystretch}{3em} % prevent overfull lines
\providecommand{\tightlist}{%
  \setlength{\itemsep}{0pt}\setlength{\parskip}{0pt}}
\setcounter{secnumdepth}{-\maxdimen} % remove section numbering


\begin{document}

\maketitle

\section{Klein}
Meredith Klein: avrebbe dato vita alla pressione matriarcale della
dottrina del peccato originale.\\
L'indagine di Freud sulla psiche inconscia ha rilevato aspetti
insospettati della vita infantile, ma prima della Klein furono fatti
poco tentativi di convalidare le teorie di Freud direttamente.\\
Era necessario utilizzare tecniche speciali

Nell'ultimo capitolo di Freud c'è accenno anche a Ferenzi e Abraham. La
Klein viene introdotta alla psicanalisi da Ferenzi e successivamente con
Abraham.

Storia su Klein:\\
era ebrea, è sfuggita ai nazisti, venendo invitata alla società
psicanalitica inglese, si trasferisce a Londra. Viene molto apprezzata,
ma nel 37 arrivano Freud e Anna Freud. Anna e Klein erano molto in
conflitto (teorico), al punto che durante i bombardamenti gli
psicanalisti rimanerono a Londra e le due donne continuavano a discutere
incuranti delle bombe, nonostante fossero viennesi (tipicamente
inibiti). Costrinsero la società psicanalitica inglese a scindersi in 2
e poi 3 filoni teorici: Il gruppo A per Klein, B per Anna Freud, e il
gruppo C, più fertile, misto di entrambi e nato successivamente.\\
Ebbe illustrissimi seguaci, tra cui Pior?\\
La tecnica psicanalitica coi bambini, la tecnica del gioco, anche se
modificata, viene ancora utilizzata.

Note biografiche:\\
Inizia a studiare medicina ma non la finisce (lei e Anna furono le
pochissime psicanaliste non medici, Freud si batté molto perché ciò
accadesse, tutt'oggi in alcuni stati devono essere medici)\\
La sua vita è costellata di lutti: parte con la morte del fratello
minore, che la spinge a sposare suo marito, nasce la primogenita che si
unisce teoricamente ad Anna Freud, cosa terribile per Klein.\\
In seguito va a Budapest per il lavoro del marito, dove incontra
Ferenzi, che la incoraggia ad utilizzare l'analisi con i bambini.\\
Dopodiché si sposta a Berlino, lavorando con Abraham, entra nella
società di Berlino, e viene incoraggiata dal maestro a trattare anche
gli psicotici, lascia il marito e muore Abraham.\\
Si trasferisce a Londra e muore Ferenzi e poi un suo figlio, e alla fine
muore anche lei.

Ha avuto pazienti tra i tra i 2 e i 14 anni, e qualche adulto.\\
Da questi pazienti Klein teorizza sullo sviluppo del primo anno di vita
(ripete l'errore di Freud)\\
Pensano di ricondurre teorie su bambini seriamente malati a bambini più
piccoli e normali. Klein ha anche analizzato i propri figli.\\
Osserva come i bambini mettono in atto attraverso il gioco tutta una
serie di fantasie, che attribuisce al primo anno di vita, e estende
l'esistenza di queste fantasie anche all'età adulta per pazienti
disturbati, anche se ritiene che tutti noi durante il corso della nostra
vita possiamo regredire a queste fantasie.

\textbf{La tecnica del gioco}\\
Lo spazio è arredato in modo che il bambino possa muoversi liberamente e
se occorre esprimere la propria aggressività senza farsi male.\\
Ci sono oggetti per il gioco non strutturati, in modo che il bambino
possa esprimere le proprie fantasie.\\
Un divano, lavandino con recipienti, piccole bambole e bambolotti
(piccole per essere `della stessa grandezza delle fantasie').\\
Un bambino invitato al gioco per lo meno osserverà i giochi e li
toccherà.\\
Klein per prima ha osservato che un bambino che non gioca e sono inibiti
hanno dei problemi.\\
Il modo in cui il bambino gioca permette di osservare un primo sguardo
alla sua vita psichica: è il modo in cui esternalizza il proprio mondo
interno.\\
Nel gioco agisce anziché parlare. È di fatto il corrispettivo della
libera associazione per gli adulti.\\
Nel gioco i bambini riproducono simbolicamente fantasie interne e
\sout{innate}.\\
Sono nello stesso linguaggio e nella stessa modalità espressiva
filogeneticamente ereditata dei sogni. È un aspetto molto criticato.\\
Il simbolismo però entra solamente in parte nel gioco, dobbiamo renderci
conto di tutti i processi di rappresentazione che lavorano nel processo
ludico.

Il gioco secondo la Klein esprime le preoccupazioni, conflitti, paure e
fantasie del bambino. Si analizza come fossero sogni e fantasie libere.
Sta interpretando i contenuti del mondo interno.\\
Altri disegni rientrano in ciò che la Klein analizza. L'obiettivo è
quello di far affiorare, come nell'adulto, i conflitti inconsci alla
coscienza.\\
Si usano le stesse regole, ad eccezione del lettino, della psicanalisi
con gli adulti.\\
Il trasfert per Freud è il processo chiave, e lo è anche per la Klein.
Questo transfert può essere sia positivo che negativo, ma sarà sempre
utile.\\
La variabilità di situazioni emozionali esprimibili dal gioco è
illimitata. Ad esempio la gelosia, il piacere di avere compagni di
gioco, sentimenti di amore-odio, angoscia per la nascita di un fratello,
colpa, desiderio di \sout{riparazione}.\\
Troviamo spesso ripetizione e residui diurni, mescolati con le
\textbf{fantasie}, che per la Klein compongono la mente umana, in
particolare quella dei bambini, e sono inconsce.\\
Ha notato, per esempio in Fretz (pseudonimo del figlio), che si
difendeva dalle angosce: per esempio lui non riusciva a fare le
divisioni. Un giorno gli raccontò che il modo in cui vedeva la divisione
era legato al fare male a sua madre, e quindi la divisione era inibita.

L'accusa che le viene fatta è di non sviluppare un linguaggio adatto a
descrivere le fantasie, ma di spiegarle in modo piuttosto dubbioso.\\
Una cosa che viene utilizzata in seguito è il suo parlare di mondo
interno e mondo esterno. Secondo lei il bambino ha un istinto
epistemofilo, ovvero una naturale tendenza a scoprire il mondo esterno,
e il gioco ne fa parte. Inoltre permette al bambino di trovare sollievo
esprimendo i propri conflitti.\\
L'angoscia del bambino viene esternalizzata nel gioco, la Klein osserva,
il bambino avrà una reazione, e il ciclo ricomincia.

Secondo la Klein nasciamo alla ricerca di un Seno. Abbiamo una fantasia
innata filogeneticamente data perché il seno è il nostro primo oggetto.

\textbf{Il conflitto Klein-Anna}\\
Klein ha la necessità di mantenere una continuità con Freud, ma allo
stesso tempo di evolverlo.\\
Sulla tecnica del gioco, secondo Melanie i bambini sono analizzabili,
secondo Anna invece no, l'unica cosa che si può fare è una pedagogia per
rafforzare il super-io in formazione.\\
Secondo Melanie l'esternalizzazione cognitiva di configurazioni
psichiche, ovvero di oggetti interni, esterni e dell'Io avviene
attraverso il gioco. L'oggetto interno è la versione di oggetto esterno
che il bambino ha internalizzato.\\
Secondo Anna l'Io è ancora debole, ciò che osserviamo sono le pulsioni
in atto, non l'Io (Melanie non parla di pulsioni).\\
Anna rifiuta l'idea che la psicanalisi si svolga nel bambino come
nell'adulto, non possono secondo lei sviluppare il transfert perché
hanno ancora investito libido sui genitori. Secondo Klein non è il
rapporto con i genitori reali ad essere trasferito, ma il rapporto con
la figura fantasmatica interna. Infatti ciò che conta è la realtà
psichica, non la realtà esterna di per sé.\\
Per Melanie i bambini soffrono di angoscie e sentono bisogno di aiuto
quanto gli adulti. L'analisi è rivolta verso l'interno del bambino. I
genitori interni sono spesso scissi tra genitori ideali e figure cattive
e persecutorie (meccanismo di difesa: scissione \emph{aggiunto al
pokedex}) oscillo fra ammore e odio (qua sembra Bettelheim). Un esempio
sono gli amici mega appiccicosi che appena un giorno non hai tempo per
lui diventi il male assoluto e ti evitano per il resto della vita.\\
Un altro esempio: ogni tanto i prof vanno a fare esami in carcere. La
collega della prof è estremamente aggressiva, coraggiosa, diretta.
Devono fare un esame a un tipo che controllando su internet scoprono
essere una bestia di satana (letteralmente, faceva parte di un gruppo
che ha rapito una coppia, uccisi, stuprato la ragazza, urinato sulla
tomba dove li hanno sepolti e fatto un qualche rituale a Satana). La
collega della prof era terrorizzata. La Tagini le propone di fare una
domanda sull'aggressività di gruppo, le mettono nella stanza della
Messa. La collega sempre più agitata. La guardia dice che non resta con
loro, lei va in sempre più in palla. Entra il tipo, un omino piccolo e
gobbo, completamente inerme. Domanda sull'attaccamento. Il tipo fissa
tutto il tempo la collega. ``Da quello che ho capito io i bambini
trattati male dai genitori succede che poi arrivano all'età adulta e
fanno cose brutte''. Alla fine la collega le racconta che ha
interpretato tutto quella che faceva come se fosse uno psicopatico,
mentre la prof ha visto un anatroccolo vulnerabile. Entrambe hanno visto
un aspetto esistente, e questa è la scissione: vedere due parti opposte
non integrate in una persona.

Il bambino si difende dall'ambivalenza interiore attraverso la
scissione. Per esempio mamma bene papà male. I metodi educativi sono del
tutto incompatibili con la psiche del bambino, Anna propone al più un
\emph{addestramento}.

Scissione: anche l'Io si scinde. La scissione tra le vedute della prof e
della collega sulla bestia di Satana, è anche provocata daal fatto che
anche la bestia ha proiettato due parti di sé distinte sulle prof (non
si proietta mai nel vuoto secondo Freud, c'è sempre un fondo di verità)

Il transfert negativo da sollievo al bambino appena il terapista
riconosce la causa del suo malessere.\\
Secondo la Klein il sottosviluppo dell'Io significa che sono più
soggetti al dominio dell'inconscio, e ciò aiuta il processo analitico
perché è più facile accedere all'inconscio.\\
Per i Freudiani infatti nella terapia vanno smontate le difese, per
esempio attraverso la libera associazione, mentre con i bambini è più
facile, perché l'inconscio è visibile, e le difese, facendo parte
dell'Io, sono ancora sottosviluppate.

Le cose che la Klein attribuisce ai bambini nei primi 12 mesi di vita
secondo lei dominano la vita dei pazienti psicotici. Le fantasie del
bambini piccoli sono terrori psicotici, e la mente è sempre in grado di
reagire alle interpretazioni profonde.\\
Profonde significa che si trovano in basso secondo le topiche
Freudiane.\\
Secondo Anna Freud la psicanalisi deve scendere pian piano, secondo i
Kleiniani bisogna andare subito nel profondo.\\
Anna Freud infatti è una psicologa dell'Io, mentre Klein dell'Es.

Un'ulteriore differenza è che i bambini disturbati osservati dalla Klein
hanno anche dei sensi di colpa e delle fantasie edipiche piuttosto
violente a età piuttosto precoci (per Freud 4-6 anni), tipo di 2 anni.
Quindi postula che esistano forme pregenitali del complesso di Edipo e
del Super-Io\\
Mentre per Anna Freud il bambino teme il genitore vero, non quello
introiettato.

Analizzo super-io e angoscia e senso di colpa (Klein).\\
In altre parole Anna aveva una sorta di modello educativo, mentre Klein
sosteneva la possibilità di analizzare i bambini.

La Klein parla del primo anno di vita. Come concepisce lo sviluppo? Il
mondo viene rappresentato attraverso il proprio corpo, immagini
corporee. Esiste per la Klein un'aggressività innata nel bambino,
filogeneticamente determinata, se io proietto temo che gli oggetti mi
distruggano (paranoia).\\
Il seno materno rappresenta il primo oggetto del bambino, e quindi il
primo oggetto del mondo esterno.\\
Amore e odio (la riconcettualizzazione della pulsione di vita e di
morte, trasformate dalla Klein in emozioni da pulsioni che erano) si
sperimentano attraverso gli oggetti.\\
Alla nascita secondo la Klein c'è già l'Io. È rudimentale, dispersivo e
poco coeso, ma non solo opera le difese, ma entra in contatto con gli
oggetti. Ciò che mi fa entrare in contatto con gli oggetti è l'Io. Tra
queste difese, introiezione e proiezione, sono quelle che permettono
all'Io di entrare in contatto con gli oggetti.\\
Le emozioni solo legate al mondo esterno sin dalla nascita. Quindi il
bambino è orientato verso la realtà fin dalla nascita, quindi No fase
narcisistica. Introietto portando dentro, Proietto espellendo.
(Respirazione Kleiniana) La Klein è quindi chiamata \textbf{La prima
teorica delle relazioni oggettuali} o anche \textbf{La prima dominatrice
dell'Aria}

\end{document}
