\documentclass[12pt, a4paper]{article}

\date{2 Dicembre 2019}
\title{Lezione 30}
\author{Dinamica}

\begin{document}

\maketitle

\section{L'Antipsichiatria}

\begin{itemize}
    \item Rifiutano il concetto stesso di malattia mentale, che si basa solo su segni (esperimento di Rosenhan).
    \item Criticano il mandato sociale dello psichiatra (potere di far rinchiudere)
\end{itemize}

\section{Laing}
Si approccia ai malati mentali sostenendo che siano comprensibili, attraverso la fenomenologia: si immedesima nell'esperienza psicotica del pazien\-te.
\medskip\\ 
Successivamente, facendosi influenzare dalla scuola di Palo Alto (Watzlavisk, malattia mentale come ambiente familiare malato).
La patologia quindi è nella struttura familiare secondo lui, non nel paziente (è un'iper\-semp\-lificazione), vede la follia come trip per scoprire se stessi.
\medskip\\ 
Laing ha contestato una serie di fenomeni che andavano contestati, ma attraverso teorie semplificatorie, generalizzanti ed estreme.
\medskip\\ 
Critica per esempio il fatto che persone folli non vengano rinchiuse solo perché la loro follia è socialmente accettata (piloti di bombardieri)
\paragraph{Kraeplin}  La decontestualizzazione:
\begin{quote}
    \emph{Presenta una serva emaciata, fa cose insensate, se la si prova a fermare reagisce molto violentemente}
\end{quote}
Secondo Laing il comportamento di Kraeplin, se decontestualizzato è altrettanto folle.
\medskip\\ 
Laing nega la malattia e la ridefinisce come una ``posizione esistenziale'' del paziente, col quale cerca di immedesimarsi.
\medskip\\ 
La psicoterapia è una cura relazionale (come Sullivan), deriva dalla conoscen\-za e dalla comprensione di questi stati esistenziali (insicurezza ontologica primaria), che impediscono la reazione degli altri.
\medskip\\ 
L'insicurezza si manifesta come scissione a 2 livelli: tra sé e altri e da psiche e soma. \\
Qua Laing si fa influenzare dai fenomenologi, che mettono in risalto l'ambi\-va\-le\-nza esistenziale del corpo, che da una parte è un oggetto nel mondo, dall'altra fa parte di noi. \\
Con questa scissione si crea un disturbo di identità, e cercare di relazionars iporta il rischio di perdersi.
\medskip\\ 
Secondo Laing ci sono 3 forme di angoscia, che rappresentano minacce relazionarie:
\paragraph{Il Risucchio} Relazionarsi diventa pericoloso, perché la propria instabilità gli fa temere costantemente di perdere nel rapporto, come se dovesse continuamente strenuamente combattere per non annegare nell'altro
\paragraph{L'Implosione} Terrore di sentire la realtà come qualcosa che può annichilire e inghiottire i confini dell'identità personale. Senso di essere nulla.
\paragraph{La Pietrificazione} Detta anche Spersonalizzazione, l'atto di trasformare l'alt\-ro in cosa, senza soggettività, non riconoscere l'autonomia, nel tentativo di affermare la propria individualità, ma invece genera un circolo vizioso, inducendo l'individuo all'isolamento.
\medskip\\
Laing concepisce il Vero Io come astratto, (!= Winnie), il che spiega la sensazione di essere distaccati cal corpo degli schizofrenici.
Il Falso Io è nel corpo, al contrario di Winnie, è nel coro.

\subsection{Periodo intermedio}

Secondo Laing l'irrazionalità deriva dalla famiglia, quindi è la famiglia ad aver bisogno di essere curata. 
Per dimostrarlo prendono 11 donne diagnosticate schizofreniche e fanno colloqui con tutti i parenti.
Caso di Maya, secondo lui, che ipersemplifica, i genitori, malinterpretando la sua crescita, inducono in lei la malattia.
\medskip\\ 
Questo porta all'idea di comunità terapeutica, per staccare i ragazzi dalla famiglia (Kingsley Hall)
















\end{document}
