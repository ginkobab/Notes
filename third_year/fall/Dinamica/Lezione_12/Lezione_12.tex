\documentclass[12pt, a4paper]{article}

\title{Lezione 12}
\date{31 Ottobre 2019}
\author{Dinamica}

\begin{document}
\maketitle

\section{Mahler - Seconda Parte}
\paragraph{Karl Popper} aveva criticato la psicanalisi sostenendo che non fosse falsificabile. Secondo Freud invece l'ipotesi viene falsificata quando il paziente è indifferente ad essa.

\paragraph{Ad esempio} in un esperimento con gruppi indifferenti/omofobi esposti a filmati pornografici omosessuali, gli omofobi si eccitavano di pi\'u.

\paragraph{La madre di Mahler} aveva partorito a 19 anni, non la voleva e si rifiutava di allattarla. Non le permetteva di entrare in cucina, la trattava male. Quando Mahler aveva 4 anni nacque una sorella, che la madre adorava. Mahler osservava la loro relazione, e si identificò con il padre, che però voleva un figlio. La prima coppia madre-bambino che pot\'e quindi osservare fu quella di sua madre e sua sorella.

\paragraph{Le fasi} identificate da Mahler sono autistica e simbiotica: nella patologia \`e molto facile osservare diadi madre-bambino.

\paragraph{Una poesia} di Ted Hughes: Crow and Mama. \emph{Dalla mamma non si sfugge.}

\paragraph{I processi intrapsichici} che succedono la fase simbiolica sono detti di \emph{separazione}, ovvero emergere dalla simbiosi materna, e \emph{individuazione}, ovvero lo sviluppo dell'individualit\`a.

La madre reagissce ai bisogni differenziali nelle diverse fasi: funge da Io ausiliario, media il rapporto con il mondo esterno. Successivamente deve staccarsi sempre di pi\`u. Descrive un movimento parallelo madre-bambino.  

\paragraph{Le sottofasi} dipendono dalla madre, se \`e distratta sono pi\`u rapide per esempio

\paragraph{Sotto-fasi} di individuazione e separazione:
\begin{itemize}
    \item \textbf{Differenziazione}
    \item 4--5 fino a 10 mesi: incubazione, attenzione alll'esterno
    \item Esplorazione della madre: compare il sorriso specifico
    \item Esplorazione di stimoli pi\`u lontani della madre (esterni alla diade)
    \item Si allontana progressivamente dalla madre
    \item Io diventa capace di distinguere interno/esterno
    \item 6 mesi: compare \emph{l'angoscia dell'estraneo}: distingue tra gli oggetti esterni
    \item \textbf{Sperimentazione}
    \item 1o periodo: 10 mesi: la capacit\`a locomotoria facilita l'allontanamento dalla madre, che diventa una casa base.
    \item 2o periodo: nascita psicologica proprimante detta, caratterizzata da deambulazione eretta. Madri iperprotettive hanno problemi in questa fase.
    \item \textbf{Riavvicinamento}
    \item 15--18--24 mesi: riscontro dei limiti del mondo reale, corrisponde al passaggio dalla fase orale a quella anale.
    \item Compare \emph{l'angoscia di separazione} dalla figura materna, rendendosi contro che la madre \`e separata
    \item Matura il linguaggio
    \item Atteggiamento ambivalente del bambino: ha bisogno della madre ma nega la dipendenza
    \item \textbf{Separazione e Individuazione}
    \item Hanno strette interconnessioni ma non si sovrappongono
    \item Per esempio una madre simbiotica nuoce al processo di separazione e individuazione
    \item Hartman aveva parlato di ambiente mediamente prevedibile, la ``realt\`a'', necessario per il bambino. Per Mahler l'ambiente viene specificato come ``quella madre di quel bambino''.
    \item Viene rifolmulato il concetto di narcisismo, ovvero un investimento di unit\`a duale.
    \item Il bambino si adatta alle esigenze del caregiver conscie e inconscie
\end{itemize}

\paragraph{L'angoscia} da separazione e dell'estraneo sono per la Mahler normali, e non averle \`e patologico.

\paragraph{Borderline} sono pazienti apparentemente normali, che mostrano sintomi psicotici sul lettino, gli sono attribuite tante caratteristiche tra cui aggressivit\`a esplosiva, tentativi di suicidio, ecc\ldots

\section{Gli Indipendenti}

Decisero di non schierarsi n\'e con Anna n\'e con Klein, tra di loro c'\`e Bowlby (odiato da tutti).

\paragraph{Attingono} in parte dalla Psicologia dell'Io, dai Kleinani e reinterpretano creativamente Freud.

\begin{quote}
    \flushleft{``Gli adulti maturi infondono vitalit\`a in ciò che \`e antico, vecchio e ortodosso, recreandolo dopo averlo distrutto''}\\
    \raggedleft{Winnie, 1965, p.94}
\end{quote}

\paragraph{Punti teorici}
\begin{itemize}
    \item Relazione oggettuali rispetto a pulsioni
    \item Importanza delle cure materne, della madre \emph{reale}
    \item Psicopatologia \`e il risultato di carenze ambientali
    \item Rifiuto della pulsione di morte
    \item Derivati della pulsioni (aggressivit\`a, zone erogene, ecc\ldots) sono reazioni ad un ambiente deficitario
    \item Attenzione allo sviluppo del s\'e
    \item Parte del processo terapeutico deve essere una regressione del bambino al momento in cui sono mancate le cure materne.
    \item La regressione non \`e quindi solo maligna.
\end{itemize}

\paragraph{A differenza} del modello classico e Kleinano il s\'e \`e sovrapposto all'altro reale, non in una relazione con l'oggetto fisso oppure no come in precedenza.

Questo \`e fondamentale per la relazione terapista/paziente, che non può in quest'ultima visione essere effettivamente oggettivo come supponeva Freud.

\begin{quote}
    \flushleft{``La psicanalisi \`e come una puttana, \`e l\`i per essere usata.''}\\ 
    \raggedleft{Winnie, 1965}
\end{quote}

\paragraph{Biografia} Winni nasce in una famiglia dominata da donne, studia medicina e diventa pediatra, poi si dedica alla psicanalisi. La sua seconda analista fu la Klein. Si diceva che quando un bambino disturbato entrava nel suo studio (da pediatra), immediatamente il bambino cambiava atteggiamento, come se entrasse in una risonanza naturale. Come analista adulto ha fatto qualche casino. Era estremamente creativo e si divertiva a distorcere il pensiero
Freudiano. 

Non ha mai avuto figli propri, forse era impotente. La sensazione che si ha leggendo di come lavorava con i bambini \`e che fosse rimasto un po' un bamino.

\subsection{La base del pensiero di Winnie}

\paragraph{Idealizzava} completamente la relazione madre-bambino. Dimenticava il padre, tranne per l'atto sessuale con la madre, che costituisce una fonte di fantasie per il bambino. 

\paragraph{La madre} come oggetto reale, crea e permette al bambino di crescere, in una maniera molto meccanicistica.

\paragraph{Lo sviluppo sano} coincide con la \emph{capacit\`a di essere creativi}, ovvero di essere s\'e stessi e di produrre qualcosa che ci rifletta.

\paragraph{L'ambiente} non produce al bambino, al massimo permette a questi di realizzare il proprio potenziale

\paragraph{Il paradosso} di Winnie: la madre crea il bambino ma non la proiezione:
\medskip
\begin{quote}\flushleft{\emph{``Quando il bambino guarda negli occhi della madre, quello che vede deve avere a che fare con s\'e, non con la madre''}} \\ \raggedleft{Winnie}
\end{quote}
\medskip

\paragraph{Le madri} per Winnie \`e meglio se non sono perfette ma sufficientemente buone, ovvero può proiettare ma poco.

\subparagraph{Le tre fasi - Dalla dipendenza all'indipendenza}

\begin{itemize}
    \item Dipendenza: Il neonato non può neanche essere descritto senza che sia in relazione con qualcuno
    \item Dipendenza relativa: Intorno ai sei mesi c'\`e un graduale emergere dall'ambiente, viene meno l'adattamento materno: la \emph{preoccupazione materna primaria} viene meno, ovvero l'identificazione della madre con figlio scompare permettendo il discioglimento dell'identit\`a duale madre-figlio. La madre \textbf{deve iniziare a fallire nella risposta ai bisogni del figlio}
    \item Indipendenza: madre introiettata
\end{itemize}

Sia le esplorazioni freudiane della nevrosi, che quelle kleinane della depressione danno per scontato che esista ``una persona'', ovvero una personalit\`a unitaria stabile.

Esistono infatti persone che ``sembrano'' persone, ovvero hanno un falso s\'e.

Per diventare una persona serve che la relazione con la madre fosse ``sufficientemente buona''.

Questo avviene attraverso 3 fasi:

\paragraph{Dalla non-integrazione all'integrazione} dal non-io all'io sono
\begin{itemize}
    \item All'inizio l'Io \`e una sorta di potenziale la cui realizzazione dipende da un cervello intatto
    \item Esistono solo brandelli di esperienze
    \item L'io raccoglie informazioni
    \item La madre ha la funzione di integrare l'Io non coeso del bambino, detta contenimento
    \item All'inizio il bambino \`e un insieme di funzioni motorie e sensoriali, però ha un potenziale per lo sviluppo
    \item Perch\'e il bambino possa iniziare a fare esperienze costruttive, il bambino deve essere protetto da ``urti'', ovvero traumim, esperienze inaspettate, e questo \`e compito della madre.
\end{itemize}

\paragraph{Tutti} i bambini quando non sono impegnati ad elaborare bisogni e stimoli hanno bisogno di una presenza materna sullo sfondo che permette a loro di dare continuit\`a alla propria esperienza del mondo.

\paragraph{Soltanto} quando il bambino smette di avere bisogni può fare esperienza della propria esistenza.
Essere solo significa assenza dell'oggetto, ma può essere sperimentato con un oggetto ``sullo sfondo''. Un esempio \`e lo stato post-orgasmo.

\paragraph{Si crea} in questo modo uno stato di \textbf{non integrazione}, e con esso il primo nucleo del s\'e, che secondo di lui \`e un senso di s\'e inviolabile, incomunicabile, sacro.

\paragraph{\`E il primo} autore che facciamo in cui le pulsioni sono secondarie rispetto ai bisogni

\end{document}
