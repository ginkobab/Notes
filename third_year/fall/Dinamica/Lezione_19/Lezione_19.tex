\documentclass[12pt, a4paper]{article}

\date{18 Novembre 2019}
\title{Lezione 19}
\author{Dinamica}

\begin{document}

\maketitle

\section{Matte-Blanco - Parte Seconda}

\paragraph{Matte e Bion}  Ci sono delle similitudini
\begin{itemize}
    \item Come è possibile trasformare ciò che impensabile in ciò che è pensabile?
    \item Le percezioni vanno trasformate per essere pensate
    \item Hanno lavorato entrambi con pazienti schizofrenici (producono pensieri senza significato, potrebbe spiegare la similitudine nelle teorie
    \item Perché qualcosa non riesce di diventare cosciente?
        \begin{itemize}
            \item In matte blanco: La coscienza non è \textbf{strutturata} in modo tale da poter contenere l'inconscio. Il terapeuta deve quindi aiutare a trasformare ciò che è incontenibile
            Il sogno non è conoscibile perché è simmetrico (è l'elaborazione secondaria a trasformarlo in conoscibile
            Il dispiegamento va quindi a ritrovare il materiale simmetrico
            \item  In Bion l'irrompere nella simmetria nella mente del paziente è il ``terrore senza nome''
        \end{itemize}

\end{itemize}

\subsection{Implicazioni generali} 

L'emozione è un miscuglio tra asimmetria e simmetria: ogni emozione porta un pensiero che ci permette di conoscere.

Matte-Blanco arriva a sostenere che noi ontologicamente miscugli di simmetria e asimmetria.

\bigskip
\begin{quote}
    \emph{È la natura dell'uomo che appare costituita da una parte generalizzante, che ci immette nei simboli, e una parte limitante che conduce al particolare. Solo nell'interazione possiamo capire i fenomeni umani}
    \begin{flushright}
        Matte-Blanco, 1975
    \end{flushright}
\end{quote}

\begin{quote}
    \emph{Vi è la possibilità che la natura più profonda e fondamentale del nostro essere è tutta simmetrica e che la coscienza ne emerge come una potenzialità più ristretta, come delle ``limitate incarnazioni'' di una realtà più vaasta, proprio come un individuo è una piccola incarnazione di una funzione proposizionale}
    \begin{flushright}
        Matte-Blanco
    \end{flushright}
\end{quote}


\section{Heinz Kohut}

Studia pazienti \textbf{narcisisti}. 
Sono pazienti con \emph{fantasie onnipotenti grandiose}, hanno costantemente bisogno di \emph{ammirazione e che rispecchino la loro immagine illusoria}. Molto spesso sono anche dei \emph{perversi}, perché l'altro svolge solo funzioni, non è un essere umano.

Si comportano in maniera completamente diversa dai nevrotici. 

\paragraph{Svolta teorica}  Da pazienti conflittuali, a pazienti con un \textbf{sé discontinuo}, l'uomo tragico.

\begin{quote}
    \emph{Ciò che ci tiene insieme è la risposta senza riserbo di coloro che ci circondano nel primo periodo di vita. E questa va di là della colpa e al di là della ricompensa}
    \begin{flushright}
        Kohut, 1976
    \end{flushright}
\end{quote}

\paragraph{Biografica}  Nasce a vienna da genitori ebrei molto artistici, pianisti. La madre era simbiotica esasperatamente. Il padre inizialmente un ottima fi\-gu\-ra, ma torna dalla prima guerra mondiale cambiato in peggio, morendo poi presto, lasciandolo in balìa della madre. Sviluppa con una relazione anche sessuale col suo maestro di pianoforte.

Studia medicina a Vienna, e l'ultimo anno arrivano i Nazisti. Ha un altro grande amico immigrato a Chicago, riesce ad andarci anche lui, si specializza in neurologia e psichiatria (con cattedre relative) e si affaccia all'istituto psicanalitico.

È stato respinto a lungo perché troppo narcisista, ma poi viene accettato, e diventa sempre più importante nell'associazione prima americana e poi internazionale, ma legandosi sempre a Freud padre e figlia.

Quando la madre muore lui sviluppa un pensiero proprio, la \textbf{psicologia del sé}

\subsection{Caso della signora F.}

Signora di 25 anni che si sentiva isolata, diversa. Soffriva di cambiamenti improvvisi di umore, associati alla sensazione che i suoi pensieri non fossero reali, sembrava cercare gli altri solo per regolare il suo umore. 

Nelle sedute si era sviluppato un pattern frustrante, F si arrabbiava con K per i suoi silenzi.

Col tempo K capì che gli bastava ripetere ciò che F diceva, la paziente si calmava, ma se aggiungeva qualcosa, F protestava.

Ipotizzò che F stesse mettendo in atto delle richieste infantili specifiche che non erano state soddisfatte. 

F necessitava risposte empatiche alle sue \emph{capacità} e ai suoi desideri di esser eapprovata, di apprezzamenti, di una sorta di eco.

Coniò il termine ``bisogno di rispecchiamento'' che nel caso di F. sembrò legato al suo bisogno di \textbf{aumentare l'autostima}

\begin{quote}
    \emph{I pazienti narcisistici non hanno un'autostima abbastanza alta}
\end{quote}
Nel bambino ciò è normale, nell'adulto è patologico.
Ovviamente anche nelle persone non patologiche gli apprezzamenti fanno piacere, ma queste non scelgono chi gli sta attorno solo e soltanto in base a quella funzione.

Con questa paziente K aveva spesso avvertito un senso di noia: sembrava che con F lui non fosse una persona ma dovesse svolgere una funzione psicologica. Era necessario comprendere empaticamente piuttosto che spiegare.

\paragraph{Connette}  il comportamento genitoriale con i disturbi dell'autostima. Per Kohler il terapista deve stare in una sorta di \emph{reverie} Bioniana, in empatia con il paziente.

\begin{quote}
    \emph{Prima che il bambino sia considerato distaccato dalla madre, que\-sto deve sviluppare funzioni autonome. Il terapista serve come funzione del narcisista perché questi non ha ancora alcune funzioni autonome.}
\end{quote}










\end{document}
