\documentclass[12pt, a4paper]{article}

\date{3 Dicembre 2019}
\title{Lezione 31}
\author{Dinamica}

\begin{document}

\maketitle

\section{Laing - Parte Seconda}

\paragraph{Kingsley Hall}  Laing decide di creare delle case a Londra, dove non vengono dati psicofarmaci, dove con i pazienti vivono psichiatri, psicanalisti e infermieri, e permettono ai pazienti di operare regressioni, l'unica regola è no violenza.

\subsection{Ultimo periodo di Laing}

La vera follia è riscontrabile nella società, quindi essere psicotici nella società malata, diventa la vera salute mentale.
\medskip\\ 
La cura non può quindi essere finalizzata alla reintegrazione (opposto di Sullivan).

\section{Infant Research}

\paragraph{Stern}  Studia l'interazione madre bambino normale tramite video registrazione e studio dei frame, va a criticare teorie dei maestri (Mahler ecc), permette di rivisitare le teorie tipo fase autistica e simbiotica.

\paragraph{Lichtenberg}  Teoria dei sistemi motivazionali che sostituisce quella sulle pulsioni.

\subsection{Ricerca contemporanea}

Parte dagli anni '80, tanti nuovi paradigmi e autori (es.\ still face per simulare mamma depressa che interagisce con il bambino)

Quasi tutti gli autori passati consideravano madre e bambino come indifferenziabili. Stern si chiede allora cosa ne è di tutte le conseguenze emotive di questa fusione (angoscie di risucchio, implosione e pietrificazione).
\medskip\\ 
Stern sostiene che la fusione sia una situazione speciale che si manifesta solo in alcuni casi, e non sono veramente fusi ma è solo una maniera speciale di stare con un'altra persona.
\medskip\\ 
Il bambino è quindi completamente in grado di sperimentare se stesso come separato.
\medskip\\ 
La psicanalisi inoltre non può essere l'unica fonte di dati su cui basare teorie cliniche, altrimenti si costruiscono teorie basate sui pazienti specifici.

Basare le teorie sulla clinica e verificarle nella clinica ci pone in una spiegazione tautologica.

Si associa a Peterforn nel criticare le teorie patomorfe.













\end{document}
