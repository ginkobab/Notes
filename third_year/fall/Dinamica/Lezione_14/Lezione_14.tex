\documentclass[12pt, a4paper]{article}

\date{5 Novembre 2019}
\title{Lezione 14}
\author{Dinamica}

\begin{document}

\maketitle

\section{Winnie - Parte Terza}

\begin{quote}
    \emph{Essere in grado di sperimentare gli oggetti come separati \`e legato anche al destino dell'aggressivit\`a}
    \begin{flushright}
        Klein
    \end{flushright}
\end{quote}

\paragraph{La posizione depressiva} secondo Winnie \`e il momento in cui si integra la madre oggetto e la madre ambiente, e per fare ciò \`e necessario il passaggio:

\begin{quote}
    \begin{center}
        Entrare in relazione con gli oggetti $\Rightarrow$ Usare gli oggetti
    \end{center}
\end{quote}

\paragraph{Per che ciò avvenga} \`e necessaria l'aggressivit\`a: il bambino deve sviluppare la \emph{capacit\`a di preoccuparsi}, ciò deve creare un nesso tra gli aspetti distruttivi e le altre componenti affettive, e distinguere gli altri da s\'e (puoi distruggere l'oggetto solo se questo \`e esterno, e per metterlo all'esterno questo \emph{va distrutto})

\paragraph{L'oggetto} deve sopravvivere alle aggressioni del bambino: ancora una volta Winnie sottolinea il compito materno.

\subparagraph{Ad esempio} Se il bambino tira un calcio alla madre, e lei si butta a terra e fa finta di morire, il bambino capisce che il rompere la simbiosi rischia di distruggere la madre, lo stesso avviene nell'adolescenza.

\subparagraph{Alcune madri} subiscono l'aggressivit\`a come un insulto personale, e questo crea problemi nel figlio.

\paragraph{L'individuo} \`e visto come una bolla, che si modifica in base alla pressione dell'ambiente.

\paragraph{La deformazione} del s\'e del bambino nascono da richieste dell'ambiente al bambino, e queste provocano la creazione di un falso s\'e. A questo punto l'unico modo per riacquisire il senso di s\'e \`e isolarsi dall'ambiente.


\subsection{Psicopatologia}

\begin{itemize}
    \item Disturbi mentali
    \item Disordini pre-s\'e (psicotici, schizoidi, borderline, falsi s\'e)
\end{itemize}


\paragraph{La colpa} di ciò \`e sempre nel fallimento sistematico materno.

Per esempio le madi psicotiche o che non sviluppano la preoccupazione primaria materna, come le madri depresse.

\paragraph{Successivamente} Winnie ricondurr\`a tutti i disturbi a vari gradi di formazione di un falso s\'e. Questo falso s\'e si estende da disturbi psicotici (disintegrazione, difese, ecc\ldots).

\subparagraph{Tutti abbiamo} un falso s\'e, che sono i ruoli che assumiamo e le buone maniere.

\paragraph{Adattarsi} alla perfezione alla realt\`a ambientale significa appiattirsi sul falso s\'e.

Anche sviluppare un'essessiva intellettualizzazione, schiacciando la spontaneit\`a e la fisicit\`a \`e una forma di falso s\'e.

\begin{quote}
    \emph{Il fascismo rappresenta un'alternativa permanente alla pubert\`a}
    \begin{flushright}
        Winnie, 1940
    \end{flushright}
\end{quote}


\subsection{Innovazioni}

\begin{itemize}
    \item Unit\`a madre-bambino, parallelismo
    \item Compiti svolti da madre
    \item Pulsione come ricerca dell'oggetto
    \item Desideri istintuali separati dall'emergere del s\'e
    \item Psicopatologia legata alla mancanza di spontaneit\`a
    \item Psicosi come disturbo di provocato da deficienza ambientale
    \item Regressione non avviene a punti di fissazione ma a momenti in cui l'ambiente \`e venuto a mancare
\end{itemize}

\paragraph{Innovazioni tecniche e teoriche}

\begin{itemize}
    \item Analista e setting diventano l'ambiente in cui il paziente si perde per ritrovarsi
    \item Analista viene ``usato'': analogia con il gioco
    \item La cosa naturale \`e il gioco, e il fenomeno equivalente \`e la psicanalisi
\end{itemize} 

\paragraph{Winnie vs Klein}

\begin{itemize}
    \item Da Klein: mondo interno, fantasie della realt\`a, ma non conoscenza filogenetica
    \item Istinto di morte invece \`e superfluo (aggressivit\`a \`e gi\`a distinta da distruttivit\`a)
    \item Madre reale acquisisce specificit\`a
    \item Posizione depressiva cambia (aggressione oggetti e integrazione madri)
\end{itemize}

\paragraph{Winnie vs Freud}

\begin{itemize}
    \item Fraintendimento sistematico, creativo
    \item La relazione diadica con la madre permette la soddisfazione istintuale e provocare altri bisogni (di significato)
    \item Contesta il narcisismo primario riconcettualizzandolo nella simbiosi
    \item Aggressivit\`a non equivale alla distruttivit\`a, ma alla vitalit\`a e motricit\`a
    \item La distruttivit\`a \`e quindi aggressivit\`a non modificata da un'adeguata relazione
\end{itemize}


























\end{document}
