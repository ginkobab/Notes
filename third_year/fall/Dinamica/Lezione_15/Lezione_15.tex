\documentclass[12pt, a4paper]{article}

\date{6 Novembre 2019}
\title{Lezione 15}
\author{Dinamica}

\begin{document}

\maketitle

\section{Fairbain}

L'io è legato agli oggetti

\subsection{Rifolmulazioni}

\begin{itemize}

    \item La libido \`e essenzialmente una ricerca di oggetto
    \item Zone erogene o fasi orali sono techniche impiegate dall'Io per regolare i rapporti con gli oggetti
    \item Teoria dello sviluppo in termini di relazioni con gli oggetti, in particolare relazioni con oggetti interiorizzati sotto la pressione deppa deprivazione e della frustrazione
    \item L'origine della psicopatologia sta nelle relazioni oggettuali (interiorizzazione di oggetti cattivi)
    \item Spiega perch\'e i nevrotici si aggrappino cos\`i tenacemente alle esperienze dolorose, in realt\`a si aggrappano a relazioni con oggetti cattivi
    \item Psicopatologia legata a relazioni disturbate con altri, non a conflitti inconsci tra senso di realt\`a e di piacere
    \item La ricerca del piacere \`e un mezzo per un fine, il piacere \`e indicatore dell'oggetto, una modalit.\`a per stabilire relazioni
\end{itemize}

\paragraph{Gli Oggetti} per Fairbairn sono esseri umani

\paragraph{In condizioni cattive} come di deprivazione emotiva, i bambini creano oggetti interni, compensazioni degli oggetti cattivi esterni.

\subparagraph{Psicologia:} studio relazioni \emph{individuo} con i suoi oggetti
\subparagraph{Psicopatologia:} studio relazioni dell'\emph{Io} con i suoi oggetti interiorizzati

\paragraph{La scissione interna} che consegue dall'interiorizzazione \`e una ripresa della Klein.

\paragraph{L'Io} all'inizio \`e unico e unitario, che interiorizzaziando di oggetti cattivi si frammenta. La differenza tra sano e non sano sta nella quantit\`a e nel grado di frammentazione

\paragraph{Questa teoria spiega} le osservazioni di Fairbain nei soggetti che osserva (schizofrenici)


\subsection{Le fasi}

\begin{enumerate}
    \item Stadio di dipendenza infantile
    \item Fase di transizione
    \item Stadio di dipendenza matura (reciprocit\`a)
\end{enumerate}

\paragraph{La dipendenza infantile} \`e uno stadio di \emph{fusione} con la madre, in continuit\`a con lo stato uterino, la madre costituisce l'ambiente.

La dipendenza, come per la Mahler, \`e caratterizzata dalla marcata differenziazione con l'ambiente (non \`e quindi narcisismo primario, come tutti gli autori che pongono enfasi sulle relazioni oggettuali lo rifiutano)

\subparagraph{Primo stadio} pre-ambivalente, dilemma del succhiare o non succhiare: entro in relazione o no? Viene chiamato dilemma schizoide, perch\'e anche essi fanno fatica a entrare in relazione

\subparagraph{Secondo stadio} dilemma del succhiare o mordere: dilemma depressivo della distruttivit\`a // Klein (i depressivi pensano di essere la causa del male)

\paragraph{La fase di transizione} Dall'indefferenziazione e le identificazioni primarie con oggetti reali esterni e oggetti interni compensatori: all'allentare la presa sugli oggetti avviene un conflitto evolutivo fra la spinta verso una dipendenza matura e la riluttanza di abbandonare legami con oggetti interni e esterni indefferenziati.

\paragraph{La dipendenza matura} La dipendenza \`e condizionale e c'\`e scelta tra diversi oggetti, le relazioni diventano cooperative e basate sulla reciprocit\`a, si arriva alla genalit\`a che \`e vista come canale per esprimere intimit\`a, ma non \`e segno di maturit\`a alla Freud.

\subsection{Il dilemma clinico}

Spesso i bambini abusati si considerano \textbf{cattivi}, l'altro \`e buono, costretto ad essere cattivo con me perch\'e io me lo merito.

\paragraph{Ciò accade perch\'e} per il bambino, riconoscere che il genitore sia cattivo significherebbe ammettere la propria totale impotenza, il suo essere in bal\`ia di un genitore dal quale dipende completamente che \`e cattivo e non si può fare nulla a riguardo.

\paragraph{Facendo ciò} il bambino crea una situazione di \textbf{cattiveria condizionale}, opposta ad una incondizionale, introiettando un oggetto cattivo e idealizzando l'oggetto esterno

\subsection{La struttura psichica}

\begin{itemize}
    \item Alla nascita Io integro e unitario alla ricerca di relazioni oggettuali
    \item Se le cure non soddisfano, l'io compensa interiorizzando oggetti
    \item Io scinde per parti dell'Io esterno che restano attaccate agli oggetti interni
    \item Quindi il primo oggetto \`e il seno materno
    \item In condizioni perfette in cui non c'\`e frustrazione, non insorge \ambivalenza
    \item Nella realt\`a c'\`e sempre un certo grado di frustrazione
    \item La frustrazione induce aggressivit\`a verso l'oggetto libidico
    \item L'ggetto diventa buono e cattivo, ambivalente
    \item Quindi avviene la scissione della madre in 2 oggetti esterni incontrollabili
    \item Il bambino trasferisce all'interno il fattore traumatico, la madre cattiva
\end{itemize}

\paragraph{L'oggetto cattivo} rimane cattivo anche dentro. L'oggetto cattivo ha 2 aspetti: frustrante e eccitante



\end{document}
