\documentclass[12pt, a4paper]{article}

\date{29 Ottobre 2019}
\title{Lezione 11}
\author{Dinamica}

\begin{document}

\maketitle

\section{Psicologia dell'Io - Parte Seconda}

Molti viennesi emigrati negli USA fanno parte di questa corrente.

\paragraph{Hartmann} lavora con lazienti schizofrenici. Giunge ad una posizione opposta alla Klein: la valutazione della realt\`a.

Per poter modificare la teoria psicanalitica in modo da dare pi\`u importanza all'Io, questi autori hanno dovuto cambiare alcuni costrutti.

Nella prima postulazione di Freud questi si era dimostrato concentrato sulla realt\`a, ma presto il focus \`e passato al mondo intrapsichico.

\paragraph{La prima generazione} post-Freudiana desidera dimostrare la leggitimit\`a del proprio pensiero, e si paragonano continuamente al maestro, come la Klein che si aggancia alle fantasie.

Gli psicologi dell'Io invece si agganciano all'ultimo modello Freudiano ottenendo la leggitimazione di Anna Freud

\paragraph{La normalit\`a} era descritta da Freud come contrapposta alla patologia; Con Hartmann la normalit\`a diviene oggetto di indagine di per s\'e.

\subsection{Le proposte teoriche}

\begin{itemize}
    \item Focus sulla realt\`a
    \item Pi\`u autonomia all'Es
    \item Pi\`u complessit\`a all'Io
    \item Meno interdipendenza Io-Es
\end{itemize}

\paragraph{Lo Sviluppo} non riguarda il superamento di angosce, ma l'adattamento ad un ambiente ``mediamente prevedibile, non tanto da fantasie inconsce, ma anzi mediamente atteso sulla base dell'evoluzione umana.

\paragraph{L'Io} acquisisce nuove funzioni: mantiene omeostasi ed esercita il controllo centrale. \`E pensato come una variabile parzialmente indipendente dalle pulsioni

\subparagraph{Dal punto di vista dinamico} L'Io diventa pi\`u forte, sviluppa motivazioni proprie indipendenti dall'Es e dal Super-Io. Il mondo esterno acquisisce maggiore influena. 

\subparagraph{Dal punto di vista economico} Attinge all'energia libidica, aggressiva creando energia de-istintualizzata, fino ad arrivare al punto di avere energia autonoma.

\subparagraph{Dal punto di vista topografico} L'io non \`e pi\`u una struttura unitaria che si \`e differenziata dall'Es, ma \`e costituito da unit\`a funzionali ordinate gerarchicamente; Io presente sin dalla nascita (punto di contatto con la Klein)

\subparagraph{Dal punto di vista genetico} L'io ha radici innate come l'Es; molte funzioni dell'Io sono primariamente autonome, non derivando dal conflitto pulsion-realt\`a, mentre altre lo diventano secondariamente.

\paragraph{Ricapitolando}

\begin{itemize}
    \item L'Io e l'Es si sviluppano quindi parallelamente a partire da un serbatoio comune di energia.
    \item Il Principio del Piacere \`e una guida poco sicura per l'autoconservazione, quindi non può precedere il Principio di Realt\`a.
    \item Se l'Io ha un origine in parte indipendente, allora ha anche una certa autonomia rispetto ai conflitti
    \item Quella parte dell'Io che possiede un'autonomia primaria rispetto ai conflitti e alle pulsioni, \`e una parte che svolge funzioni che non di pendono dall'Es
\end{itemize}


\paragraph{La libert\`a dai conflitti} Bisogna anche spiegare cosa succede nella patologia: Hartman sostiene che l'autonomia dell'Io può andare perduta:

\begin{quote}
    \emph{L'indisturbato funzionamento dell'Io dipende in parte da quanto le attivit\`a dell'Io sono al riparo dalla regressione (da quanto non c'\`e patologia che porta alla regressione) e dall'investimento di cariche pulsionali.}
    \begin{flushright}
        Hartman
    \end{flushright}
\end{quote}

Può infatti succedere che le funzioni primarie siano investite dall'Es, portando a \textbf{sessualizzazione} e \textbf{aggressivizzazione}

\paragraph{Neutralizzazione} In Freud solo l'energia sessuale poteva essere deviata dalla sua meta naturale e messa al servizione dell'Io. Secondo Hartman anche l'energia dell'aggressivit\`a può essere deviata.

\paragraph{Il contrasto} Io vs Es non \`e costante, ma anzi l'Io può decidere di aiutare l'Es, Oppure può impedire la realizzazione dei fini pulsionali.

\begin{quote}
    \emph{A volte un sigaro può essere soltanto un sigaro}
    \begin{flushright}
        Rapport
    \end{flushright}
\end{quote}


\paragraph{Il contributo di Kris} Si dedica a un filone ch riguarda la creativit\`a e la psicanalisi, al legame disturbo-creativit\`a.

Per Kris la similarit\`a tra artista e malato di mente \`e la \emph{regressione}, che però nell'artista \`e controllata, al servizio dell'Io.

\subsection{Le implicazioni cliniche}

Nel processo analitico la relazione del paziente con la realt\`a interna ed esterna viene ricostruita, le distorsioni annullate e sostituite con una vision pi\`u realistica: \emph{si spinge il paziente ad abbandonare le fantasie inconsce e ritornare alla realt\`a.}

\begin{quote}
    \emph{``L'interpretazione deve partire da quanto \`e pi\`u vicino possibile all'esperienza conscia del paziente, e chiarire la struttura delle difese, prima di procedere a ciò che proviene dall'Es''}
    \begin{flushright}
        Anna Freud
    \end{flushright}
\end{quote}

Nell'analisi bisogna fare ricorso alle capacit\`a dell'Io: il paziente viene visto come alleato del processo terapeutico.  
\`E quindi necessario costruire un rapporto con il paziente, che deve voler stare meglio.

\subsection{Caso tipico della psicologia dell'Io}

Il paziente era tornato in analisi dopo una prima analisi tradizionale, limitata agli aspetti dell'Es, come i desideri infantili.
Il paziente era un giovane scienziato di circa 30 anni, preoccupato di essere incapace di pubblicare le proprie ricerche. Nella prima analisi, era emerso il suo senso di colpa che gli impediva di essere produttivo. Sentiva la pressione costante di usare le idee altrui, in particolare di un suo giovane amico con cui discuteva per ore. Nella prima nalisi, le intrpretazioni vertevano sul suo ``desiderio di plagio'' come desiderio aggressivo di rubare e divorare le idee altrui, ovvero
aggressivit\`a orale primitiva o invidia kleiniana.

Kris, il secondo analista, decise di approfondire le operazioni difensive, nell'ipotesi che se i problemi di inibizione non erano stati risolti, agissero ancora. Kris iniziò ad esaminare i testi che il paziente temeva di plagiare, approfondendo la sua area di ricerca, e studiò le conversazioni con il suo amico. Un giorno emerse che il paziente, invece che plagiare, aveva introdotto in una discussione con l'amico le idee che l'amico aveva poi usato, sviluppandole e
pubblicandole, senza riconoscergliene la paternit\`a. Il paziente leggendole non le aveva riconosciute come sue. Invece di essere un plagiario, era un ghostwriter.  

L'interpretazione di Kris era che la distorsione difensiva era legata al suo desiderio infantile di ammirare e apprendere da un padre deludente e deluso che era fallito professionalmente. Per crearsi un padre degno, il paziente aveva proiettato le proprie capacit\`a sull'amico pi\`u anziano, trattando il suo lavoro con reverenza e ammirazione.

Emersero anche conflitti edipici successivi, legati alla competizione con l'amico/padre e il desiderio di rubare la sua ``potenza''. Ciò andava punito attraverso l'inibizione e le autoaccuse di furto.

\paragraph{Ricapitolando}
\begin{itemize}
    \item L'interpretazione della seconda analisi forn\`i una maggiore concretezza, e perciò apriva la strada al collegamento tra presente e passato, tra sintomatologia adulta e fantasia infantile
    \item Dal punto di vista dell'Es \`e emerso il desiderio di privare il padre il pene
    \item Dal punto di vista del Super-Io la punizione attraverso l'inibizione al lavoro e il senso di colpa
    \item L'Io mediava tra i due attraverso due difese: la proiezione del suo desiderio e lo spostamento sull'amico del padre

\end{itemize}

\section{Mahler}

\paragraph{Biografia} Nata in Ungheria, \`e una delle ebree costretta a migrare a New York.

Fa osservazioni standardizzate in un laboratorio di interazione madre-bambino normali, \`e una delle prime a farlo.

Essendo una pediatra, fa ricerche sui primi 3 anni di vita, attraverso un paradigma osservativo e psicanalitico. Utilizza setting standardizzati, osserva attraverso specchio, video o dall'interno.
Utilizza anche studi longitudinali.

Teorizza fasi di sviluppo, autistica e simbiotica.

\subsection{La fase autistica}

Il bambino nel ventre materno e poi nelle braccia della madre paragonato ad un pulcino nel guscio. Le prime settimane sono viste in continuit\`a con la vita uterina.

Lo scopo del bambino in questo periodo \`e quello di mantenere l'omeostasi, dalla quale rottura nasce la frustrazione.

Come Freud pensa che il bambino non abbia oggetti.

\paragraph{Il pensiero attuale} sulla fase autistica

\begin{quote}
    \emph{Non esistono barriere agli stimoli, anzi i neonati sono in grado di elaborare aspetti rilevanti dell'ambiente esterno, dalla nascita hanno una preferenza per alcuni stimoli.}
\end{quote}

\paragraph{Nei primi 3 mesi} i neonati hanno una preferenza per l'esplorazione di stimoli perfettamente contingenti che risulatano dalla propria attivit\`a motoria.

Ipotesi: costruzione di una rappresentazione primaria dello schema corporeo

\paragraph{Caso di bambino con psicosi autistica}

\begin{quote}
    Barry, 6 anni, Q.I. 170, ospedalizzato dopo aver tentato di trapanare la testa di un compagno di scuola, per vedere se c'era qualcosa dentro. La madre aveva sofferto di una psicosi \emph{post-partum} e fasi intermpittenti di depressione psicotica, durante le quali metteva Barry nel lettone accanto a s\'e in una stanza semi-buia. Se Barry dava segni di vita, gli lanciava dei libri.

    Il padre nutriva grandi speranze per suo figlio, insegnandogli l'alfabeto, e Barry a 2 anni sorprese tutti citando frasi di ammonimento, lette in riviste, parlando con un vocabolario adulto. Appariva distanziato, letargico, non sembrava entrare in contatto emotivo con gli altri, aveva un linguaggio suo privato.

    Sembrava vivere in un mondo tutto suo, il mondo sotterraneo, composto da simboli personificati. La comunicazione avveniva attraverso un linguaggio gestuale, per esempio la saggezza veniva comunicata attraverso l'abbassamento degli occhi, le emozioni attraverso il cambio del colore della pelle.

    Barry parlava esclusivamente delle persone sotterranee, sulle quali regnava, mostrava emozioni solo se si cercava di strapparlo al mondo sotterraneo. \\
    Nel corso della terapia ripeteva senza emozione ciò che aveva sentito dire da suo padre: ``Mia mamma non mi vuole bene'', e per questo preferiva il mondo sotterraneo.
\end{quote}
    Psicosi autistica per Mahler vuol dire che lui \`e rimasto fissaato a fase autistica normale

\paragraph{Caso di bambino in psicosi simbiotica}

\begin{quote}
    La madre di Steve aveva sviluppato disturbi del sonno durante la gravidanza, temendo che il bambino fosse maschio perch\'e temeva diventasse come il fratello, un delinquente.

    Secondo la madre, Steve dormiva troppo poco, non sopportava i suoi stati di veglia. Appena Steve si svegliava, lo prendeva in braccio e camminava su e gi\'u per la stanza finch\'e non sentiva pi\`u le braccia e non riusciva pi\`u a capire dove finiva lei e dove cominciava lui. 

    A 4 anni la madre, scocciata dalla sua dipendenza, lo mise in un istituto, anche perch\'e la sua masturbazione coatta l'aveva urtata (la masturbazione \`e un atto di indipendenza rispetto alla madre).

    All'osservazione a 8 anni uno dei sintomi di Steve era quello di correre su e gi\`u, chiedendo compulsivamente a chiunque: ``queste sono le mie mani? Queste mani possono uccidere? Io sono tante persone?''
    Sembrava inoltre personificare le diverse persone viste in televisione.

    Non sapeva dove finisse lui e dove iniziassero gli altri, e non era in grado di interagire con le persone.
\end{quote}

\paragraph{Fase pre-oggettuale:} due esseri sono ``racchiusi da una membrana''.

\paragraph{3--4 settimane:} maturazione fisiologica, aumenta la sensibilit\`a agli stimoli, consapevolezza confusa della madre, grazie al fatto che lei funga da Io ausiliario (vedi Bion), la madre \`e pi\`u una funzione che una persona. \`E osservabile il sorriso alla Gestalt umana.

Investimento libidico periferico nell'unit\`a duale madre-bambino, che percepisce la coppia come un'unit\`a confusa e simbiotica.

L'organizzazione dell'esperienza avviene grazie alla memoria e alla distinzione buono-cattivo.

\paragraph{Dal punto di vista economico:} Le fasi autistica e simbiotica corrispondono al narcisismo primario.

Per la Mahler però il narcisismo \`e il rapporto che il bambino ha con la madre quando ancora non la percepisce, quando \`e confuso con lei.

\paragraph{In realt\`a} non \`e vero che i bambini e le madri sono confusi tra loro, esistono meccanismi percettivi precoci di differenziazione.  Esiste percezione amodale (esperimenti ciuccio ruvido e liscio).






























\end{document}
