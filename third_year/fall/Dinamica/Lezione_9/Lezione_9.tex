\documentclass[12pt, a4paper]{article}

\date{24 Ottobre 2019}
\title{Lezione 9}
\author{Dinamica}

\begin{document}

\maketitle

\section{Bion - Parte Seconda}

Bion interpreta i concetti di seno buono e cattivo dicendo che \textbf{seno cattivo diventa seno assente}.

Infatti dall'assenza del seno il bambino trasforma la frustrazione in immagine di seno assente, perch\'e ha gi\`a sperimentato il seno come fonte di latte.
Questo però avviene \textbf{solo se \`e presente la funzione $\alpha$}, altrimenti si attiva il processo Kleinano, ovvero una creazione della falsa presenza del seno cattivo, attraverso IP su oggetti inanimati, creando un ``oggetto bizzarro'', come ad es.\ il grammofono con occhi.

La funzione $\alpha$ viene acquisita attraverso la madre, e negli schizofrenici invece l'angoscia distrugge la mente cosciente prima che questa venga acquisita.

\paragraph{Riassunto} 
\begin{itemize}
    \item $\beta$ intollerabili vengono espulsi attraverso IP
    \item $\alpha$ sono elementi ipotetici di un modello di cui conosciamo input e ouput ma non l'interno. 
    \item La trasformazione \`e un'astrazione dagli elementi $\beta$ agli $\alpha$
\end{itemize}

\subsection{Il ruolo materno}

Nella teoria di Klein la madre \emph{bonifica} l'aggressivit\`a. \\
Nella teoria di Bion la madre \`e un \emph{contenitore} di elementi $\alpha$, che elabora per il bambino gli elementi $\beta$ che lui le proietta addosso.

La madre \emph{digerisce} le sensazioni per il bambino. La madre \`e in grado di identificarsi con il bambino. \\
\medskip
\begin{quote}\emph{Ogni volta che la madre fornisce al bambino un elemento $\alpha$, a questo passa anche un pezzo della sua funzione $\alpha$.} 
\end{quote}
\medskip
Riprende il concetto che la mente sia \textbf{collettiva}. La madre/analista in questo processo \`e in uno stato di \textbf{revenue}, uno stato sognante simbiotico.

Se la madre non digerisce l'angoscia $\beta$, IP sul bambino ancora pi\`u angoscia e questo crea dei problemi.

\paragraph{Modello di Fonagy} vedere dalle slides

\subsection{Modello contenuto-contenitore}

Molti oggetti, pochi tipi di relazioni:  

Elabora un modello di configurazione generale e costante basato sull'IP

\paragraph{Il contenitore} degli elementi $\beta$ del bambino \`e la madre, l'accoppiamento tra funzione $\alpha$ (contenitore) e elementi $\beta$ (contenuto) genera il pensiero

\subsection{Modello Schizoparanoide $\leftrightarrow$ Depressivo}

La capacit\`a di apprendere dipende dalla capacit\`a di tollerare le posizioni Schizoparanoide e Depressiva e l'oscillazione tra loro.

In Klein la fase depressiva era successiva e quindi superiore a quella paranoide, in Bion sono solo diverse.
\begin{itemize}
    \item Nella posizione schizoparanoide predomina la scissione dei dati, il disordine
    \item E in quella depressiva?
\end{itemize}

Finch\'e non si raggiunge un momento in cui le cose si connettono, una \textbf{congiunzione costante} Humiana, il processo reitera astraendo sempre di pi\`u.

Bion pone molta attenzione alla differenza tra vero e falso.

\smallskip
\begin{quote}\emph{Il pensiero fallace nasce dall'angoscia di non sapere}
\end{quote}




















\end{document}
