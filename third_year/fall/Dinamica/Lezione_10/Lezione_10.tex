\documentclass[12pt, a4paper]{article}

\date{28 Ottobre 2019}
\title{Lezione 10}
\author{Dinamica}
\usepackage{soul}

\begin{document}

\maketitle

\section{Bion - Parte Terza} 

Il compito dell'analista \`e ricercare la verit\`a e riuscire a non precluderla al paziente a causa della verit\`a fittizia.

L'analista con fantasie di guarire il paziente sta subendo l'IP di questo che si pensa come qualcosa di rotto, a causa della madre. Le fantasie cessano quando l'analista percepisce paziente con amore.

L'accettazione dell'analista diventa accettazione di s\'e del paziente.

\medskip
\begin{quote}\emph{La diversit\`a ci angoscia per la nostra mancanza di conoscenza, ed \`e questa angoscia che fa s\`i che chiudiamo la nostra conoscenza delle persone attraverso falsit\`a come stereotipi e pregiudizi}
\end{quote}
\medskip

Secondo Bion ``per fare un individuo ce ne vogliono due'': in una relazione analitica/materna due menti si alternano nell'essere contenuto e contenitori.

\subsection{L'analisi}

\paragraph{La realt\`a psichica} \`e infinita, non \`e oggetto dei sensi e non può essere appresa direttamente. 

Possiamo solo ambire ad intuirla, ma non a osservarla. Non essendo vincolata, ma procedendo per associazioni, \`e fonte di pensiero molto pi\`u complesso rispetto al pensiero quotidiano aristotelico e lineare

\paragraph{L'analista} deve affrontare l'analisi senza memoria: non deve sovrapporre una teoria a tutti i costi, e dev'essere senza desiderio: non deve voler guarire il paziente e cambiarlo come lui pensa che sia meglio.

\paragraph{\`E necessario} essere aperti al nuovo, allo sconosciuto e riuscire a tollerare l'angoscia.

\paragraph{L'analista} inoltre ascoltando produce coerenza in modo da poter formulare un'interpretazione, un'ipotesi di sintesi degli elementi non percepiti dal paziente.

Infatti la sua mente (del paziente) non \`e ancora evoluta per coglierli o se la sua mente  non \`e pronta ad accoglierli e contenerli

L'analista deve \textbf{unificare} una massa di fenomeni apparentemente dispersi per mezzo di un'intuizione improvvisa..

\subsection{Lo stato mentale dell'analista}

In Freud questo era l'attenzione libera fluttuante, per Bion:

\begin{itemize}
    \item Evitare memoria e desiderio
    \item Capacit\`a di mettersi nella condizione di ``ciecit\`a del paziente''
    \item Il paziente deve essere considerato nuovo ad ogni seduta
    \item Capacit\`a di tollerare la frustrazione del non capire
\end{itemize}

\begin{quotation}
    \emph{``Che fa lo psicanalista? Egli osserva una massa di elementi conosciute da tempo, ma finch\'e non formula la sua interpretazione, essi sono frammentiati e apparentemente estranei l'uno all'altro. Se riesce a tollerare la posizione depressiva, può formulare l'interpretazione;}

    \emph{L'interpretazione non \`e altro che uno fra gli unici fatti che peritano la sua attenzione, che secondo Poincar\'e, introduce l'ordine in questa comlessit\`a e quindi la rende accessibile. Il paziente \`e aiutato in questo modo a trovare, grazie alle capacit\`a dell'analista di selezionare, uno fra i fatti unificanti.''}
\end{quotation}

\subsection{Delucidazioni}

L'analista deve essere paziente, tollerando la pfrustrazione del non conoscere, mentre osserva una massa di associazioni apparentemente casuali e caotiche.

Tolleranza della non-conoscenza \`e avvicinata da Bion alla capacit\`a negativa di Keats, 
\begin{quote}``\emph{\ldots quella capacit\`a che un uomo possiede se sa preservare nelle incertezze attraverso i misterni e i dubbi, senza lasciarsi andare a una agitata ricerca di fatti e ragioni''}
\end{quote}
\begin{flushright}
    Keats, 1817
\end{flushright}

Lo stato dell'analista \`e uno stato di attesa, credere che esista una coerenza e ambirvi pazientemente, come un pensiero vuoto che attende di essere riempito dal contenuto.

\paragraph{L'obiettivo} dell'analisi \`e la crescita psichica, ovvero l'aumento di dimensioni del contenitore

Il paziente usa invece l'analisi come:
\begin{itemize}
    \item Antidoto contro angosce psicotiche
    \item Rifugio nella patologia per imbrigliare le sue responsabilit\`a
    \item Ricerca di consigli e direttive
\end{itemize}

\subsection{Ricapitolando}

\paragraph{Pazienti psicotici} Per Freud nella psicosi l'Io si pone al servizio dell'Es, ritirandosi dalla realt\`a, dando il via al narcisismo secondario.  
Secondo Klein la psicosi \`e un eccessivo uso delle difese della posizione schizoparanoide, che porta a scissione e a IP.

\subparagraph{L'IP} Viene introdotta da Klein nel '46, come difesa contro angoscia paranoica. \`E una proiezione di parti di s\'e, che porta a percepire gli altri come fossero parti cattive di s\'e stessi.


\section{La Psicologia dell'Io}

\`E l'opposto della Psicologia dell'Es. L'Io contiene il legame con la realt\`a e il processo secondario.

\paragraph{Gli Psicologi dell'Io} hanno:
\begin{itemize}
    \item Proposto la \textbf{prospettiva genetica}, lo sviluppo dei bambini normali
    \item Proposto teorie in cui cercano nessi causali tra passato e presente
    \item Studiato cosa succede nella separazione dagli oggetti primari (genitori)
    \item Introdotto metodi di osservazione nuovi
\end{itemize}

\subsection{Anna Freud}

Studia i meccanismi di difesa come funzioni adattive dell'Io. \\
Sostiene l'importanza dell'osservazione diretta dei bambini per una teoria psicoanalitica dello sviluppo infantile

\paragraph{\`E sopravvissuto} poco della sua teoria. \\ Tra ciò il \emph{microscopio}, l'osservazione dell'angoscia di separazione negli orfani e lo studio dell'Io e dei meccanismi di difesa.

\paragraph{\`E la prima} a tentare di standardizzare e di valutare oggettivamente le difese.

\paragraph{Le difese} che individua sono quelle del padre:

\begin{itemize}
    \item Regressione
    \item Rimozione
    \item Formazione reattiva
    \item Isolamento
    \item Annullamento retroattivo
    \item Proiezione
    \item Introiezione
    \item Rivolgimento contro se stessi
    \item Trasformazione nel contrario
\end{itemize}

Pi\`u alcune definite da lei:

\begin{itemize}
    \item \textbf{Identificazione con l'aggressore}: Per quale motivo un bambino abusato può arrivare ad agire la stessa violenza sui propri figli? Il bambino terrorizzato dall'aggressore si identifica con lui, e cos\`i riesce a non subire passivamente, ma ad assumere un senso di sicurezza del tipo ``ora comando io''. 

        Il soggetto si trasforma cos\`i da minacciato a minacciante, e la paura in sicurezza.
    \item \textbf{Altruismo}: Il dimenticare i propri impulsi in favore degli impulsi e desideri degli altri. Anna Freud non ha mai avuto figli n\'e mariti, mai avuto una propria carriera, si \`e dedicata completamente al padre. 

        \`E il vivere attraverso un'altra persona.
    \item \textbf{Ascetismo}: Il rifiuto generalizzato di qualsiasi desiderio pulsionale, es.\ anoressia. \`E tipico dell'adolescenza, come il prossimo.
    \item \textbf{Intellettualizzazione}: Mentre nella latenza l'interesse del bambino \`e tutto rivolto al reale e al concreto, all'inizio del periodo adolescenziale avviene una trasformazione degli interessi che divengono sempre pi\`u astratti. Rappresenta un tentativo di elaborare mentalmente il conflitto interno attraverso il pensiero
\end{itemize}

\paragraph{Le difese} agiscono sia nella malattia che in modo funzionale, ad esempio quando sono:

\begin{itemize}
    \item Adatte all'et\`a, secondo una successione evolutiva stabile
    \item Flessibili in funzione del contesto
    \item Di intensit\`a moderata
    \item Reversibili e quindi disattivabili
    \item Efficaci nel controllo dell'angoscia
    \item Interferiscono poco con l'esame di realt\`a
\end{itemize}

\paragraph{Lo sviluppo} delle difese secondo Anna Freud segue una linea cronologica connessa con il quadro Es $\rightarrow$ Io $\rightarrow$ Super-Io

\subparagraph{Le difese dell'Es} sono per esempio la regressione proiezione. Si manifestano con la comparsa degli istinti o quando appaiono i primi conflitti tra pulsioni e realt\`a. 

\subparagraph{Le difese dell'Io} sono per esempio la rimozione, che richiede la distinzione dell'Io dall'Es

\subparagraph{Le difese del Super-Io} sono quelle che richiedono distinzione tra impulsi buoni e cattivi, come la \emph{sublimazione}.

Nel mezzo tra Io e Super-Io ci sono le difese come proiezione e introiezione che richiedono la distinzione tra s\'e e altri.

Ovviamente queste teorie sono in contrasto con quelle della Klein, che ritiene che la proiezione e l'introiezione siano le pi\`u primitive.

\paragraph{Un altro modo} per distinguere le difese pi\`u o meno primitive \`e di distinguere difese psicotiche e nevrotiche:

Pi\`u \`e primitivo infatti pi\`u distorce la realt\`a. Le difese nevrotiche sono fissazioni in fasi sessuali previe alla fallica, mentre quelle psicotiche al narcisismo.

Per esempio il dinego \`e una difesa psicotica, pi\`u grave della rimozione, e pi\`u potente.

\begin{quote}
    \emph{``Questa scena fu immediatamente dimenticata [rimozione], con il che prese avvio il processo \ldots che portò ai dolori di natura isterica. \`E questo un caso molto istruttivo perch\'e permette di vedere con precisione per quali vie la nevrosi tenti di liquidare il conflitto. La reazione psicotica sarebbe stata quella di rinnegare invece il fatto stesso della morte della sorella.''}
    \begin{flushright}
        Freud, 1924
    \end{flushright}
\end{quote}

\paragraph{La rimozione} viene quindi vista come difesa matura, rispetto al dinego per lo meno.

\subparagraph{Esempi} di difese:
\begin{itemize}
    \item Detesto mio padre $\rightarrow$ Nessuna difesa
    \item Mio padre mi detesta $\rightarrow$ Proiezione
    \item Detesto il cane di mio padre $\rightarrow$ Spostamento
    \item Detesto il modo di fare affari che ha mio padre $\rightarrow$ Intellettualizzazione
    \item Non capisco perch\'e mi sento agitato $\rightarrow$ Rimozione
    \item Adoro mio padre $\rightarrow$ Formazione reattiva
\end{itemize}

\paragraph{Nota Bene} Per il determinismo psichico Freudiano ogni atto ha sempre un corrispondente psichico

\paragraph{Nota Benissimo} Queste difese sono inconsce, se anche me ne rendo conto questo non \`e sufficiente per smettere (in opposizione alle teorie cognitiviste)

\paragraph{La cura} consiste nel trovare la causa all'angoscia sottostante che mi fa usare la difesa in maniera non funzionale e mi fa perdere contatto con la realt\`a


\subparagraph{Ad esempio} persone abusate si dissociano: vengono picchiate edd escono dal proprio corpo, non sentono il dolore, tipo Sansa Stark. \`E molto grave perch\'e \st{poi non si riesce pi\`u a mandare avanti la serie tv} poi torna in altre situazioni e porta a ``buchi''. Eppure \`e adattivo nel senso che permette di sopportare una realt\`a insopportabile.




















\end{document}
