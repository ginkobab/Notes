\documentclass[12pt, a4paper]{article}

\renewcommand{\labelitemii}{$\star$}

\date{4 Novembre 2019}
\title{Lezione X}
\author{Psicobiologia dei disturbi comportamentali}

\begin{document}

\maketitle

\section{Gli stati di coscienza}

La relazione tra consapevolezza e vigilanza \`e complessa.

\subsection{Lo stato vegetativo}

Ha ricevuto molta attenzione mediatica: non c'\`e assenza di risposte, anzi ci sono dei comportamenti attivi da parte dell'individuo.

L'etica della situazione \`e pi\`u complicata del coma perch\'e non si comprende se c'\`e intenzionalit\`a dietro ai comportamente e quindi se c'\`e volont\`a di rimanere in vita.


\paragraph{La valutazione dello stato di coscienza} \`e necessaria per valutare le implicazioni etiche di queste situazioni.

\paragraph{Le differenze} tra stato vegetativo e coma: capacit\`a di aprire gli occhi. I danni nello stato vegetativo sono estesi e nella materia bianca sottocorticale e dei nucli talamici.

Si pensa che nello stato vegetativo ma la coscienza sia completamente assente

Gli unici movimenti presenti sono elicitati sottocorticalmente, in modo automatico e non risponde agli stimoli, tranne che con gli occhi, che in realt\`a \`e anch'esso automatico.

\paragraph{Il paziente} appare quindi completamente sveglio con cicli di chiusura e apertura occhi.

Ma in realt\`a \`e inconsapevole e privo di risposte volontarie e finalizzate (la chiusura del palmo al contatto con un oggetto non \`e un segno di consapevolezza, essendo organizzato a livello sottocorticale).

Vengono spesso misclassificati (1/3--1/5 delle volte), quando \`e presente uno stato di coscienza minimale.

\paragraph{Lo strumento} per valutare lo stato di coscienza \`e la \textbf{Glasgow Coma Scale}, che valuta almeno tre parametri, sommando poi i risultati dei diversi item, che sono:

\begin{itemize}
    \item Risposta verbale
    \item Apertura degli occhi
    \item Risposta motoria
\end{itemize}

\paragraph{Il problema} \`e quello di trovare strumenti pi\`u precisi, per valutare casi come:

\begin{quote}
    Paziente 23F con danno cerebrale da incidente stradale, mancanza di risposte per 5 mesi, mantenimento ciclo sonno-veglia, fMRI indica risposte corticali appropriate per frasi parlate, \emph{questo significa che c'\`e consapevolezza?}. Potrebbe essere apprendimento implicito, Broca si attiva per elaborare ma non \`e detto che ci riesca n\'e che ci sia consapevolezza.
\end{quote}

\paragraph{Una possibilit\`a} \`e che attraverso compiti di immaginazione mentale si possa provare la coscienza attraverso il segnale BOLD delle aree specifiche attivate.

Dopodich\'e venne chiesto ai pazienti di immaginare di giocare a tennis per esempio quando voleva rispondere s\`i a delle domande, e di immaginare altro per rispondere no.

Questo prova che il paziente non \`e in stato vegetativo.

\paragraph{Sono necessarie} nuove tecnologie per facilitare la diagnosi e la comunicazione di questi pazienti, per indagare i loro desideri (se vogliono rimanere in vita)

\paragraph{Le strutture} correlate con la consapevolezza sono:

\begin{itemize}
    \item Default mode network
    \item Auditory network
    \item Left and Right Executive Control network
\end{itemize}

\subsection{Il coma}

\paragraph{Il coma farmacologico} \`e provvisorio, causato da dosi controllate da farmaci ipnotici come:

\begin{itemize}
    \item barbiturici
    \item benzodiazepine
    \item oppiacei
\end{itemize}

Il barbiturico per esempio riduce il consumo metabolico dei tessuti cerebrali e il flusso sanguigno cerebrale, che riduce la pressione intracranica riducendo l'entit\`a dei disturbi.

\paragraph{Milioni di farfalle} \`e un libro scritto da un neurochirurgo dopo che si \`e risvegliato da uno stato di coma indotto, nel quale descrive un mondo di luce pieno di farfalle colorate, sostenendo che si trattasse dell'aldil\`a. La gente gli ha creduto sulla base del titolo, inoltre ha portato avanti argomenti fisiologici, dicendo per esempio il suo cervello non funzionasse per nulla.

\paragraph{L'inchiesta Esquire} Le dottoresse occupatesi del paziente affermano invece che lo stato del paziente (Alexander) fosse \emph{cosciente ma in stato allucinatorio}. Sostiene che probabilmente le allucinazioni siano avvenute durante la fase di risveglio, e che siano state provocate da attivazione di aree che vengono stimolate dalle percezioni tangibili.

Queste aree sia attivano oltre a una certa soglia di attivit\`a metabolica.

\paragraph{Ciònonostante} Le esperienze pre-morte hanno delle conseguenze profonde sulle credenze degli individui

\paragraph{La morte cerebrale} reale \`e causata da:

\begin{itemize}
    \item Attivit\`a corticale richiede un rifornimento costante di ossigeno e glucosio
    \item Ridotto flusso sanguigno porta ad uno stato di incoscienza
    \item Nei secondi iniziali del ridotto flusso sanguigno si produce una cascata di ``risposte di sopravvivenza''
    \item Sottile linea di confine tra stato di coscienza e incoscienza
    \item La soglia \`e 23 ml/100g per brain tissue/min: si perde coscienza nel giro di 10s
    \item La consapevolezza può ritornare se il flusso sanguigno risale oltre alla soglia
    \item Morte neurale dopo minuti di completa cessazione del flusso sanguigno. 
    \item Non c'\`e un preciso momento in cui avviene, diversi neuroni possono morire in momenti differenti
    \item Anche  con gruppi elevati di neuroni morti a livello talamico e corticale \`e possibile mantenere stati di coscienza minimale.
\end{itemize}

\paragraph{I meccanismi adattivi di difesa} 

\begin{itemize}
    \item Locus coeruleus contribuisce a regolare coscienza tramite un processo di scarica temporizzata
    \item Ipossia, paura, stress forte aumentano il suo tasso di scarica
    \item Se il flusso sanguigno diventa profondamente basso la materia grigia periacquedottale riduce il suo tasso di scarica
    \item Viene introdotto lo stato di coscienza REM
\end{itemize}

\paragraph{La morte cerebrale} si identifica con:

\begin{itemize}
    \item Criteri cardiologici
    \item Criteri respiratori
    \item Criteri neurologici:
        \begin{itemize}
            \item EEG piatto permanente e irreversibile per 30 minuti per 2 volte in 6 ore
            \item Nessuna risposta al dolore
            \item Nessun riflesso veicolato da nervi cranici (e.g./ dilatazione pupille)
        \end{itemize}
\end{itemize}

\paragraph{Le esperienze pre-morte} sono simili tra diverse culture:

\begin{itemize}
    \item Emozioni piacevoli
    \item Pace
    \item Uscita dal corpo
    \item Vista esterna dal corpo
    \item Tunnel scuro con luce brillante
    \item Incontro con esseri luminosi o familiari deceduti
\end{itemize}

\paragraph{Uno studio} ha cercato di capire le cause di queste esperienze, correlandole con variabili fisiologiche. Vengono indagati paziente con arresto cardiaco.

Sono stati trovati 63 soggetti su 2000 con memoria del periodo di incoscienza.

Vengono indagati con la ``near death experience scale'' e correlati con fattori fisiologici, psicologici e trascendentali.

Dato che molti pazienti hanno riportato di poter vedere il corpo dall'alto, hanno montato dei pannelli sul soffitto con dei simboli che potevano essere visti solo dall'alto.

\subparagraph{Risultati} 
\begin{itemize}
    \item 11\% dei pazienti hanno consapevolezza, 6\% con NDE
    \item Probabilmente NDE avviene durante il risveglio
    \item Nessuna differenza tra soggetti con e senza NDE
    \item Ma:
        \begin{itemize}
            \item Tutti i soggetti con NDE erano cattolici
            \item Miglior ossigenazione sanguigna in NDE
        \end{itemize}
\end{itemize}

\paragraph{In un altro studio} su solo 52 pazienti venne indotto un arresto cardaico per verificare funzionamento di defibrillatore interno e nessuno ha avuto NDE\@.

Ha importanza il fatto che i pazienti fossero sicuri che non ci fosse rischio di morte?

Ha importanza l'effetto amnesico dei edativi pre-anestetici?

\paragraph{Le interpretazioni} attuali sono:

\begin{itemize}
    \item Effetto delle aspettative
    \item Effetto dei farmaci
    \item Endorfine
    \item Anossia cerebrale
    \item Ruolo del lobo temporale
    \item Dio
\end{itemize}

\paragraph{Supportate} dal fatto che:

\begin{enumerate}
    \item NDE accadono anche in casi di persone che credono di star morendo anche se clinicamente non c'\`e pericolo
    \item I dettagli di NDE variano con le aspettative
    \item MA il pattern \`e generale
    \item MA i tentati suicidi non vedono l'Inferno
\end{enumerate}


















\end{document}
